\documentclass[11pt]{article}

% packages
\usepackage{physics}
% margin spacing
\usepackage[top=1in, bottom=1in, left=0.5in, right=0.5in]{geometry}
\usepackage{hanging}
\usepackage{amsfonts, amsmath, amssymb, amsthm}
\usepackage{systeme}
\usepackage[none]{hyphenat}
\usepackage{fancyhdr}
\usepackage[nottoc, notlot, notlof]{tocbibind}
\usepackage{graphicx}
\graphicspath{{./images/}}
\usepackage{hyperref}

% colors
\usepackage{xcolor}
\definecolor{p}{HTML}{FFDDDD}
\definecolor{g}{HTML}{D9FFDF}
\definecolor{y}{HTML}{FFFFCF}
\definecolor{b}{HTML}{D9FFFF}
\definecolor{o}{HTML}{FADECB}
%\definecolor{}{HTML}{}

% \highlight[<color>]{<stuff>}
\newcommand{\highlight}[2][p]{\mathchoice%
  {\colorbox{#1}{$\displaystyle#2$}}%
  {\colorbox{#1}{$\textstyle#2$}}%
  {\colorbox{#1}{$\scriptstyle#2$}}%
  {\colorbox{#1}{$\scriptscriptstyle#2$}}}%

% header/footer formatting
\pagestyle{fancy}
\fancyhead{}
\fancyfoot{}
\fancyfoot[R]{\thepage}
\renewcommand{\headrulewidth}{0pt}

% paragraph indentation/spacing
\setlength{\parindent}{0cm}
\setlength{\parskip}{5pt}
\renewcommand{\baselinestretch}{1.25}

\begin{document}
{\huge Sai Sivakumar}

Email: \href{mailto:sivakumars@ufl.edu}{\texttt{sivakumars@ufl.edu}}\hspace{5cm} DoB: July 27\textsuperscript{th}, 2002

Cell: (904)--708--0721\hspace{6.38cm} \href{https://github.com/metalninja27}{\texttt{github.com/metalninja27}} \\

\noindent\makebox[1.3cm][l]{\rule{1cm}{5pt}}{\textbf{Goals}}

Seeking to obtain a B.Sc. in mathematics, then to earn a Ph.D in mathematics; I do not know which specialization it would be in.

\noindent\makebox[1.3cm][l]{\rule{1cm}{5pt}}{\textbf{Education}}

\textsl{August 2020 - present:} Working on B.Sc. in Mathematics with a 3.98 GPA currently, University of Florida (UF).

\textsl{August 2016 - June 2020:} Graduated with IB diploma with a 4.00 GPA, Stanton College Preparatory High School.

%\noindent\makebox[1.3cm][l]{\rule{1cm}{5pt}}{\textbf{Research}}

%[none]

\noindent\makebox[1.3cm][l]{\rule{1cm}{5pt}}{\textbf{Talks and Presentations}}

\textsl{February - March 2022:} To be expected. Will give a series of lectures/talks which briefly outline Fourier analysis on finitely generated abelian groups with some neat results, as they appear in Stein and Shakarchi I.

\textsl{October 2021:} Gave the annual \LaTeX\hspace{1pt} joint seminar with the UF Graduate Mathematics Association. This seminar is designed to demonstrate how \LaTeX\hspace{1pt} works and what it can do, and to encourage mathematics students to learn \LaTeX\hspace{1pt}.

\textsl{June 2021:} Discussed the integral definition of the inverse Laplace transform, as well as how to compute the integral using the residue theorem, at an elementary level. ({\color{blue}\href{https://youtu.be/20Xbrit2chw}{YouTube}})

\textsl{March 2021:} Gave a talk on proving the fundamental theorem of calculus at a highschool/pre-real analysis level. ({\color{blue}\href{https://youtu.be/l4GO-n-2ETQ}{YouTube}})

%\noindent\makebox[1.3cm][l]{\rule{1cm}{5pt}}{\textbf{Experience}}

\noindent\makebox[1.3cm][l]{\rule{1cm}{5pt}}{\textbf{Skills}}

3+ years of \LaTeX\hspace{1pt} experience (high proficiency).

Proficiency in Java, C\verb!++!, and understanding of data structures and algorithms. 

\noindent\makebox[1.3cm][l]{\rule{1cm}{5pt}}{\textbf{Outreach/Service}}


\textsl{September 2021 - present:} Member of the Algebra seminar group.

\textsl{August 2021 - present:} Teaching assistant for \textsl{MAP2302} Elementary Differential Equations.

\textsl{August 2021 - present:} Member of the Association for Women in Mathematics' UF chapter.

\textsl{August 2021 - present:} Academic Director of the University Math Society at UF. I schedule all talks from professors and give talks myself, as well as encouraging other students to give talks as well.

\textsl{March 2021 - present:} Moderator for a large online community (exceeding 75,000 members globally) which seeks to stimulate mathematical discussion and interest, as well as to provide assistance with math problems/concepts.

\textsl{August 2020 - December 2020:} Typed up many solutions for \textsl{Concepts in Calculus
III} by Miklos Bona and Sergei Shabanov (around 47 pages or so, working with two others to form in total 141 pages of solutions compiled in a solution manual).

\textsl{August 2019 - February 2020:} Started a small unofficial mathematics club (in highschool) where students presented on topics of mathematical interest; there I gave three informal talks.

\noindent\makebox[1.3cm][l]{\rule{1cm}{5pt}}{\textbf{Honors/Awards}}

\textsl{Fall 2020, Spring 2021, Summer 2021} Dean's list.

\textsl{2020} National Merit Scholarship Commended

\textsl{2020} National AP Scholar

\noindent\makebox[1.3cm][l]{\rule{1cm}{5pt}}{\textbf{Relevant Coursework}}

\textsl{From most recent to earliest, and items marked by a \textsuperscript{\textdagger} are graduate or mixed graduate/undergraduate level courses:}

\textsl{MTG4303\textsuperscript{\textdagger}:} Introductory Topology II -- (Self-studied material found in the first semester before enrollment in this course.) Second semester of introductory topology, covering basic algebraic topology and more topics in point-set topology. Chapters 5-6, 9-12 Munkres

\textsl{MAA4212:} Advanced Calculus II -- Second semester of introductory real analysis, covering analysis in metric spaces and theory of functions of several real variables. Professor's notes. Spring 2022

\textsl{MAP4341\textsuperscript{\textdagger}:} Introduction to Partial Differential Equations -- Elementary theory of solving partial differential equations. Professor's notes and lectures. Spring 2022

\textsl{MAT4930\textsuperscript{\textdagger}:} Introduction to Algebra II -- Second semester graduate level algebra; covering rings, fields, modules. Chapters 7-13 of Dummit and Foote. Spring 2022

\textsl{MAA4211:} Advanced Calculus I -- First semester of introductory real analysis. Chapters 1-7 of Abbott. Fall 2021

\textsl{MAS4413:} Fourier Analysis -- Elementary theory of Fourier analysis. Chapters 1-7 of Stein and Shakarchi I. Fall 2021

\textsl{MAT4930\textsuperscript{\textdagger}:} Introduction to Algebra I -- First semester graduate level algebra; covering group theory. Chapters 1-6 from Dummit and Foote. Fall 2021

\textsl{MAP4305:} Ordinary Differential Equations -- Second course in ordinary differential equations. Covered methods of using matrices for systems of linear ODEs, the method of Frobenius for second order ODEs, solving regular Sturm-Liouville boundary value problems, and using Green's functions. Professor's lectures. Summer 2021

\textsl{MAA4402:} Introductory Complex Analysis -- Elementary theory of functions of a complex variable. Chapters 1-7 of Brown and Churchill. Spring 2021

\textsl{MAS4105:} Introductory Linear Algebra -- Proof-based linear algebra. Chapters 1-6 of Friedberg, Insel, Spence. Spring 2021

\textsl{MAS4203:} Introductory Number Theory -- Elementary concepts in number theory. Chapters 1-3 of Niven and Zuckerberg. Spring 2021

\textsl{MAS4301:} Introductory Abstract Algebra -- Elementary group theory. Chapters 1-11 of Gallian. Spring 2021

\textsl{MAC3474:} Honors Calculus III -- Basic multivariable calculus. Chapters 1-5 of \textsl{Concepts in Calculus III} by Miklos Bona and Sergei Shabanov. Fall 2020

\textsl{MAP2302:} Honors Elementary Differential Equations -- Covered how to solve various basic ODEs, basic notions of existence and uniqueness, and applications to physics. Chapters 1-8 in Nagle Saff Snider 7th edition. Fall 2020

\textsl{MHF3202:} Sets and Logic -- Taught elementary set theory and how to write basic proofs. Chapters 1,2,3, 5-10, 12, 14 in Book of Proof by Richard Hammack. Fall 2020
\end{document}