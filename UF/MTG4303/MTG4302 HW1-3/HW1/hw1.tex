\documentclass[11pt]{article}
\headheight = 14pt
% packages
\usepackage{physics}
% margin spacing
\usepackage[top=1in, bottom=1in, left=0.5in, right=0.5in]{geometry}
\usepackage{hanging}
\usepackage{amsfonts, amsmath, amssymb, amsthm}
\usepackage{systeme}
\usepackage[none]{hyphenat}
\usepackage{fancyhdr}
\usepackage[nottoc, notlot, notlof]{tocbibind}
\usepackage{graphicx}
\graphicspath{{./images/}}
\usepackage{float}
\usepackage{siunitx}
\usepackage{esint}
\usepackage{cancel}
\usepackage{enumitem}

% colors
\usepackage{xcolor}
\definecolor{p}{HTML}{FFDDDD}
\definecolor{g}{HTML}{D9FFDF}
\definecolor{y}{HTML}{FFFFCF}
\definecolor{b}{HTML}{D9FFFF}
\definecolor{o}{HTML}{FADECB}
%\definecolor{}{HTML}{}

% \highlight[<color>]{<stuff>}
\newcommand{\highlight}[2][p]{\mathchoice%
  {\colorbox{#1}{$\displaystyle#2$}}%
  {\colorbox{#1}{$\textstyle#2$}}%
  {\colorbox{#1}{$\scriptstyle#2$}}%
  {\colorbox{#1}{$\scriptscriptstyle#2$}}}%

% header/footer formatting
\pagestyle{fancy}
\fancyhead{}
\fancyfoot{}
\fancyhead[L]{MTG4302}
\fancyhead[C]{HW 1}
\fancyhead[R]{Sai Sivakumar}
\fancyfoot[R]{\thepage}
\renewcommand{\headrulewidth}{1pt}

% paragraph indentation/spacing
\setlength{\parindent}{0cm}
\setlength{\parskip}{5pt}
\renewcommand{\baselinestretch}{1.25}

% extra commands defined here
\newcommand{\br}[1]{\left(#1\right)}
\newcommand{\sbr}[1]{\left[#1\right]}
\newcommand{\cbr}[1]{\left\{#1\right\}}

% bracket notation for inner product
\usepackage{mathtools}

\DeclarePairedDelimiterX{\abr}[1]{\langle}{\rangle}{#1}

\DeclareMathOperator{\Span}{span}
\DeclareMathOperator{\card}{card}
\DeclareMathOperator{\Int}{Int}

% set page count index to begin from 1
\setcounter{page}{1}

\begin{document}
\begin{enumerate}
    \item Prove that for any sets $A$ and $B$, \[A\cup (B\cap C) = (A\cup B)\cap (A\cup C).\] \begin{proof}
        For any $x$, the statement $x\in A\cup (B\cap C)$ is written using logical symbols as $x\in A \lor (x\in B \land x\in C)$.
        
        Observe that the truth table for this statement coincides with the truth table for the statement $(x\in A \lor x\in B)\land (x\in A \lor x\in C)$, which is equivalent to saying $x\in (A\cup B)\cap (A\cup C)$:
        \begin{table}[h]
            \centering
            \begin{tabular}{c|c|c|c|c}
                $x\in A$ & $x\in B$ & $x\in C$ & $x\in A \lor (x\in B \land x\in C)$ & $(x\in A \lor x\in B)\land (x\in A \lor x\in C)$\\
                \hline
                $T$ & $T$ & $T$ & $T$ & $T$\\
                $T$ & $T$ & $F$ & $T$ & $T$\\
                $T$ & $F$ & $T$ & $T$ & $T$\\
                $T$ & $F$ & $F$ & $T$ & $T$\\
                $F$ & $T$ & $T$ & $T$ & $T$\\
                $F$ & $T$ & $F$ & $F$ & $F$\\
                $F$ & $F$ & $T$ & $F$ & $F$\\
                $F$ & $F$ & $F$ & $F$ & $F$\\
            \end{tabular}
        \end{table}

        Hence $x\in A\cup (B\cap C)$ is logically equivalent to $x\in (A\cup B)\cap (A\cup C)$, and because $x$ was arbitrary the two sets $A\cup (B\cap C)$ and $(A\cup B)\cap (A\cup C)$ are mutually included in each other and hence equal.
    \end{proof}
    \item Determine if the following statements are true or false. \begin{enumerate}[label=(\alph*)]
        \item For $f\colon A\to B$ and $A_0, A_1\subset A$, \[f(A_0\cup A_1) = f(A_0)\cup f(A_1).\]
        \textbf{TRUE} \begin{proof}
            We will show that both sets are mutually included in each other. In the case that $A_0$ or $A_1$ is empty, the statement holds trivially, since the image of the empty set under $f$ is the empty set. So suppose $A_0, A_1$ are nonempty.

            Then let $x\in f(A_0\cup A_1)$, so that there exists $a\in A_0\cup A_1$ such that $f(a) = x$. Thus either $a\in A_0$ or $a\in A_1$ (or both), so that $x = f(a) \in f(A_0)\cup f(A_1)$, and since $x$ was arbitrary, $f(A_0\cup A_1) \subset f(A_0)\cup f(A_1)$. 

            Then let $x\in f(A_0)\cup f(A_1)$, so that $x\in f(A_0)$ or $x\in f(A_1)$ (or both). So either there exists $a_0\in A_0$ such that $x = f(a_0)$ or there exists $a_1\in A_1$ such that $x = f(a_1)$ (or both), and since $a_0,a_1\in A_0\cup A_1$, it follows that $x\in f(A_0\cup A_1)$. Since $x$ was arbitrary the reverse inclusion $f(A_0\cup A_1) \supset f(A_0)\cup f(A_1)$ holds.

            Hence $f(A_0\cup A_1) = f(A_0)\cup f(A_1)$.
        \end{proof}
        \item For any sets $A$ and $B$, \[A-(B-A) = A-B.\]
        \textbf{TRUE} \begin{proof}
            The set $A-(B-A)$ is given by $\cbr{s\mid s\in A \land s\not\in(B-A)}$. Decompose the statement $s\not\in (B-A)$ into $\lnot (s\in B \land s\not\in A)$, which is equivalent to $s\not\in B \lor s\in A$.

            Then $s\in A \land s\not\in(B-A)$ is logically equivalent to $s\in A \land (s\not\in B \lor s\in A)$, which is equivalent to $(s\in A \land s\not\in B) \lor (s\in A \land s\in A)$. This last statement is logically equivalent to $s\in A \land s\not\in B$, and so the set $A-(B-A)$ can be written as $\cbr{s\mid s\in A \land s\not\in B}$, which is exactly the definition of $A-B$. 

            Hence $A-(B-A) = A-B$.
        \end{proof}
        \item For any sets $V_1, W_1, V_2, W_2$, \[(V_1\times W_1)\cup (V_2\times W_2) = (V_1\cup V_2)\times (W_1\cup W_2).\]
        \textbf{FALSE} \begin{proof}
            When $V_1, W_1, V_2, W_2$ are nonempty, let $v_1\in V_1$ and $w_2\in W_2$ be given. Then the ordered pair $(v_1,w_2)$ is an element of $(V_1\cup V_2)\times (W_1\cup W_2)$ but cannot be an element of $(V_1\times W_1)\cup (V_2\times W_2)$. In this case we have $(V_1\times W_1)\cup (V_2\times W_2) \subsetneq (V_1\cup V_2)\times (W_1\cup W_2)$.
        \end{proof}
        \item If $A\times B$ is finite then both $A$ and $B$ are finite.
        \textbf{FALSE} \begin{proof}
            Taking $A$ to be an infinite set and $B$ to be the empty set, the cartesian product $A\times B$ is also the empty set because there are no elements $b\in B$ which we could use to form an element $(a,b) \in A\times B$ for any element $a\in A$.

            Hence $A\times B$ is finite while $A$ is infinite.
        \end{proof}
        \item If $A$ and $B$ are nonempty and $A\times B$ is finite then both $A$ and $B$ are finite.
        \textbf{TRUE} \begin{proof}
            Let $A, B$ be nonempty sets. Then we prove the contrapositive. To that end, suppose either $A$ or $B$ is an infinite set. Without loss of generality let $A$ be an infinite set. Note that since $A,B$ are nonempty, $A\times B$ is nonempty.
            
            Choose one $b\in B$ (the set $B$ is nonempty). Then there are an infinite number of elements of the form $(a,b)$ for $a\in A$ in $A\times B$  since the number of options for $a$ is not finite, as $A$ is an infinite set.

            Thus $A\times B$ is necessarily an infinite set. By the contrapositive, when $A,B$ are nonempty, if $A\times B$ is finite then both $A$ and $B$ are finite.
        \end{proof}
    \end{enumerate}
    \item Suppose that $X$ is a set and $\cbr{A_\alpha\colon \alpha \in I}$ is an indexed family of sets (which contains at least one set). Prove that \[X-\bigcup_{\alpha\in I}A_{\alpha} = \bigcap_{\alpha\in I}(X-A_\alpha).\]
    \begin{proof}
        For any $s$, the statement $s\in X-\bigcup_{\alpha\in I} A_\alpha$ is equivalently written as $s\in X \land (s\not\in \bigcup_{\alpha\in I}A_\alpha)$, which is equivalent to \[s\in X \land \sbr{\bigwedge_{\alpha \in I}s\not\in A_{\alpha}}.\] Distributing, we have that $\bigwedge_{\alpha \in I}(s\in X \land s\not\in A_{\alpha})$, which is equivalent to saying that $s\in \bigcap_{\alpha\in I}(X-A_\alpha)$.

        Hence $X-\bigcup_{\alpha\in I}A_{\alpha} = \bigcap_{\alpha\in I}(X-A_\alpha)$.
    \end{proof}
    \item Define a relation $S$ on the set of real numbers $\mathbb{R}$ by \[S = \cbr{(a,b)\in \mathbb{R}\times \mathbb{R}\colon a-b \text{ is an integer}}.\] Prove that $S$ is an equivalence relation on $\mathbb{R}$.
    \begin{proof}
        Reflexivity: Let $x\in \mathbb{R}$. Then $x-x = 0\in \mathbb{Z}$ so that $(x,x)\in S$. Hence $S$ is reflexive.

        Symmetry: Let $x,y\in \mathbb{R}$, and suppose that $(x,y)\in S$; that is, $x-y\in\mathbb{Z}$. Then observe that $y-x = -(x-y)$, and since $x-y\in\mathbb{Z}$, then $-(x-y) = y-x\in\mathbb{Z}$ also since $\mathbb{Z}$ is closed under addition (and subtraction). Hence $(y,x)\in S$, and so $S$ is symmetric.

        Transitivity: Let $x,y,z\in \mathbb{Z}$ with $(x,y), (y,z)\in S$ so that $x-y, y-z\in \mathbb{Z}$. Then observe that $x-z = (x-y)+(y-z)$ and again by closure of addition in $\mathbb{Z}$, $x-z\in \mathbb{Z}$ so that $(x,z)\in \mathbb{Z}$. Hence $S$ is transitive also. 

        It follows that $S$ is an equivalence relation on $\mathbb{R}$.
    \end{proof}
    \item Suppose that $<_A$ is a strict linear order on $A$, and $<_B$ is a strict linear order on $B$. Prove that the dictionary order relation on $A\times B$ is a strict linear order on $A\times B$.
    \begin{proof}
        Let $(a_1,b_1), (a_2, b_2), (a_3, b_3)$ be arbitrary elements of $ A\times B$, and endow $A\times B$ with the dictionary order $<$.

        Comparability: Suppose that $(a_1, b_1)\neq (a_2, b_2)$. We show that either $(a_1, b_1) < (a_2, b_2)$ (so that either $a_1 <_A a_2$ or $a_1 = a_2\land b_1<_B b_2$), or $(a_2, b_2) < (a_1, b_1)$ (so that either $a_2 <_A a_1$ or $a_2 = a_1\land b_2<_B b_1$).
        
        This yields three cases: $a_1\neq a_2$, or $b_1\neq b_2$, or both. In the cases where $a_1\neq a_2$ or both $a_1\neq a_2$ and $b_1\neq b_2$, either $a_1<_A a_2$ or $a_2 <_A a_1$ so that from the definition of the dictionary order on $A\times B$ given comparability holds. When $b_1\neq b_2$ (and $a_1 = a_2$), again also either $b_1 <_B b_2$ or $b_2<_B b_1$ so that comparability follows from the definition of the dictionary order on $A\times B$. Hence comparability holds in general for the dictionary order on $A\times B$.

        Reflexivity never holds: For some $(a_1, b_1)$ if we suppose by way of contradiction that $(a_1, b_1) < (a_1, b_1)$, then by the definition of the dictionary order on $A\times B$, either $a_1<_A a_1$ or $a_1 = a_1\land b_1 <_B b_1$, which is always in contradiction to the consequences of the total orders on $A$ and $B$. Hence by contradiction reflexivity never holds for the dictionary order on $A\times B$.

        Transitivity: Suppose $(a_1, b_1) < (a_2, b_2)$, so that $a_1 <_A a_2$ or $a_1 = a_2\land b_1 <_B b_2$; suppose also that $(a_2, b_2) < (a_3, b_3)$, so that $a_2 <_A a_3$ or $a_2 = a_3\land b_2<_Bb_3$. These statements yield four cases, each of which imply transitivity: \begin{enumerate}[label=(\arabic*)]
            \item Suppose that $a_1 <_A a_2$ and $a_2 <_A a_3$, so that due to the total ordering on $A$, $a_1 <_A a_3$. It follows that $(a_1,b_1) < (a_3, b_3)$.
            \item Suppose that $a_1 <_A a_2$ and $a_2 = a_3\land b_2<_Bb_3$, which again from the total ordering on $A$ imply that $a_1 <_A a_3$. It follows that $(a_1,b_1) < (a_3, b_3)$.
            \item Suppose that $a_1 = a_2\land b_1 <_B b_2$ and $a_2 <_A a_3$, so that $a_1 <_A a_3$. Again $(a_1,b_1) < (a_3, b_3)$.
            \item Suppose that $a_1 = a_2\land b_1 <_B b_2$ and $a_2 = a_3\land b_2<_Bb_3$, so that by the total ordering on $B$, $b_1 <_B b_3$. It also follows that $a_1 = a_3$, and so it follows that $(a_1,b_1) < (a_3, b_3)$.
        \end{enumerate} Thus the dictionary order on $A\times B$ is transitive.

        Hence the dictionary order on $A\times B$ is a total ordering.
    \end{proof}
    \item Suppose that $<_A$ is a strict linear order on $A$, and $<_B$ is a strict linear order on $B$ and there exists a bijection $\phi\colon A\to B$ so that \[a <_A a^{\prime} \text{ implies } \phi(a) <_B \phi(a^{\prime}).\] Show that \[\phi(a) <_B \phi(a^{\prime}) \text{ implies } a <_A a^{\prime}.\]
    \begin{proof}
        We prove the contrapositive. Suppose that $a <_A a^{\prime}$ is false; that is, either $a = a^{\prime}$ or $a^{\prime}<_A a$. Since $\phi$ is a bijection which preserves order in one direction, it follows that either $\phi(a) = \phi(a^{\prime})$ or $\phi(a^{\prime}) <_B \phi(a)$. This is equivalent to saying that $\phi(a) <_B \phi(a^{\prime})$ is false, so by the contrapositive the statement \[\phi(a) <_B \phi(a^{\prime}) \text{ implies } a <_A a^{\prime}\] is true.
    \end{proof}
    \item Assuming that the set of real numbers $\mathbb{R}$ has the least upper bound property, show that it has the greatest lower bound property.
    \begin{proof}
        Suppose that $\mathbb{R}$ has the least upper bound property. Then take any nonempty subset $A$ of $\mathbb{R}$ which is bounded below, and let $\alpha\in \mathbb{R}$ be one such lower bound. So the set of lower bounds of $A$, call it $L$, is a nonempty subset of $\mathbb{R}$ as it contains $\alpha$; furthermore $L$ is bounded above since $A$ is nonempty and contains some $a\in\mathbb{R}$ such that for any lower bound $\gamma\in \mathbb{R}$ of $A$, $\gamma\leq a$.

        Because $\mathbb{R}$ has the least upper bound property, the set $L$ has a least upper bound in $\mathbb{R}$; without loss of generality let $\alpha$ be the least upper bound of $L$. We claim that $\alpha$ is the greatest lower bound of $A$.

        Let $\varepsilon> 0$ be given. Then $\alpha-\varepsilon < \alpha$, and it follows that $\alpha-\varepsilon$ is not an upper bound of $L$ and is also not in $A$ since there exists some $\ell\in L$ such that $\alpha-\varepsilon < \ell \leq a$ for any $a\in A$. Then since $\alpha - \varepsilon< \ell \leq \alpha < \alpha + \varepsilon$ for any $\varepsilon > 0$, it follows that $\alpha \in L$ ($\ell$ depends on $\varepsilon$, and as $\varepsilon$ is taken to be as small as we wish then $\ell$ must eventually take on the value of $\alpha$, and $\ell$ is always an element of $L$). So $\alpha$ is indeed a lower bound for $A$.

        Then $\alpha$ is also the greatest lower bound for $A$. For given $\varepsilon >0$, $\alpha + \varepsilon$ is no longer a lower bound for $A$ since $\alpha$ was the lowest upper bound for $L$; that is, $\alpha + \varepsilon \not\in L$. Hence $\alpha$ is necessarily the greatest lower bound of $A$.
    \end{proof}
    \item Suppose that $A$ and $B$ are sets. Suppose that $A$ is countable, and there is a surjective function $f\colon A\to B$. Prove that $B$ is countable.
    \begin{proof}
        Because $A$ is countable, it is either finite or countably infinite. In the case that $A$ is the empty set, the image of $A$ under $f$ (which is $B$ since $f$ is surjective) is also the empty set so that $B$ is the empty set, which is finite. In this case $B$ is countable.

        So let $A$ be nonempty and countable, so that there exists a surjective map $g\colon \mathbb{Z}_+ \to A$. Then the composition $f\circ g\colon \mathbb{Z}_+\to B$ is surjective since the composition of surjective maps is surjective, and so in all cases $B$ is countable.
    \end{proof}
    \item Prove that the set of rational numbers is countable.
    \begin{proof}
        Since $\mathbb{Z}\times \mathbb{Z}$ is countable (as finite products of countable sets is countable, and $\mathbb{Z}$ is countable), there exists a surjection $g\colon \mathbb{Z}_+\to \mathbb{Z}\times\mathbb{Z}$. Then take the surjection $f\colon \mathbb{Z}\times\mathbb{Z}\to \mathbb{Q}$ given by $f((n,m)) = n/m$, and compose it with $g$ to form the composite map $f\circ g\colon \mathbb{Z}_+ \to \mathbb{Q}$, which is surjective because the composition of two surjective functions is surjective. Hence $\mathbb{Q}$ is countable.
    \end{proof}
    \item A real number is said to be algebraic if and only if it satisfies some polynomial equation of positive degree of the form \[a_nx^n + a_{n-1}x^{n-1} + \cdots + a_1x + a_0 = 0,\] where each $a_i$ is an integer. Assuming that a degree $n$ polynomial has at most $n$ distinct roots, prove that the set of all algebraic numbers is countable.
    \begin{proof}
        Since a countable union of countable sets is countable, we consider polynomials of degree $n$ for each $n\in \mathbb{Z}_+$, and determine that the set of roots of polynomials of degree $n$ is countable for each $n\in\mathbb{Z}_+$.

        For any given positive integer $n$, we first determine the cardinality of the set of polynomials of degree $n$. A polynomial of degree $n$ evaluated at $x\in\mathbb{R}$ is of the form \[a_nx^n + a_{n-1}x^{n-1} + \cdots + a_1x + a_0\] where each $a_i\in \mathbb{Z}$ (though we may want to say that $a_n\neq 0$ since the polynomial is of degree $n$ specifically). Polynomials of degree $n$ are completely determined in the natural way by the $(n+1)$-tuples $(a_n, \dots, a_0)\in \mathbb{Z}^{n+1}$, and because $\mathbb{Z}^{n+1}$ is a finite product of the countable set $\mathbb{Z}$, it is also countable. Thus there are countably many polynomials of degree $n$ for each positive integer $n$.

        Since every polynomial of degree $n$ has at most $n$ roots, it follows from the finiteness of $n$ that the cardinality of the set of roots of polynomials of degree $n$ is countable: Each polynomial of degree $n$ corresponds to a finite set of at most $n$ real numbers, and we can take the countable union of these finite sets (since there are countably many polynomials of degree $n$). Thus there are countably many real numbers which are roots of polynomials of degree $n$.

        Then take the countable union over every positive integer $n$ of each of these countable sets of real numbers which are roots of polynomials of degree $n$, and so it follows that the set of algebraic numbers is countable.
    \end{proof}
\end{enumerate}
\end{document}