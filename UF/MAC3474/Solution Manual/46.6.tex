\documentclass{article}
\usepackage[utf8]{inputenc}
\usepackage{amsmath}

\title{Solution Manual}
\author{N. Kapsos, M. Schrank, S. Sivakumar}
\date{}

\begin{document}

\maketitle
\setcounter{secnumdepth}{0}

\section{46.6 Exercises}

\subsection{46.6.2}

The divergence of a vector field can be found by taking the formal dot product $$\nabla \cdot \vec{F} = \left\langle \frac{\partial}{\partial x}, \frac{\partial}{\partial y} ,\frac{\partial}{\partial z} \right\rangle \cdot \vec{F}$$

Perform this computation for the vector field $\vec{F}$. Rewrite the vector field as $\vec{F} = \frac{1}{||\vec{r}||}\vec{r} = \frac{1}{\sqrt{x^2+y^2+z^2}}\langle x,y,z\rangle$:
$$\nabla \cdot \vec{F} = \left\langle \frac{\partial}{\partial x}, \frac{\partial}{\partial y} ,\frac{\partial}{\partial z} \right\rangle \cdot \left\langle \frac{x}{\sqrt{x^2+y^2+z^2}},\frac{y}{\sqrt{x^2+y^2+z^2}},\frac{z}{\sqrt{x^2+y^2+z^2}}\right\rangle$$
$$\to \frac{y^2+z^2}{(x^2+y^2+z^2)^{\frac{3}{2}}}+\frac{x^2+z^2}{(x^2+y^2+z^2)^{\frac{3}{2}}}+\frac{x^2+y^2}{(x^2+y^2+z^2)^{\frac{3}{2}}}$$
$$\to \frac{2}{\sqrt{x^2+y^2+z^2}}=\frac{2}{r}$$

\subsection{46.6.13}

Apply the other form of Green's theorem:
$$\oint_{\partial D}\vec{F}\cdot \hat{n}ds = \iint_D \nabla\cdot\vec{F}dA$$

We should find that (give $D$ as the planar region bounded by $C$):
$$\oint_C \vec{a}\cdot\hat{n}ds = \iint_D \nabla\cdot\vec{a} dA = 0$$

Note that the vector $\vec{a}$ is a constant vector. The divergence of the vector field given by $\vec{F}=\vec{a}$ is zero, since derivatives of constants vanish. Thus the double integral vanishes as shown above.

\subsection{46.6.18}

For this it may be easier to compute the flux using the divergence theorem, instead of taking the surface integral. Give $D$ as the rectangular region bounded by $S$. Then compute the divergence of the vector field $\vec{F} = \langle x^2,y^2,z^2\rangle$:
$$\nabla \cdot \vec{F} = \left\langle \frac{\partial}{\partial x}, \frac{\partial}{\partial y} ,\frac{\partial}{\partial z} \right\rangle \cdot  \langle x^2,y^2,z^2\rangle  = 2x+2y+2z$$

From the divergence theorem we know that: $$\iint_S \vec{F}\cdot \hat{n}dS = \iiint_D (2x+2y+2z) dV$$

Because the region $D$ is a rectangular prism, apply Fubini's theorem directly to find the flux:
$$\iiint_D (2x+2y+2z) dV = \int_0^c\int_0^b\int_0^a (2x+2y+2z)dxdydz = a^2bc+ab^2c+abc^2$$

Because the flux is positive, we have a faucet within the region.

\subsection{46.6.21}

Give $D$ as the unit ball bounded by the $S$, and apply the divergence theorem. We seek to find the divergence of the vector field first:
$$\nabla\cdot\vec{F} = \left\langle \frac{\partial}{\partial x}, \frac{\partial}{\partial y} ,\frac{\partial}{\partial z} \right\rangle \cdot \langle -xy^2, -yz^2, zx^2 \rangle= -y^2 -z^2 + x^2$$

The statement of the divergence theorem uses an outwardly oriented surface. In this problem the surface $S$ is oriented inwards, so we must adjust our use of the divergence theorem to reflect this. Notice that we may form the outward orientation of $S$ by simply negating the unit normal $\hat{n}$, and so turn the surface integral into one that we can apply the divergence theorem on. The integral becomes:
$$\iint_S \vec{F}\cdot (-1)\hat{n}dS = \iiint_D (-y^2 -z^2 + x^2)dV$$
$$\to \iint_S \vec{F}\cdot \hat{n}dS = -\iiint_D (-y^2 -z^2 + x^2)dV$$

Make a change of variables into spherical coordinates where the $x$ axis is the vertical axis. This means that $x = \rho\cos(\phi)$, and $y^2+z^2=r^2 = \rho^2\sin^2(\phi)$. Since the whole ball is the region of integration, all parameters take on their natural ranges (and $\rho$ ranges from $0$ to $1$).
$$-\iiint_D (-y^2 -z^2 + x^2)dV $$
$$\to -\int_0^{2\pi}\int_0^{\pi}\int_0^1(-\rho^2\sin^2(\phi) + \rho^2\cos^2(\phi))(\rho^2\sin(\phi))d\rho d\phi d\theta$$
$$\to -2\pi\int_0^{\pi}\int_0^1(\rho^4(2\cos^2(\phi)-1))\sin(\phi))d\rho d\phi$$
$$ \to -\frac{2\pi}{5}\int_0^{\pi}(2\cos^2(\phi)-1))\sin(\phi)d\phi = \frac{4\pi}{15}$$

Because the surface $S$ was oriented inwards and we found a positive flux, the interpretation is that the vector field seems to be converging somewhere within the sphere. Thus we have a sink.

\subsection{46.6.22}

Give $D$ as the cylindrical region within the cylindrical surface $S$. It is quickly apparent that we can use the divergence theorem directly. First compute the divergence of the vector field $\vec{F} = \langle xy,z^2y,zx\rangle$:
$$\nabla \cdot \vec{F} = \left\langle \frac{\partial}{\partial x}, \frac{\partial}{\partial y} ,\frac{\partial}{\partial z} \right\rangle \cdot  \langle xy,z^2y,zx\rangle  = y+z^2+x$$

Then the integral becomes:
$$\iint_S \vec{F}\cdot \hat{n}dS = \iiint_D \nabla \cdot \vec{F}dV \to \iiint_D (y+z^2+x)dV$$

Use cylindrical coordinates where $-2\leq z \leq 2$, $0\leq r \leq 2$, and $\theta$ takes on its natural range:
$$\iiint_D (y+z^2+x)dV \to \int_0^{2\pi}\int_0^2\int_{-2}^2 (r\sin(\theta)+z^2+r\cos(\theta))dz(r)drd\theta$$
$$\to \int_0^{2\pi}\int_0^2 \left(4r^2\sin(\theta)+4r^2\cos(\theta)+\frac{16}{3}r\right)drd\theta$$
$$\to \int_0^{2\pi} \frac{32}{3}(\sin(\theta)+\cos(\theta)+1) = \frac{64}{3}\pi$$

Since we have a positive flux on the outwardly oriented surface $S$, we have a faucet.

\subsection{46.6.23}

Give $D$ as the solid region bounded by $S$. Then compute the divergence of the vector field $\vec{F} = \langle xz^2,\frac{y^3}{3},zy^2+xy\rangle$:
$$\nabla \cdot \vec{F} = \left\langle \frac{\partial}{\partial x}, \frac{\partial}{\partial y} ,\frac{\partial}{\partial z} \right\rangle \cdot  \langle xz^2,\frac{y^3}{3},zy^2+xy\rangle  = z^2+y^2+y^2$$

Remembering that the orientation of $S$ was inwards, we know that due to the divergence theorem the integral becomes:
$$\iint_S \vec{F}\cdot (-1)\hat{n}dS = -\iiint_D \nabla \cdot \vec{F}dV \to -\iiint_D (z^2+2y^2)dV$$

This is easier to do in spherical coordinates where $\rho$ varies from $0$ to $1$, and $\phi$ and $\theta$ both vary from $0$ to $\frac{\pi}{2}$. Continuing the computation:
$$-\iiint_D (z^2+2y^2)dV$$
$$ \to -\int_0^{\frac{\pi}{2}}\int_0^{\frac{\pi}{2}}\int_0^1 (\rho^2\cos^2(\phi)+2\rho^2\sin^2(\phi)\sin^2(\theta))\rho^2\sin(\phi)d\rho d\phi d\theta$$
$$\to -\frac{1}{5}\int_0^{\frac{\pi}{2}}\int_0^{\frac{\pi}{2}} \left(\cos^2(\phi)\sin(\phi)+2\sin^3(\phi)\sin^2(\theta)\right)d\phi d\theta$$
$$\to -\frac{1}{5}\int_0^{\frac{\pi}{2}} \left(\frac{1}{3}+\frac{4}{3}\sin^2(\theta)\right)d\theta = -\frac{\pi}{10}$$

\subsection{46.6.24}

We will use the divergence theorem. Give $D$ as the solid region bounded by $S$. First we must find the divergence of the vector field:
$$\nabla\cdot\vec{F} = \left\langle \frac{\partial}{\partial x}, \frac{\partial}{\partial y} ,\frac{\partial}{\partial z} \right\rangle \cdot \langle yz, z^2x+y, z-xy \rangle= 2$$

Then we can apply the divergence theorem to find the flux:
$$\iint_S \vec{F}\cdot \hat{n}dS = \iiint_D (2)dV$$

The integral is fairly simple to compute, especially using spherical coordinates. Since we have a unit sphere and the coefficient of $\sqrt{x^2+y^2}$ is 1, the bounds are nice. The parameter $\rho$ varies from $0$ to $1$, $\theta$ takes on its natural range, and $\phi$ ranges from $0$ to $\frac{\pi}{4}$. Then the integral becomes:
$$2\iiint_D dV \to 2\int_0^{2\pi}\int_0^{\frac{\pi}{4}}\int_0^1 \rho^2\sin(\phi)d\rho d\phi d\theta = \frac{4\pi}{3}\left(1-\frac{1}{\sqrt{2}}\right)$$

\subsection{46.6.32}

We would like to use the divergence theorem, which means we want to form a closed surface somehow. Notice that if we take the union of $S$ with a disk of radius $2$ centered at the origin (call this disk $S_d$, and orient it upwards to mimic the inward orientation of $S$), we have a surface that is closed. In particular this surface would be the boundary of the upper hemisphere of a ball of radius 2.

Keep in mind that if we want to use the divergence theorem, we must introduce negative signs into the surface integral since the statement involves surfaces that are oriented outward, unlike our surface $S\cup S_d$. Give the region bounded by this surface as $D$. 

This deformation by introducing the disk $S_d$ means that the following equality (by additivity) is true:
$$\iint_S \vec{F}\cdot (-1)\hat{n}dS + \iint_{S_d} \vec{F}\cdot (-1)\hat{n}dS = \iiint_D \left(\nabla \cdot \vec{F}\right) dV$$
$$\to \iint_S \vec{F}\cdot \hat{n}dS = -\iiint_D \left(\nabla \cdot \vec{F}\right) dV - \iint_{S_d} \vec{F}\cdot \hat{n}dS$$

Compute the divergence of the vector field $\vec{F}$:
$$\nabla\cdot\vec{F} = \left\langle \frac{\partial}{\partial x}, \frac{\partial}{\partial y} ,\frac{\partial}{\partial z} \right\rangle \cdot \langle xy^2, yz^2,zx^2+x^2 \rangle= y^2+z^2+x^2$$

We shall first compute the triple integral using the divergence we just computed. It will be useful to use spherical coordinates once more. Keep in mind that $\rho$ ranges from $0$ to $2$ and $\phi$ ranges from $0$ to $\frac{\pi}{2}$, where $\theta$ takes on its natural range.
$$-\iiint_D \left(\nabla \cdot \vec{F}\right) dV \to -\iiint (y^2+z^2+x^2) dV$$
$$ \to -\int_0^{2\pi}\int_0^{\frac{\pi}{2}}\int_0^2 (\rho^2)(\rho^2\sin(\phi))d\rho d\phi d\theta = -\frac{64\pi}{5}$$

Then we must compute the surface integral over the disk $S_d$. To aid in computation, note that it is easiest to give the unit normal vector as $\hat{e}_3$, so that the integrand reduces to a simpler form:
$$-\iint_{S_d} \vec{F}\cdot\hat{e}_3 dS \to -\iint_{S_d}\langle xy^2, yz^2,zx^2+x^2 \rangle \cdot \langle0,0,1\rangle dS $$
$$\to -\iint_{S_d} x^2(z+1)dS$$

On all points on the disk $S_d$, the value of $z$ is $0$. The integral becomes much simpler and it is useful to use polar coordinates (remember the radius of the boundary of the disk is $2$) to compute the integral:
$$-\iint_{S_d}x^2dS\to -\int_0^{2\pi}\int_0^2 (r^2\cos^2{\theta})(r)drd\theta = -4\pi$$

Add the two results we found from both of the integrals to find the desired flux over the original surface $S$, which is $-4\pi - \frac{64\pi}{5} = -\frac{84}{5}\pi$.



\end{document}