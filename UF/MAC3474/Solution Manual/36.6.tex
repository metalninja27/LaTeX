\documentclass{article}
\usepackage[utf8]{inputenc}
\usepackage{amsmath}

\title{Solution Manual}
\author{N. Kapsos, M. Schrank, S. Sivakumar}
\date{}

\begin{document}

\maketitle
\setcounter{secnumdepth}{0}

\section{36.6 Exercises}

\subsection{36.6.1}

This is fairly direct.

$$J = \bigg|\det \begin{pmatrix}
    x^{\prime}_u &x^{\prime}_v &x^{\prime}_w  \\
    y^{\prime}_u &y^{\prime}_v &y^{\prime}_w  \\
    z^{\prime}_u &z^{\prime}_v &z^{\prime}_w 
\end{pmatrix} \bigg| = \bigg|\det \begin{pmatrix}
   \frac{1}{v} &-\frac{u}{v^2} &0 \\
   \\
   0 &\frac{1}{w} &-\frac{v}{w^2} \\
   \\
   -\frac{w}{u^2}&0 &\frac{1}{u}
\end{pmatrix} \bigg| = 0$$

\subsection{36.6.6}

First change the variables to find that $u+v+w\leq a$, which is the region under a plane that intersects the $u$, $v$, and $w$ axes at $a$ (meaning the points are $(0,0,a)$, $(0,a,0)$, $(a,0,0)$). It is easiest to give the bounds in a vertically simple manner, meaning to start with $0\leq w\leq a-u-v$. Then to find the bounds in $v$ we may give the line where the plane intersects with the $uv$ plane as $v=5-u$, so evidently $0\leq v\leq a-u$. And $u$ will then vary from $0$ to $a$.

Then we compute the Jacobian as follows for $dxdy = Jdudv$. We must rewrite the equations given into $x=u^2$, $y=v^2$, and $z=w^2$.
Then:
$$J = \bigg|\det \begin{pmatrix}
    2u &0 &0 \\
    0 &2v &0 \\
    0 &0 &2w 
 \end{pmatrix} \bigg| = 8uvw$$

 Then the triple integral becomes
 $$\int_{0}^{a}\int_{0}^{a-u}\int_{0}^{a-u-v} (8uvw) dwdvdu$$
 $$ \to \int_{0}^{a} 4u \int_{0}^{a-u}(a^2 v - 2 a u v + u^2 v - 2 a v^2 + 2 u v^2 + v^3)dvdu$$
 $$\frac{1}{3}\int_0^a u(a - u)^4 du = \frac{1}{90}a^6$$

 It may be helpful to use auxiliary substitutions for the later integrals.

\subsection{36.6.13}

From the bounds give $u=\frac{x}{3}$, $v=\frac{y}{2}$, and $w=z$. The Jacobian is given by $\frac{1}{J}$ where $J$ is given by $$J = \bigg| \det \begin{pmatrix}
    \frac{1}{3} & 0 & 0 \\
    \\
    0 & \frac{1}{2} & 0 \\
    \\
    0 &0 & 1
\end{pmatrix}\bigg| = \frac{1}{6}$$

The bound given by the paraboloid remains a paraboloid, but it becomes a circular paraboloid $w = u^2+v^2$ that intersects with the same plane $z=w=10$. Call this region $E^{\prime}$.

The triple integral after substitution becomes $$216\iiint_{E^{\prime}} (u^2-v^2)dA^{\prime}$$

However, notice that there is symmetry of both the integrand (skew symmetry) and the region of integration (geometric symmetry) across the line $u=v$, so we may apply the transformation $(u,v,w)\to (v,u,w)$ to find that the sign of the integral will flip and so the integral vanishes.

\end{document}