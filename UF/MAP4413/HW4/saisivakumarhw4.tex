\documentclass[11pt]{article}
\headheight = 14pt
% packages
\usepackage{physics}
% margin spacing
\usepackage[top=1in, bottom=1in, left=0.5in, right=0.5in]{geometry}
\usepackage{hanging}
\usepackage{amsfonts, amsmath, amssymb, amsthm}
\usepackage{systeme}
\usepackage[none]{hyphenat}
\usepackage{fancyhdr}
\usepackage[nottoc, notlot, notlof]{tocbibind}
\usepackage{graphicx}
\graphicspath{{./images/}}
\usepackage{float}
\usepackage{siunitx}
\usepackage{esint}
\usepackage{cancel}
\usepackage{enumitem}

% colors
\usepackage{xcolor}
\definecolor{p}{HTML}{FFDDDD}
\definecolor{g}{HTML}{D9FFDF}
\definecolor{y}{HTML}{FFFFCF}
\definecolor{b}{HTML}{D9FFFF}
\definecolor{o}{HTML}{FADECB}
%\definecolor{}{HTML}{}

% \highlight[<color>]{<stuff>}
\newcommand{\highlight}[2][p]{\mathchoice%
  {\colorbox{#1}{$\displaystyle#2$}}%
  {\colorbox{#1}{$\textstyle#2$}}%
  {\colorbox{#1}{$\scriptstyle#2$}}%
  {\colorbox{#1}{$\scriptscriptstyle#2$}}}%

% header/footer formatting
\pagestyle{fancy}
\fancyhead{}
\fancyfoot{}
\fancyhead[L]{MAP4413 Dr. Zhang}
\fancyhead[C]{HW4}
\fancyhead[R]{Sai Sivakumar}
\fancyfoot[R]{\thepage}
\renewcommand{\headrulewidth}{1pt}

% paragraph indentation/spacing
\setlength{\parindent}{0cm}
\setlength{\parskip}{5pt}
\renewcommand{\baselinestretch}{1.25}

% extra commands defined here
\newcommand{\ihat}{\boldsymbol{\hat{\textbf{\i}}}}
\newcommand{\jhat}{\boldsymbol{\hat{\textbf{\j}}}}
\newcommand{\dr}{\vec{r}~^{\prime}(t)}
\newcommand{\dx}{x^{\prime}(t)}
\newcommand{\dy}{y^{\prime}(t)}

\newcommand{\br}[1]{\left(#1\right)}
\newcommand{\sbr}[1]{\left[#1\right]}
\newcommand{\cbr}[1]{\left\{#1\right\}}

\newcommand{\dprime}{\prime\prime}
\newcommand{\lap}[2]{\mathcal{L}[#1](#2)}

% bracket notation for inner product
\usepackage{mathtools}

\DeclarePairedDelimiterX{\abr}[1]{\langle}{\rangle}{#1}

\DeclareMathOperator{\Span}{span}
\DeclareMathOperator{\nullity}{nullity}
\DeclareMathOperator\Arg{Arg}
\DeclareMathOperator\Log{Log}


% set page count index to begin from 1
\setcounter{page}{1}

\begin{document}
p.88: 2, 8, 9, 10, 11, 14

2. Prove that the vector space $\ell^2(\mathbb{Z})$ over $\mathbb{C}$ is complete.
\begin{proof}
    Suppose $A_k = \cbr{a_{k,n}}_{n\in\mathbb{Z}}$ with $k = 1,2,\dots$ is a Cauchy sequence.

    For each $n\in\mathbb{Z}$, observe that the collection of $n$-th terms in each of the sequences $A_k$ for $k = 1,2,\dots$ form a Cauchy sequence $\cbr{a_{k,n}}_{k=1}^{\infty}$. This is because for each $n\in \mathbb{Z}$ and $\varepsilon > 0$, there exists $K\in \mathbb{Z}^{+}$ such that for $k,k^{\prime} > K$, 
    \[\abs{a_{k,n} - a_{k^{\prime},n}}^2 \leq \norm{A_k - A_{k^{\prime}}}^2 = \sum_{n\in\mathbb{Z}} \abs{a_{k,n} - a_{k^{\prime},n}}^2 < \varepsilon^2,\]
    implying that $\abs{a_{k,n} - a_{k^{\prime},n}} < \varepsilon$. Then because $\cbr{a_{k,n}}_{k=1}^{\infty}$ is Cauchy, it converges to some $b_n\in \mathbb{C}$. In particular, this means that the sequence $A_k$ converges to some sequence $B = \cbr{b_n}_{n\in\mathbb{Z}}$. To see this, observe that for any given $\varepsilon > 0$, we may choose a positive integer $K$ large enough so that for all $k,k^{\prime} > K$, the partial sums $\sum_{n=-N}^{N}\abs{a_{k,n} - a_{k^{\prime},n}}^2$ of $\norm{A_k - A_{k^{\prime}}}^2$ satisfy
    \[\sum_{n=-N}^{N}\abs{a_{k,n} - a_{k^{\prime},n}}^2 \leq \norm{A_k - A_{k^{\prime}}}^2 < \varepsilon^2,\] for any positive integer $N$. Then let $k^{\prime}$ tend to positive infinity, so that
    \[\sum_{n=-N}^{N}\abs{a_{k,n} - b_n}^2 < \varepsilon^2.\] Because $N$ was arbitrary, it follows that $\norm{A_k - B} < \varepsilon$ for $k > K$, so $A_k$ does converge to $B$.
    
    The vector space $\ell^2(\mathbb{Z})$ is complete if we show that $B\in\ell^2(\mathbb{Z})$; that is, $\norm{B}^2$ is finite. By the triangle inequality, $\norm{B} = \norm{B-A_k+A_k} \leq \norm{B-A_k} + \norm{A_k}$. But $\norm{A_k}$ and $\norm{B-A_k}$ are finite, so $B\in \ell^{2}(\mathbb{Z})$. 
    
    Hence $\ell^{2}(\mathbb{Z})$ over $\mathbb{C}$ is complete.
\end{proof}

8. 
\begin{enumerate}[label=(\alph*)]
    \item The Fourier coefficients of $f(\theta) = \abs{\theta}$ are $a_0 = \pi/2$ and $a_n = ((-1)^n+1)/(\pi n^2)$ for $n\neq 0$. Then because $a_n$ vanishes for nonzero even $n$, we have by Parseval's identity that
    \begin{align*}
      \frac{1}{2\pi}\int_{-\pi}^{\pi}\abs{\theta}^2\dd{\theta} = \frac{\pi^2}{3} &= \br{\frac{\pi}{2}}^2 + \sum_{\substack{n\in \mathbb{Z} \\n\neq 0}} \br{\frac{(-1)^n-1}{\pi n^2}}^2\\
      &= \frac{\pi^2}{4} + \frac{8}{\pi^2}\sum_{n \text{ odd }\geq 1 } \frac{1}{n^4},
    \end{align*} so that 
    \[\sum_{n \text{ odd }\geq 1} \frac{1}{n^4} = \sum_{n=0}^{\infty}\frac{1}{(2n+1)^4} = \frac{\pi^4}{96}.\] Then observe that
    \begin{align*}
      \sum_{n=1}^{\infty}\frac{1}{n^4} &= \sum_{n\text{ even }\geq 2 } \frac{1}{n^4} + \sum_{n \text{ odd }\geq 1} \frac{1}{n^4}\\
      &= \frac{1}{16}\sum_{n=1}^{\infty}\frac{1}{n^4} + \frac{\pi^4}{96},
    \end{align*} which implies that \[\sum_{n=1}^{\infty}\frac{1}{n^4} = \frac{\pi^4}{90}.\]
    \item The Fourier coefficients of the odd function defined on $[0,\pi]$ by $f(\theta) = \theta(\pi-\theta)$ are given by $-4i/(\pi n^3)$ for odd $n\in \mathbb{Z}$. Then 
    \begin{align*}
      \frac{1}{2\pi}\int_{-\pi}^{\pi} \abs{f(\theta)}^2\dd{\theta} &= \frac{1}{2\pi}\int_{-\pi}^{0}\theta^2(\pi+\theta)^2 \dd{\theta} + \frac{1}{2\pi}\int_{0}^{\pi}\theta^2(\pi-\theta)^2 \dd{\theta}\\
      &= \frac{1}{\pi}\int_{0}^{\pi}\theta^2(\pi-\theta)^2 \dd{\theta} \\
      &=\frac{\pi^4}{30},
    \end{align*} so that 
    \begin{align*}
      \frac{\pi^4}{30} = \sum_{n\text{ odd }\geq 1}\abs{\frac{-4i}{\pi n^3}}^2 &= 2\sum_{n\text{ odd }\geq 1}\br{\frac{4}{\pi n^3}}^2\\
      &= \frac{32}{\pi^2}\sum_{n=0}^{\infty}\frac{1}{(2n+1)^6}.
    \end{align*} Hence \[\sum_{n=0}^{\infty}\frac{1}{(2n+1)^6} = \frac{\pi^6}{960}.\]
    Then observe \begin{align*}
      \sum_{n=1}^{\infty}\frac{1}{n^6} &= \sum_{n\text{ even }\geq 2}\frac{1}{n^6} + \sum_{n\text{ odd }\geq 1}\frac{1}{n^6} \\
      &= \frac{1}{64}\sum_{n=1}^{\infty}\frac{1}{n^6} + \frac{\pi^6}{960},
    \end{align*} which implies that \[ \sum_{n=1}^{\infty}\frac{1}{n^6} = \frac{\pi^6}{945}.\]
\end{enumerate}

9. The Fourier coefficients of $\frac{\pi}{\sin(\pi\alpha)}e^{i(\pi-x)\alpha}$ are  
\begin{align*}
  a_n = \frac{1}{2\pi}\int_0^{2\pi}\br{\frac{\pi}{\sin(\pi\alpha)}e^{i(\pi-x)\alpha}}e^{-inx}\dd{x} &= \frac{e^{i\pi \alpha}}{2\sin(\pi\alpha)}\int_0^{2\pi} e^{-i(n+\alpha)x}\dd{x}\\
  &= \frac{e^{i\pi \alpha}}{2\sin(\pi\alpha)}\br{\eval{\frac{e^{-i\alpha x}}{-i(n+\alpha)}}_{0}^{2\pi}}\\
  &= \frac{e^{-i\pi\alpha} - e^{i\pi\alpha}}{-2i\sin(\pi\alpha)(n+\alpha)}\\
  &= \frac{1}{n+\alpha}.
\end{align*}
Hence the Fourier series of $\frac{\pi}{\sin(\pi\alpha)}e^{i(\pi-x)\alpha}$ is given by \[\frac{\pi}{\sin(\pi\alpha)}e^{i(\pi-x)\alpha} \sim \sum_{n=-\infty}^{\infty}\frac{e^{inx}}{n+\alpha}.\]
Hence by applying Parseval's formula we have \begin{align*}
  \frac{1}{2\pi}\int_0^{2\pi}\abs{\frac{\pi e^{i(\pi-x)\alpha}}{\sin(\pi\alpha)}}^2\dd{x}  = \frac{1}{2\pi}\int_0^{2\pi}\br{\frac{\pi}{\sin(\pi\alpha)}}^2\dd{x} = \frac{\pi^2}{\sin^2(\pi\alpha)} &= \sum_{n=-\infty}^{\infty}\abs{\frac{e^{inx}}{n+\alpha}}^2\\
  &= \sum_{n=-\infty}^{\infty} \frac{1}{(n+\alpha)^2}.
\end{align*}

10. Show that the total energy $E(t) = \frac{1}{2}\rho \int_0^L \br{\pdv{u}{t}}^2\dd{x} + \frac{1}{2}\tau \int_0^L \br{\pdv{u}{x}}^2\dd{x}$ of a vibrating string whose displacement $u(x,t)$ satisfies the wave equation $\rho u^{\dprime}_{tt} = \tau u^{\dprime}_{xx}$ with initial conditions $u(x,0) = f(x)$ and $u^{\prime}_t(x,0) = g(x)$ is constant.
\begin{proof}
  The total energy is constant if $E^{\prime}(t)$ vanishes. With $u(x,t)$ smooth enough and satisfying the wave equation, we have
  \begin{align*}
    \dv{t}E(t) &= \dv{t}\br{\frac{1}{2}\rho \int_0^L \br{\pdv{u}{t}}^2\dd{x} + \frac{1}{2}\tau \int_0^L \br{\pdv{u}{x}}^2\dd{x}}\\
    &= \frac{1}{2}\rho \int_0^L \dv{t}\br{\pdv{u}{t}}^2\dd{x} + \frac{1}{2}\tau \int_0^L \dv{t}\br{\pdv{u}{x}}^2\dd{x}\\
    &= \rho \int_0^L \pdv[2]{u}{t}\pdv{u}{t}\dd{x} + \tau \int_0^L \pdv[2]{u}{x}{t}\pdv{u}{x}\dd{x}\\
    &= \rho \int_0^L \pdv[2]{u}{t}\pdv{u}{t}\dd{x} + \tau \int_0^L \pdv[2]{u}{t}{x}\pdv{u}{x}\dd{x}\\
    &= \rho \int_0^L \pdv[2]{u}{t}\pdv{u}{t}\dd{x} + \tau\br{\eval{\pdv{u}{x}\pdv{u}{t}}_0^L} - \tau \int_0^L \pdv[2]{u}{x}\pdv{u}{t}\dd{x}\\
    &= \rho \int_0^L \pdv[2]{u}{t}\pdv{u}{t}\dd{x} + \tau\br{ \pdv{u}{x}\br{L,t} \pdv{u}{t}\br{L,t} - \pdv{u}{x}\br{0,t} \pdv{u}{t}\br{0,t} } - \tau \int_0^L \pdv[2]{u}{x}\pdv{u}{t}\dd{x}\\
    &= \rho \int_0^L \pdv[2]{u}{t}\pdv{u}{t}\dd{x} - \tau \int_0^L \pdv[2]{u}{x}\pdv{u}{t}\dd{x} = \int_0^L \pdv{u}{t}\br{\rho \pdv[2]{u}{t} - \tau \pdv[2]{u}{x}}\dd{x} = \int_0^L 0\dd{x} =0,
  \end{align*} where we used integration by parts and the periodicity of $u^{\prime}(x,t)$ in $x$ to arrive at $E^{\prime}(t) = 0$, which means that $E(t)$ is constant for all time, and in particular, is equal to $E(0)$.
\end{proof}

11.
\begin{enumerate}[label=(\alph*)]
    \item \begin{proof}
      Let $f$ be $T$-periodic, continuous, and piecewise $C^1$ with $\int_0^T f(t)\dd{t} = 0$ as given. Then observe that the condition $\int_0^T f(t)\dd{t} = 0$ yields $\hat{f}(0) = 0$. Since $f$ is $C^1$, we can use integration by parts to find that for $n\neq 0$, \begin{align*}
        a_n = \hat{f}(n) &= \frac{1}{T}\int_0^T f(t)e^{int(2\pi/T)}\dd{t}\\
        &= \eval{f(t)\frac{T e^{int(2\pi/T)} }{-2\pi in} }_0^T + \br{\frac{T}{2\pi in}}\frac{1}{T}\int_0^T f^{\prime}(t)e^{int(2\pi/T)}\dd{t}\\
        &= \frac{T}{2\pi in}\widehat{f^{\prime}}(n) = \frac{T}{2\pi in} b_n.
      \end{align*}
      Then we can apply Parseval's identity with $\hat{f}(0) = 0$ to $\int_0^T \abs{f(t)}^2\dd{t}$ to find
      \begin{align*}
        \int_0^T \abs{f(t)}^2\dd{t} &= T\sum_{n\neq 0} \abs{a_n}^2\\
        &= T\sum_{n\neq 0} \abs{\frac{T}{2\pi in} b_n}^2\\
        &= \frac{T^3}{4\pi^2}\sum_{n\neq 0}\frac{\abs{b_n}^2}{n^2} \\
        &\leq \frac{T^3}{4\pi^2}\sum_{n=-\infty}^{\infty}\abs{b_n}^2\\
        &= \frac{T^2}{4\pi^2}\int_0^T \abs{f^{\prime}(t)}^2\dd{t}.
      \end{align*}
      We have equality if for $\abs{n}\geq 2$, $a_n = 0$ (we also need $b_0 = 0$). This is because we must have $n^2 = 1$ for $n = \pm 1$, and this requirement also forces $b_0 = 0$ since $f(t) = a_1e^{int(2\pi/T)} + a_{-1}e^{-int(2\pi/T)} = A\sin(2\pi t/T) + B\cos(2\pi t/T)$, whose derivative is clearly periodic over $T$ as well.
    \end{proof}
    \item \begin{proof}
      Let $g$ be $C^1$ and $T$-periodic as given, with $a_n = \hat{f}(n)$, $b_n = \hat{g}(n)$, and $c_n = \widehat{g^{\prime}}(n)$. Note that for $n\neq 0$, $b_n = \frac{T}{2\pi in} c_n$. Then with $a_0 = 0$ and using the Cauchy-Schwarz inequality, \begin{align*}
        \abs{\int_0^T \overline{f(t)}g(t)\dd{t}}^2 &= T^2\abs{\sum_{n=-\infty}^{\infty} \overline{a_n}b_n}^2 =T^2\abs{\sum_{n\neq 0} \overline{a_n}b_n}^2\\
        &\leq \br{T\sum_{n\neq 0}\abs{a_n}^2}\br{T\sum_{n\neq 0}\abs{b_n}^2}\\
        &\leq \br{T\sum_{n=-\infty}^{\infty}\abs{a_n}^2}\br{\frac{T^3}{4\pi^2}\sum_{n\neq 0}\abs{c_n}^2 + Tc_0}\\
        &\leq \int_0^T \abs{f(t)}^2\dd{t} \br{\frac{T^3}{4\pi^2}\sum_{n = -\infty}^{\infty}\abs{c_n}^2}\\
        &= \frac{T^2}{4\pi^2}\int_0^T \abs{f(t)}^2\dd{t}\int_0^T \abs{g^{\prime}(t)}^2\dd{t},
      \end{align*} so $\abs{\int_0^T \overline{f(t)}g(t)\dd{t}}^2\leq \frac{T^2}{4\pi^2}\int_0^T \abs{f(t)}^2\dd{t}\int_0^T \abs{g^{\prime}(t)}^2\dd{t}$ as desired.
    \end{proof}
    \item \begin{proof}
      Let $[a,b]$ be any compact interval, and let $f$ be any continuously differentiable function such that $f(a) = f(b) = 0$ as given. Then extend $f$ to be an odd function centered at $a$ (that is, $f(t+a)$ is an odd function about the origin) and periodic with period $T = 2(b-a)$, which means any integral over an interval of length $T$, like $\int_{2a-b}^{b}f(t)\dd{t} $, vanishes. Then this extended $f$ satisfies the hypotheses of part (a) up to a change of variables (translation), so
      \begin{align*}
        \int_a^b \abs{f(t)}^2\dd{t} = \frac{1}{2}\int_{2a-b}^b \abs{f(t)}^2\dd{t} \leq \frac{(b-a)^2}{2\pi^2}\int_{2a-b}^b \abs{f^{\prime}(t)}^2\dd{t} = \frac{(b-a)^2}{\pi^2}\int_a^b \abs{f^{\prime}(t)}^2\dd{t},
      \end{align*} where by construction, $\abs{f^{\prime}(t)}^2$ is symmetric (even) about $a$ in an interval of length $T$. In the case of equality, from part (a) we have $f(t) = A\sin(2\pi (t-a)/T) + B\cos(2\pi (t-a)/T)$ which is centered at $a$, but it should not have the cosine term since $f$ is odd about $a$. Then $f(t) = A\sin(2\pi (t-a)/T) = A\sin(\pi \frac{t-a}{b-a})$.
    \end{proof}
\end{enumerate}

14. Prove that the Fourier series of a continuously differentiable function $f$ on the circle is absolutely convergent.
\begin{proof}
  With $a_n = \hat{f}(n)$ and $b_n = \widehat{f^{\prime}}(n)$, the Fourier series for $f$ is given by $\sum_{n=-\infty}^{\infty} a_n e^{int}$. Note that $a_n = (in)^{-1}b_n $ for $n\neq 0$, and observe that the Fourier series converges if the series $\sum_{n\neq 0} a_n e^{int}$ converges (removing a finite number of terms does not affect convergence). Then
  \begin{align*}
    \abs{\sum_{n\neq 0} a_n e^{int}} \leq \sum_{n\neq 0} \abs{a_n} &= \sum_{n\neq 0} \abs{(in)^{-1}b_n}\\
    &= \sum_{n\neq 0} \abs{n}^{-1}\abs{b_n}\\
    &\leq \br{2\sum_{n\geq 1}\abs{n}^{-2}}^{\frac{1}{2}}\br{\sum_{n\neq 0} \abs{b_n}^2}^{\frac{1}{2}}\\
    &\leq \sqrt{\frac{\pi^2}{3}}\cdot \frac{1}{2\pi}\int_{-\pi}^{\pi}\abs{f^{\prime}(t)}^2\dd{t} < \infty.
  \end{align*}
  Hence the Fourier series of $f$ converges absolutely.
\end{proof}
\end{document}