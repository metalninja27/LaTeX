\documentclass[11pt]{article}

% packages
\usepackage{physics}
% margin spacing
\usepackage[top=1in, bottom=1in, left=0.5in, right=0.5in]{geometry}
\usepackage{hanging}
\usepackage{amsfonts, amsmath, amssymb, amsthm}
\usepackage{systeme}
\usepackage[none]{hyphenat}
\usepackage{fancyhdr}
\usepackage[nottoc, notlot, notlof]{tocbibind}
\usepackage{graphicx}
\graphicspath{{./images/}}
\usepackage{float}
\usepackage{siunitx}
\usepackage{esint}
\usepackage{cancel}

% colors
\usepackage{xcolor}
\definecolor{p}{HTML}{FFDDDD}
\definecolor{g}{HTML}{D9FFDF}
\definecolor{y}{HTML}{FFFFCF}
\definecolor{b}{HTML}{D9FFFF}
\definecolor{o}{HTML}{FADECB}
%\definecolor{}{HTML}{}

% \highlight[<color>]{<stuff>}
\newcommand{\highlight}[2][p]{\mathchoice%
  {\colorbox{#1}{$\displaystyle#2$}}%
  {\colorbox{#1}{$\textstyle#2$}}%
  {\colorbox{#1}{$\scriptstyle#2$}}%
  {\colorbox{#1}{$\scriptscriptstyle#2$}}}%

% header/footer formatting
\pagestyle{fancy}
\fancyhead{}
\fancyfoot{}
\fancyhead[L]{MAS4105 Dr. Zhang}
\fancyhead[C]{HW15}
\fancyhead[R]{Sai Sivakumar}
\fancyfoot[R]{\thepage}
% remove underlined header
%\renewcommand{\headrulewidth}{0pt}

% paragraph indentation/spacing
\setlength{\parindent}{0cm}
\setlength{\parskip}{5pt}
\renewcommand{\baselinestretch}{1.25}

% extra commands defined here
\newcommand{\ihat}{\boldsymbol{\hat{\textbf{\i}}}}
\newcommand{\jhat}{\boldsymbol{\hat{\textbf{\j}}}}
\newcommand{\dr}{\vec{r}~^{\prime}(t)}
\newcommand{\dx}{x^{\prime}(t)}
\newcommand{\dy}{y^{\prime}(t)}

\newcommand{\br}[1]{\left(#1\right)}
\newcommand{\sbr}[1]{\left[#1\right]}
\newcommand{\cbr}[1]{\left\{#1\right\}}
\newcommand{\abr}[1]{\left\langle#1\right\rangle}

\newcommand{\dprime}{\prime\prime}
\newcommand{\lap}[2]{\mathcal{L}[#1](#2)}

% mathtools
\usepackage{mathtools}

%\DeclarePairedDelimiterX{\abr}[1]{\langle}{\rangle}{#1}

\DeclarePairedDelimiter\ceil{\lceil}{\rceil}
\DeclarePairedDelimiter\floor{\lfloor}{\rfloor}

\DeclareMathOperator{\Span}{span}
\DeclareMathOperator{\nullity}{nullity}

% set page count index to begin from 1
\setcounter{page}{1}

% theorem/corollary/lemma
\newtheorem{theorem}{Theorem}[section]
\newtheorem{corollary}{Corollary}[theorem]
\newtheorem{lemma}[theorem]{Lemma}

\newcommand{\vardbtilde}[1]{\tilde{\raisebox{0pt}[0.85\height]{$\tilde{#1}$}}}

\newtheorem*{theorem*}{Theorem}

\begin{document}
6.2: 2(a,c,i),  5, 9, 15(a) \\

2. In each part:

\hspace{1cm}1. Obtain an orthogonal basis for $\Span(S)$. (Gram-Schmidt)

\hspace{1cm}2. Obtain an orthonormal basis for $\Span(S)$. (Normalize all of the vectors obtained in 1.)

\hspace{1cm}3. Compute the Fourier coefficients of the given vector relative to $\beta$, and use Theorem 6.5 to verify.

(a) Observe that $S$ is a basis for $\mathbb{R}^3$, and that the inner product in this space is just the dot product.

%\frac{()\cdot()}{\norm{()}^2}()

An orthogonal basis is $\boxed{\cbr{(1,0,1), \br{-\frac{1}{2}, 1, \frac{1}{2}} , \br{-1,0,-2}}}$. \begin{align*}
  v_1 &= (1,0,1) \\
  v_2 &= (0,1,1) - \frac{(0,1,1)\cdot(1,0,1)}{\norm{(1,0,1)}^2}(1,0,1) = \br{-\frac{1}{2}, 1, \frac{1}{2}} \\
  v_3 &= (1,3,3) - \frac{(1,3,3)\cdot(1,0,1)}{\norm{(1,0,1)}^2}(1,0,1) - \frac{(1,3,3)\cdot\br{-\frac{1}{2}, 1, \frac{1}{2}}}{\norm{\br{-\frac{1}{2}, 1, \frac{1}{2}}}^2}\br{-\frac{1}{2}, 1, \frac{1}{2}} = \br{\frac{1}{3}, \frac{1}{3}, -\frac{1}{3}}
\end{align*}

Normalize to find the orthonormal basis $\boxed{\cbr{\br{\frac{1\sqrt{2}}{2}, 0, \frac{1\sqrt{2}}{2}},\br{\frac{-\sqrt{6}}{6}, \frac{\sqrt{6}}{3}, \frac{\sqrt{6}}{6}},\br{\frac{\sqrt{3}}{3}, \frac{\sqrt{3}}{3}, -\frac{\sqrt{3}}{3}}}}$.

To compute the Fourier coefficients for $(1,1,2)$, use the definition. They are $(1,1,2)\cdot\br{\frac{1\sqrt{2}}{2}, 0, \frac{1\sqrt{2}}{2}} = \frac{3\sqrt{2}}{2}$, $(1,1,2)\cdot \br{\frac{-\sqrt{6}}{6}, \frac{\sqrt{6}}{3}, \frac{\sqrt{6}}{6}}= \frac{\sqrt{6}}{2}$, $(1,1,2)\cdot\br{\frac{\sqrt{3}}{3}, \frac{\sqrt{3}}{3}, -\frac{\sqrt{3}}{3}} = 0$. Then we can confirm that $(1,1,2) = \frac{3\sqrt{2}}{2}\br{\frac{1\sqrt{2}}{2}, 0, \frac{1\sqrt{2}}{2}} + \frac{\sqrt{6}}{2}\br{\frac{-\sqrt{6}}{6}, \frac{\sqrt{6}}{3}, \frac{\sqrt{6}}{6}} + 0$.

(c) The orthogonal basis is $\boxed{\cbr{1, x-\frac{1}{2}, x^2-x+\frac{1}{6}}}$. \begin{align*}
  v_1 &= 1 \\
  v_2 &= x - \frac{\int_0^1 \br{1t}\dd{t}}{\int_0^1 \br{1}^2\dd{t}}(1) = x-\frac{1}{2} \\
  v_3 &= x^2 - \frac{\int_0^1 \br{1t^2}\dd{t}}{\int_0^1\br{1}^2\dd{t}}(1) - \frac{\int_0^1 \br{t^2\br{t-\frac{1}{2}}}\dd{t}}{\int_0^1\br{t-\frac{1}{2}}^2\dd{t}}\br{x-\frac{1}{2}} = x^2-x+\frac{1}{6}
\end{align*}

Normalize each vector to find the orthonormal basis $\boxed{\cbr{1, \sqrt{12}\br{x-\frac{1}{2}}, \sqrt{180}\br{x^2-x+\frac{1}{6}}}}$.

Then the Fourier coefficients for $h(x)$ are $$\int_0^1\br{1+t}\br{1}\dd{t} = \frac{3}{2}$$
$$\int_0^1\br{1+t}\br{\sqrt{12}\br{t-\frac{1}{2}}}\dd{t} = \frac{\sqrt{12}}{12}$$
$$\int_0^1\br{1+t}\br{\sqrt{180}\br{t^2-t+\frac{1}{6}}}\dd{t} = 0$$ and see that $\frac{3}{2} + \frac{\sqrt{12}}{12}\br{\sqrt{12}\br{x-\frac{1}{2}}} = 1+x = h(x)$.

(i) An orthogonal basis is $\boxed{\cbr{1, t-\frac{\pi}{2}, \sin(t) - \frac{2}{\pi}, \cos(t) + \frac{24}{\pi^3}\br{t-\frac{\pi}{2}}}}$. \begin{align*}
  v_1 &= 1 \\
  v_2 &= t - \frac{\int_0^{\pi} 1t \dd{t}}{\int_0^{\pi} 1^2 \dd{t}}(1) = t-\frac{\pi}{2} \\
  v_3 &= \sin(t) - \frac{\int_0^\pi 1\sin(t)\dd{t}}{\int_0^\pi 1^2 \dd{t}}(1) - \frac{\int_0^\pi \sin(t)\br{t-\frac{\pi}{2}}\dd{t}}{\int_0^\pi \br{t-\frac{\pi}{2}}^2\dd{t}}\br{t-\frac{\pi}{2}} = \sin(t) - \frac{2}{\pi} \\
  v_4 &= \cos(t) - \frac{\int_0^\pi 1\cos(t)\dd{t}}{\int_0^\pi 1^2\dd{t}}(1) - \frac{\int_0^\pi \cos(t)\br{t-\frac{\pi}{2}}\dd{t}}{\int_0^\pi \br{t-\frac{\pi}{2}}^2\dd{t}}\br{t-\frac{\pi}{2}} - \frac{\int_0^\pi \cos(t)\br{\sin(t)-2}\dd{t}}{\int_0^\pi \br{\sin(t)-2}^2\dd{t}}\br{\sin(t)-2}\\
  &\hspace{5cm}= \cos(t) + \frac{24}{\pi^3}\br{t-\frac{\pi}{2}}
\end{align*}

Normalizing, we will find that the orthonormal basis is \\$\boxed{\cbr{ \frac{1}{\pi}, \sqrt{\frac{12}{\pi^3}}\br{t-\frac{\pi}{2}} , \br{\sqrt{\frac{\pi}{2} + \frac{4}{\pi}}}^{-1}\br{\sin(t)-\frac{2}{\pi}} , \br{\sqrt{\frac{\pi}{2} - \frac{48}{\pi^3}}}^{-1}\br{\cos(t) + \frac{24}{\pi^3}\br{t-\frac{\pi}{2}}}}}$.

Then the Fourier coefficients are found similarly: 
$$\int_0^\pi (2t+1)\br{1}\dd{t} = \pi^2+\pi$$ 
$$\int_0^\pi (2t+1)\br{\sqrt{\frac{12}{\pi^3}}\br{t-\frac{\pi}{2}}}\dd{t} = \sqrt{\frac{12}{\pi^3}}\br{\frac{\pi^3}{6}}$$
$$\int_0^\pi (2t+1)\br{\br{\sqrt{\frac{\pi}{2} + \frac{4}{\pi}}}^{-1}\br{\sin(t)-\frac{2}{\pi}} }\dd{t} = 0$$
$$\int_0^\pi (2t+1)\br{\br{\sqrt{\frac{\pi}{2} - \frac{48}{\pi^3}}}^{-1}\br{\cos(t) + \frac{24}{\pi^3}\br{t-\frac{\pi}{2}}}}\dd{t} = 0$$ and see that $h(t) = 2t+1 = (\pi^2+\pi)(\pi^{-1}) + \br{\sqrt{\frac{12}{\pi^3}}\br{\frac{\pi^3}{6}}}\br{\sqrt{\frac{12}{\pi^3}}\br{t-\frac{\pi}{2}}} + 0 + 0$.

5. Let $S_0 = \cbr{x_0}$, where $x_0$ is a nonzero vector in $\mathbb{R}^3$. Describe $S_0^{\perp}$ geometrically. Now suppose that $S = \cbr{x_1,x_2}$ is a linearly independent subset of $\mathbb{R}^3$. Describe $S^\perp$ geometrically.

In the first case we interpret $S_0^{\perp}$ as the plane passing through the initial point of $x_0$ (the origin) which is perpendicular to $x_0$. 

In the second case we interpret $S^\perp$ as the intersection of two planes, the first plane perpendicular to $x_1$ (in a similar manner to the first case), and the second plane perpendicular to $x_2$. The intersection will form a line passing through the origin which is mutually perpendicular to the two vectors (we can also think of this as the span of the singular vector formed by taking the vector cross product $x_1\times x_2$.)

9. Let $\mathsf{W} = \Span(\cbr{(i,0,1)})$ in $\mathbb{C}^3$. Find orthonormal bases for $\mathsf{W}$ and $\mathsf{W}^\perp$. 

We can use the Gram-Schmidt algorithm to generate two more orthogonal vectors in $\mathbb{C}^3$ from the one already given in the definition for $\mathsf{W}$, by orthogonalizing two more linearly independent vectors $(1,0,0), (0,1,0)$. \begin{align*}
  (1,0,0) - \frac{1\overline{i} + 0 + 0 }{i\overline{i} + 0 + 1}(i,0,1) &= \br{\frac{1}{2}, 0, \frac{i}{2}} \\ 
  (0,1,0) - \frac{(0)}{i\overline(i)+ 0 + 1}(i,0,1) - \frac{(0)}{\br{\frac{1}{2}}^2 + 0 + \frac{i}{2}\overline{\frac{i}{2}}}\br{\frac{1}{2}, 0, \frac{i}{2}} &= (0,1,0)
\end{align*}

So then $\cbr{(i,0,1)}$ is an orthogonal basis for $\mathsf{W}$, and $\cbr{ \br{\frac{1}{2}, 0, \frac{i}{2}},(0,1,0) }$ is an orthogonal basis for $\mathsf{W}^\perp$. Then normalize all three vectors to find that $\cbr{\br{\frac{\sqrt{2}i}{2} , 0 , \frac{\sqrt{2}}{2} }}$ is an orthonormal basis for $\mathsf{W}$ and $\cbr{ \br{\frac{\sqrt{2}}{2} , 0 , \frac{\sqrt{2}i}{2}} , \br{0,1,0} }$ is an orthonormal basis for $\mathsf{W}^\perp$.

15. Let $\mathsf{V}$ be a finite-dimensional inner product space over $\mathbb{F}$.

(a) \textit{Parseval's Identity.} Let $\cbr{v_1,v_2,\dots,v_n}$ be an orthonormal basis for $\mathsf{V}$. For any $x,y \in\mathsf{V}$, prove that $$\abr{x,y} = \sum_{i=1}^n\abr{x,v_i}\overline{\abr{y,v_i}}.$$

\begin{proof}
  Let $\cbr{v_1,v_2,\dots,v_n}$ be an orthonormal basis for $\mathsf{V}$, and let $x = \sum_{i=1}^n\abr{x,v_i}v_i$. Then by the linearity of the inner product in the first component and conjugation, we have $$\abr{x,y} = \abr{\sum_{i=1}^n\abr{x,v_i}v_i, y} = \sum_{i=1}^n\abr{x,v_i}\abr{v_i,y} = \sum_{i=1}^n\abr{x,v_i}\overline{\abr{y, v_i}}.$$ Hence Parseval's Identity holds.
\end{proof}

\end{document}