\documentclass[11pt]{article}

% packages
\usepackage{physics}
% margin spacing
\usepackage[top=1in, bottom=1in, left=0.5in, right=0.5in]{geometry}
\usepackage{hanging}
\usepackage{amsfonts, amsmath, amssymb, amsthm}
\usepackage{systeme}
\usepackage[none]{hyphenat}
\usepackage{fancyhdr}
\usepackage[nottoc, notlot, notlof]{tocbibind}
\usepackage{graphicx}
\graphicspath{{./images/}}
\usepackage{float}
\usepackage{siunitx}
\usepackage{esint}
\usepackage{cancel}
\usepackage{enumitem}

% permutations (second line is for spacing)
\usepackage{permute}
\renewcommand*\pmtseparator{\,}

% colors
\usepackage{xcolor}
\definecolor{p}{HTML}{FFDDDD}
\definecolor{g}{HTML}{D9FFDF}
\definecolor{y}{HTML}{FFFFCF}
\definecolor{b}{HTML}{D9FFFF}
\definecolor{o}{HTML}{FADECB}
%\definecolor{}{HTML}{}

% \highlight[<color>]{<stuff>}
\newcommand{\highlight}[2][p]{\mathchoice%
  {\colorbox{#1}{$\displaystyle#2$}}%
  {\colorbox{#1}{$\textstyle#2$}}%
  {\colorbox{#1}{$\scriptstyle#2$}}%
  {\colorbox{#1}{$\scriptscriptstyle#2$}}}%

% header/footer formatting
\pagestyle{fancy}
\fancyhead{}
\fancyfoot{}
\fancyhead[L]{MAS5311}
\fancyhead[C]{Assignment 7}
\fancyhead[R]{Sai Sivakumar}
\fancyfoot[R]{\thepage}
\renewcommand{\headrulewidth}{1pt}

% paragraph indentation/spacing
\setlength{\parindent}{0cm}
\setlength{\parskip}{10pt}
\renewcommand{\baselinestretch}{1.25}

% extra commands defined here
\newcommand{\ihat}{\boldsymbol{\hat{\textbf{\i}}}}
\newcommand{\jhat}{\boldsymbol{\hat{\textbf{\j}}}}
\newcommand{\dr}{\vec{r}~^{\prime}(t)}
\newcommand{\dx}{x^{\prime}(t)}
\newcommand{\dy}{y^{\prime}(t)}

\newcommand{\br}[1]{\left(#1\right)}
\newcommand{\sbr}[1]{\left[#1\right]}
\newcommand{\cbr}[1]{\left\{#1\right\}}

\newcommand{\dprime}{\prime\prime}
\newcommand{\lap}[2]{\mathcal{L}[#1](#2)}

\newcommand{\divides}{\mid}

% bracket notation for inner product
\usepackage{mathtools}

\DeclarePairedDelimiterX{\abr}[1]{\langle}{\rangle}{#1}

\DeclareMathOperator{\Span}{span}
\DeclareMathOperator{\nullity}{nullity}
\DeclareMathOperator\Aut{Aut}
\DeclareMathOperator\Inn{Inn}
\DeclareMathOperator{\Orb}{Orb}
\DeclareMathOperator{\lcm}{lcm}

% set page count index to begin from 1
\setcounter{page}{1}

\begin{document}
Let $G$ be a group and let $A$ be a nonempty set.
\begin{enumerate}
    \item (DF4.1.1) Let $G$ act on the set $A$. Prove that if $a,b\in A$ and $b = g\cdot a$ for some $g\in G$, then $G_b = gG_ag^{-1}$. Deduce that if $G$ acts transitively on $A$ then the kernel of the action is $\cap_{g\in G}~ gG_ag^{-1}$.
    \begin{proof}
        Let $G$ act on $A$ with $b = g\cdot a$ for $a,b\in A$ for some $g\in G$. We also have that $g^{-1}\cdot b = g^{-1}\cdot (g\cdot a) = (g^{-1}g)\cdot a = 1\cdot a = a$.

        With $gG_ag^{-1} = \cbr{gxg^{-1}\mid x\in G_a}$, observe that any element $gxg^{-1}\in gG_ag^{-1}$ satisfies \[(gxg^{-1})\cdot b = (gx)\cdot (g^{-1}\cdot b) = g\cdot (x\cdot a) = g\cdot a = b,\] so that $gxg^{-1} \in G_b$. Hence $gG_ag^{-1} \subseteq G_b$.

        Similarly, observe that for any $y \in G_b$, we may find $ghg^{-1}\in gG_ag^{-1}$ such that $y = ghg^{-1}$. Choose $h = g^{-1}yg$, where indeed $h = g^{-1}yg\in G_a$ because \[h\cdot a = (g^{-1}yg)\cdot a = (g^{-1}y)\cdot b = g^{-1}\cdot b = a.\]
        Then $y\in gG_ag^{-1}$, and hence $G_b\subseteq gG_ag^{-1}$.

        If $G$ acts transitively on $A$; that is, there is only one orbit and so for any $a,c\in A$, there is some $g\in G$ such that $a = g\cdot c$. We may obtain the kernel of this action by finding $\cap_{c\in A}~ G_c$, but because this action is transitive on $A$, we may use the above result to rewrite this set intersection.
        
        For $a,c\in A$, there exists $g\in G$ such that $G_c = gG_ag^{-1}$; as a result, if we fix $a$ and let $g$ take on every element in $G$, then the sets $gG_ag^{-1}$ take on every $G_c$ for $c\in A$. Hence $\cap_{c\in A}~ G_c = \cap_{g\in G}~ gG_ag^{-1}$, which is the kernel of the transitive action of $G$ on $A$.
    \end{proof}
    \item (DF4.1.4) Let $S_3$ act on the set $\Omega$ of ordered pairs: $\cbr{(i,j)\mid 1\leq i,j\leq 3}$ by $\sigma((i,j)) = (\sigma(i),\sigma(j))$. Find the orbits of $S_3$ on $\Omega$. For each $\sigma\in S_3$ find the cycle decomposition of $\sigma$ under this action (i.e., find its cycle decomposition when $\sigma$ is considered as an element of $S_9$ \textbf{---} first fix a labelling of these nine ordered pairs). For each orbit $\mathcal{O}$ of $S_3$ acting on these nine points pick some $a\in \mathcal{O}$ and find the stabilizer of $a$ in $S_3$.
    
    The orbit of $S_3$ containing $a\in \Omega$ takes on the form $\cbr{\sigma(a)\mid \sigma\in S_3}$, and we know that the group action will partition $A$ into disjoint orbits of this form. We find the orbits by taking the six permutations of $S_3$ and applying them to $(1,1)$ and $(1,2)$; we need not try any others since after this point we find all of the elements in $\Omega$. The following table exhibits this method:
    \begin{center}
        \begin{tabular}{ c|c|c }
         $\sigma$ & $\sigma((1,1))$ & $\sigma((1,2))$ \\
         \hline 
         $1$ & $(1,1)$ & $(1,2)$ \\
         $\pmt*{(12)}$ & $(2,2)$ & $(2,1)$ \\
         $\pmt*{(23)}$ & $(1,1)$ & $(1,3)$ \\
         $\pmt*{(13)}$ & $(3,3)$ & $(3,2)$ \\
         $\pmt*{(123)}$ & $(2,2)$ & $(2,3)$ \\
         $\pmt*{(132)}$ & $(3,3)$ & $(3,1)$
        \end{tabular}
    \end{center}
    So the two orbits that form are $\cbr{(c,c)\mid 1\leq c \leq 3}$ (the first column) and $\cbr{(a,b),(b,a)\mid a\neq b, 1\leq a,b\leq 3}$ (the second column). Notice they are disjoint and their union forms $\Omega$ as expected.

    We use a suggestive notation to simplify forming the cycle decomposition of $\sigma$ under this group action. Using the matrices below we can establish a labelling of the elements of $\Omega$:
    \[\begin{pmatrix}
        \mathbf{1} & \mathbf{2} & \mathbf{3} \\
        \mathbf{4} & \mathbf{5} & \mathbf{6} \\
        \mathbf{7} & \mathbf{8} & \mathbf{9} 
    \end{pmatrix} = \begin{pmatrix}
        (1,1) & (1,2) & (1,3) \\
        (2,1) & (2,2) & (2,3) \\
        (3,1) & (3,2) & (3,3) 
    \end{pmatrix}\]
    Then by tracking how each element in $S_3$ permutes $\cbr{\mathbf{1},\dots,\mathbf{9}}$, we can find the following cycle decompositions (viewing them as elements of $S_9$):
    \begin{center}
        \begin{tabular}{c|c}
            $\sigma$ & cycle decomposition for $\sigma$ \\
            \hline
            $1$ & $1$ \\
            $\pmt*{(12)}$ & $(\mathbf{1}\,\mathbf{5})(\mathbf{2}\,\mathbf{4})(\mathbf{3}\,\mathbf{6})(\mathbf{7}\,\mathbf{8})(\mathbf{9})$ \\
            $\pmt*{(23)}$ & $(\mathbf{2}\,\mathbf{3})(\mathbf{4}\,\mathbf{7})(\mathbf{5}\,\mathbf{9})(\mathbf{6}\,\mathbf{8})(\mathbf{1})$ \\
            $\pmt*{(13)}$ & $(\mathbf{1}\,\mathbf{9})(\mathbf{2}\,\mathbf{8})(\mathbf{3}\,\mathbf{7})(\mathbf{4}\,\mathbf{6})(\mathbf{5})$ \\
            $\pmt*{(123)}$ & $(\mathbf{1}\,\mathbf{5}\,\mathbf{9})(\mathbf{2}\,\mathbf{6}\,\mathbf{7})(\mathbf{3}\,\mathbf{4}\,\mathbf{8})$ \\
            $\pmt*{(132)}$ & $(\mathbf{1}\,\mathbf{9}\,\mathbf{5})(\mathbf{2}\,\mathbf{7}\,\mathbf{6})(\mathbf{3}\,\mathbf{8}\,\mathbf{4})$
        \end{tabular}
    \end{center}
    It is clear from these cycle decompositions that for $a\in \cbr{(c,c)\mid 1\leq c \leq 3}$ (the first orbit), the stabilizer of $a$ in $S_3$ is ${S_3}_a = \cbr{1,\pmt*{(xy)} \mid x,y\neq a, 1\leq x,y\leq 3}$; for example, $\pmt*{(12)}((3,3)) = (3,3)$ since $3$ is not found in the cycle $\pmt*{(12)}$. Then for $b\in \cbr{(a,b),(b,a)\mid a\neq b, 1\leq a,b\leq 3}$ (the second orbit), the stabilizer of $b$ in $S_3$ is ${S_3}_b = \cbr{1}$, since the only $1$-cycles present in any of the cycle decompositions above are those that fix elements from the first orbit.

    \item (DF4.1.10) Let $H$ and $K$ be subgroups of the group $G$. For each $x\in G$ define the $HK$ \textit{double coset} of $x$ in $G$ to be the set \[HxK = \cbr{hxk \mid h\in H,k\in K}.\]
    \begin{enumerate}[label=\textbf{(\alph*)}]
        \item Prove that $HxK$ is the union of the left cosets $x_1K,\dots,x_nK$ where $\cbr{x_1K,\dots,x_nK}$ is the orbit containing $xK$ of $H$ acting by left multiplication on the set of left cosets of $K$.
        \begin{proof}
            Let $\cbr{x_1K,\dots,x_nK}$ be the orbit containing $xK$ of $H$ acting by left multiplication on the set of left cosets of $K$ (without loss of generality each $x_iK$ here is distinct). This means that this set is equal to $\cbr{h\cdot xK \mid h\in H}$ (there is a bijection between the sets), so that $hxK$ for each $h\in H$ is equal to some $x_iK$ for $1\leq i\leq n$. Similarly, each $x_iK$ for $1\leq i\leq n$ is equal to $hxK$ for some $h\in H$.
            
            Consider any element $g\in HxK$, so that we may write $g = hxk$ for some $h\in H$ and $k\in K$. Then $g = hxk\in hxK$, but there exists an $x_jK$ in the orbit containing $xK$ of $H$ such that $g\in x_jK$ with $hxK = x_jK$. Then it is clear that $g \in \cup_{i=1}^n~x_iK$, and because $g$ was an arbitrary element of $HxK$, $HxK\subseteq \cup_{i=1}^n~x_iK$.

            Then take any element $g\in \cup_{i=1}^n~x_iK$, so that $g$ lies in some coset $x_jK$. There exists $hxK$ for some $h\in H$ such that $x_jK = hxK$, so $g\in hxK$. Then because $hxK\subseteq HxK$, it follows that $g\in HxK$. Hence $\cup_{i=1}^n~x_iK\subseteq HxK$, which means that $\cup_{i=1}^n~x_iK = HxK$.
        \end{proof}
        \item Prove that $HxK$ is a union of right cosets of $H$.
        \begin{proof}
            Similarly, let $\cbr{Hx_1,\dots,Hx_n}$ be the orbit containing $Hx$ of $K$ acting by right multiplication on the set of right cosets of $H$ (without loss of generality each $Hx_i$ here is distinct). Then $\cbr{Hx_1,\dots,Hx_n} = \cbr{Hx \cdot k \mid k\in K}$, so that $Hxk$ for each $k\in K$ is equal to some $Hx_i$ for $1\leq i\leq n$ and each $Hx_i$ for $1\leq i \leq n$ is equal to $Hxk$ for some $k\in K$.

            Then for any $g\in HxK$, we have that $g = hxk$ for some $k\in K$ and $h\in H$. Then $g = hxk\in Hxk$, but there exists some $Hx_j$ in the orbit containing $Hx$ of $K$ such that $g\in Hx_j = Hxk$. Thus $g\in \cup_{i=1}^n~ Hx_i$, so that $HxK\subseteq \cup_{i=1}^n~ Hx_i$.

            For any $g\in \cup_{i=1}^n~ Hx_i$, we have that $g\in Hx_j$ for some $1\leq j\leq n$. Then since $Hx_j$ is in the orbit containing $Hx$ of $K$, there exists $Hxk$ for some $k\in K$ such that $g\in Hxk = Hx_j$, which means that $g\in HxK$. Hence $\cup_{i=1}^n~ Hx_i\subseteq HxK$, so $HxK = \cup_{i=1}^n~ Hx_i$.
        \end{proof}
        \item Show that $HxK$ and $HyK$ are either the same set or are disjoint for all $x,y\in G$. Show that the set of $HK$ double cosets partitions $G$.
        \begin{proof}
            For any $g\in G$, we have that $g = 1g1 \in HgK$, since $H,K$ are subgroups of $G$. Then each $HgK$ for $g\in G$ contains $g$, so that $G = \cup_{g\in G}~ HgK$. What remains is to show that for two double cosets $HuK, HvK$ for $u,v\in G$, either $HuK \cap HvK = \varnothing$ or $HuK = HvK$. If $u=v$ then $HuK = HvK$, so suppose $u\neq v$.

            Then suppose that $HuK\cap HvK \neq \varnothing$, so that there exists $h_1uk_1\in HuK$ and $h_2vk_2\in HvK$ such that $h_1uk_1 = h_2vk_2$. Then it follows that $u = h_1^{-1}h_2vk_2k_1^{-1} = h_3vk_3$ for some $h_3\in H, k_3\in K$ (or $v = h_3uk_3$).

            So then for any element $huk\in HuK$, we have that $huk = hh_3vk_3k = h_4vk_4 \in HvK$, so that $HuK\subseteq HvK$. By interchanging the roles of $u$ and $v$ it follows that $ HvK\subseteq HuK$, so that $HuK = HvK$ whenever $HuK\cap HvK \neq \varnothing$.

            Thus the set of $HK$ double cosets partitions $G$, since each double coset is disjoint from each other and the union of the $HK$ double cosets was shown to be $G$.
        \end{proof}
        \item Prove that $\abs{HxK} = \abs{K}\cdot \abs{H\colon H\cap xKx^{-1}}$.
        \begin{proof}
            Consider the action of $H$ acting by left multiplication on the set of left cosets of $K$.
            Observe that the stabilizer $H_{xK}$ equals $H\cap xKx^{-1}$ because for $h\in H_{xK}$,
            \[h\cdot xK = hxK = xK\]
            implies that $x^{-1}hx\in K$, so that $h\in xKx^{-1}$. So combined with $h\in H$, we have that $H_{xK} = H\cap xKx^{-1}$.

            We saw from part \textbf{(a)} that the orbit containing $xK$ of $H$ was $\Orb(xK) = \cbr{x_1K,\dots, x_nK}$ (disjoint cosets), so $\abs{\Orb(xK)} = n$. Furthermore, we saw that $HxK = \cup_{i=1}^n~x_iK$, which because each $x_iK$ is disjoint and has the same cardinality as $K$, $\abs{HxK} = \abs{\cup_{i=1}^n~x_iK} = n\abs{K}$.
            
            Hence $\abs{HxK}/\abs{K} = n = \abs{\Orb(xK)}$, so by the Orbit-Stabilizer Theorem, we have that \[\abs{HxK}/\abs{K} = \abs{H\colon H_{xK}} = \abs{H\colon H\cap xKx^{-1}},\]
            which implies that $\abs{HxK} = \abs{K}\cdot \abs{H\colon H\cap xKx^{-1}}$.
        \end{proof}
        \item Prove that $\abs{HxK} = \abs{H}\cdot \abs{K\colon K\cap x^{-1}Hx}$.
        \begin{proof}
            Similarly, consider the action of $K$ acting by right multiplication on the set of right cosets of $H$.
            Observe that the stabilizer $K_{Hx}$ equals $K\cap x^{-1}Hx$ because for $k\in K_{Hx}$,
            \[Hx \cdot k = Hxk = Hx\]
            implies that $xkx^{-1} = H$, so that $k\in x^{-1}Hx$. With $k\in K$, we have that $K_{Hx} = k\cap x^{-1}Hx$.

            From part \textbf{(b)}, the orbit containing $Hx$ of $K$ was $\Orb(Hx) = \cbr{Hx_1,\dots,Hx_n}$ (disjoint cosets), so $\abs{\Orb(Hx)} = n$. We also had that $HxK = \cup_{i=1}^n~ Hx_i$, and because each $Hx_i$ is disjoint and has the same cardinality as $H$, $\abs{HxK} = \abs{\cup_{i=1}^n~ Hx_i} = n\abs{H}$.

            Hence $\abs{HxK}/\abs{H} = n = \abs{\Orb(Hx)}$, so by the Orbit-Stabilizer Theorem, we have that 
            \[\abs{HxK}/\abs{H} = \abs{K\colon K_{Hx}} = \abs{K\colon k\cap x^{-1}Hx},\]
            which implies that $\abs{HxK} = \abs{H}\cdot \abs{K\colon K\cap x^{-1}Hx}$.
        \end{proof}
    \end{enumerate}
    \item Q4. Let $G$ be a finite group and $H$ a subgroup. Consider the partition of $G$ into double cosets $HgH$  as in problem 10.
    \begin{enumerate}[label=\textbf{(\alph*)}]
        \item Prove that every left coset contained in a given double coset has nonempty intersection with every right coset contained in the same double coset.
        \begin{proof}
            Let $HgH$ be some given double coset, and let $xH$ be any left coset contained in $HgH$, and let $Hy$ be any right coset contained in $HgH$. Then due to containment, we may write $xH = h_1gH$ and $Hy = Hgh_2$, for $h_1,h_2\in H$. Because $h_1,h_2\in H$, we have that $h_1gh_2\in h_1gH$ and $h_1gh_2\in Hgh_2$, which means that $h_1gh_2\in h_1gH\cap Hgh_2 = xH\cap Hy$, so that the intersection is nonempty.

            Because $xH$ and $Hy$ were arbitrary cosets contained in $HgH$, it follows that every left coset contained in a given double coset has nonempty intersection with every right coset contained in the same double coset.
        \end{proof}
        \item Deduce that if $n=|G:H|$ then there exist elements $g_1,\dots, g_n$ in $G$ that belong to distinct left cosets and to distinct right cosets.
    
        This means that $G$ is the disjoint union of the $Hg_i$ and also the disjoint union of the $g_iH$.
        \begin{proof}
            Let $n=|G:H|$, so that there are exactly $n$ disjoint left cosets of $H$ in $G$, and exactly $n$ disjoint right cosets of $H$ in $G$.
            
            Considering the partition of $G$ into distinct double cosets $HgH$, observe that for each $HgH$, there are left cosets $h_1gH\subseteq HgH$ and right cosets $Hgh_2\subseteq HgH$ such that for any $h_1,h_2\in H$, we have $h_1gH\cap Hgh_2\neq \varnothing$ from part \textbf{(a)}. We saw earlier that one such element in the intersection was $h_1gh_2$, so that this element is a representative for the left and right coset we identified. So we can write $h_1gH = h_1gh_2H$ and $Hgh_2 = Hh_1gh_2$. What remains is to find $n$ many left (right) cosets in this form which are all disjoint from each other, because there are exactly $n$ disjoint left (right) cosets of $H$ in $G$.

            In the double coset $HgH$ there may be $n^{\prime}\leq n$ many left (right) cosets that are contained in $HgH$. For some left coset $a_iH\subseteq HgH$, it will have a nonempty intersection with some right coset $Hb_i\subseteq HgH$, with $g_i$ being an element of this intersection. Hence it is a representative for both, so that $a_iH = g_iH$ and $Hb_i = Hg_i$. We can pick some other left coset $a_j H\subseteq HgH$ and a right coset $Hb_j \subseteq HgH$ where $a_j$ is disjoint from $a_i$ and $b_j$ is disjoint from $b_i$, and again find a single representative $g_j$ to form the disjoint left (right) cosets $g_jH$ and $Hg_j$.
            
            We may do this $n^{\prime}$ many times to find $n^{\prime}\leq n$ many disjoint left and right cosets $g_iH$ and $Hg_i$ of $G$. After exhausting $HgH$ of all of these cosets, if we do not have $n$ many disjoint cosets, we consider another distinct double coset $Hg^{\prime}H$, and find more (say $n^{\dprime}$ many more) distinct cosets $g_kH$ and $Hg_k$ which because $HgH$ and $Hg^{\prime}H$ are disjoint, it forces these new cosets $g_kH$ and $Hg_k$ to be disjoint to the old ones $g_iH$ and $Hg_i$ (left and right cosets respectively).

            We may do this for every double coset which is part of the partition of $G$ because they are disjoint from each other. Because the union of each of these double cosets is $G$, we must necessarily be able to find $n$ many left and right cosets $g_iH$ and $Hg_i$ in total, after finding a smaller number (or equal number if there is only one double coset) of left and right cosets in each of the $HgH$ that partition $G$.

            Therefore there exist $g_1,\dots, g_n$ in $G$ that belong to distinct left cosets and to distinct right cosets. Thus the disjoint unions $\cup_{i= 1}^n~ g_iH$ and $\cup_{i= 1}^n~ Hg_i$ are equal to $G$.
        \end{proof}
    \end{enumerate}
\end{enumerate}
\end{document}