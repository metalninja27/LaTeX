\documentclass[11pt]{article}

% packages
\usepackage{physics}
% margin spacing
\usepackage[top=1in, bottom=1in, left=0.5in, right=0.5in]{geometry}
\usepackage{hanging}
\usepackage{amsfonts, amsmath, amssymb, amsthm}
\usepackage{systeme}
\usepackage[none]{hyphenat}
\usepackage{fancyhdr}
\usepackage[nottoc, notlot, notlof]{tocbibind}
\usepackage{graphicx}
\graphicspath{{./images/}}
\usepackage{float}
\usepackage{siunitx}
\usepackage{esint}
\usepackage{cancel}

% permutations (second line is for spacing)
\usepackage{permute}
\renewcommand*\pmtseparator{\,}

% colors
\usepackage{xcolor}
\definecolor{p}{HTML}{FFDDDD}
\definecolor{g}{HTML}{D9FFDF}
\definecolor{y}{HTML}{FFFFCF}
\definecolor{b}{HTML}{D9FFFF}
\definecolor{o}{HTML}{FADECB}
%\definecolor{}{HTML}{}

% \highlight[<color>]{<stuff>}
\newcommand{\highlight}[2][p]{\mathchoice%
  {\colorbox{#1}{$\displaystyle#2$}}%
  {\colorbox{#1}{$\textstyle#2$}}%
  {\colorbox{#1}{$\scriptstyle#2$}}%
  {\colorbox{#1}{$\scriptscriptstyle#2$}}}%

% header/footer formatting
\pagestyle{fancy}
\fancyhead{}
\fancyfoot{}
\fancyhead[L]{\textbf{Induction}}
\fancyhead[C]{}
\fancyhead[R]{Sai Sivakumar}
\fancyfoot[R]{\thepage}
\renewcommand{\headrulewidth}{0pt}

% paragraph indentation/spacing
\setlength{\parindent}{0cm}
\setlength{\parskip}{10pt}
\renewcommand{\baselinestretch}{1.25}

% extra commands defined here
\newcommand{\ihat}{\boldsymbol{\hat{\textbf{\i}}}}
\newcommand{\jhat}{\boldsymbol{\hat{\textbf{\j}}}}
\newcommand{\dr}{\vec{r}~^{\prime}(t)}
\newcommand{\dx}{x^{\prime}(t)}
\newcommand{\dy}{y^{\prime}(t)}

\newcommand{\br}[1]{\left(#1\right)}
\newcommand{\sbr}[1]{\left[#1\right]}
\newcommand{\cbr}[1]{\left\{#1\right\}}

\newcommand{\dprime}{\prime\prime}
\newcommand{\lap}[2]{\mathcal{L}[#1](#2)}

\newcommand{\divides}{\mid}

% bracket notation for inner product
\usepackage{mathtools}

\DeclarePairedDelimiterX{\abr}[1]{\langle}{\rangle}{#1}

\DeclareMathOperator{\Span}{span}
\DeclareMathOperator{\nullity}{nullity}
\DeclareMathOperator\Aut{Aut}
\DeclareMathOperator\Inn{Inn}

% set page count index to begin from 1
\setcounter{page}{1}

\begin{document}
\textbf{8.} Let $n\geq 0$ and $x\in[0,\pi]$. Prove that $\abs{\sin(nx)}\leq n\sin(x)$. \begin{proof}
    We will show this by induction. Observe for $n = 0$, $\abs{\sin(0x)} = 0 \leq 0 = 0\sin(x)$. Then suppose that $\abs{\sin(nx)}\leq n\sin(x)$, and we will show that $\abs{\sin((n+1)x)}\leq (n+1)\sin(x)$. So
    \begin{align*}
        \abs{\sin((n+1)x)} = \abs{\sin(nx+x)} &= \abs{\sin(nx)\cos(x) + \sin(x)\cos(nx)}\\
        &\leq \abs{\sin(nx)\cos(x)}+\abs{\sin(x)\cos(nx)}\\
        &= \abs{\sin(nx)}\abs{\cos(x)}+\sin(x)\abs{\cos(nx)},
        \intertext{where the triangle inequality was used as well as omitting the absolute value bars for $\sin(x)$ since on $[0,\pi]$, $\sin(x)$ is nonnegative. Then from the inductive hypothesis,}
        &\leq n\sin(x)\abs{\cos(x)} + \sin(x)\abs{\cos(nx)}\\
        &\leq n\sin(x) + \sin(x) = (n+1)\sin(x),
    \end{align*} because $\cos(x)$ and $\cos(nx)$ are bounded functions so we may bound them above by $1$.
    
    Hence $\abs{\sin((n+1)x)}\leq (n+1)\sin(x)$, and by induction we have that for all $n\geq 0$ and $x\in[0,\pi]$, $\abs{\sin(nx)}\leq n\sin(x)$.
\end{proof}

\textbf{13.} Find the $100$-th derivative of $f(x) = 1/(5-x^2)$. 

The $100$-th derivative of $f(x)$ is $\frac{\sqrt{5}}{10}\left(\frac{100!}{(x+\sqrt{5})^{101}} -\frac{100!}{(x-\sqrt{5})^{101}}\right)$.

We can find a closed form for the $n$-th derivative of $f(x)$ (where $n = 0$ means to take no derivatives). First we can rewrite $f(x)$ by partial fraction decomposition, so that 
\[f(x) = \frac{1}{5-x^2} = \frac{\sqrt{5}}{10}\left(\frac{1}{x+\sqrt{5}} -\frac{1}{x-\sqrt{5}}\right).\] Then by taking successive derivatives, we find that 
\[f^{(n)}(x) = \frac{\sqrt{5}}{10}\left(\frac{(-1)^nn!}{(x+\sqrt{5})^{n+1}} -\frac{(-1)^nn!}{(x-\sqrt{5})^{n+1}}\right),\] and we prove that this is the closed form for the $n$-th derivative of $f(x)$.
\begin{proof}
    Let $f$ be given as above. Then for $n = 0$, we saw earlier that 
    \[f(x) = \frac{1}{5-x^2} = \frac{\sqrt{5}}{10}\left(\frac{1}{x+\sqrt{5}} -\frac{1}{x-\sqrt{5}}\right),\] and for $n=1$ we also have 
    \[f^{\prime}(x) = \frac{2x}{(5-x^2)^2} = \frac{\sqrt{5}}{10}\left(\frac{(-1)}{(x+\sqrt{5})^2} -\frac{(-1)}{(x-\sqrt{5})^2}\right).\]

    Then suppose that the $n$-th derivative of $f$ is given as above, so that $f^{(n)}(x) = \frac{\sqrt{5}}{10}\left(\frac{(-1)^nn!}{(x+\sqrt{5})^{n+1}} -\frac{(-1)^nn!}{(x-\sqrt{5})^{n+1}}\right)$. Then to show that this formula holds for the $n+1$-th derivative, see that \begin{align*}
        f^{(n+1)}(x) = \dv{x}f^{(n)}(x) &= \dv{x} \frac{\sqrt{5}}{10}\left(\frac{(-1)^nn!}{(x+\sqrt{5})^{n+1}} -\frac{(-1)^nn!}{(x-\sqrt{5})^{n+1}}\right)\\
        &= \frac{\sqrt{5}}{10}\left(\dv{x}\frac{(-1)^nn!}{(x+\sqrt{5})^{n+1}} -\dv{x}\frac{(-1)^nn!}{(x-\sqrt{5})^{n+1}}\right)\\
        &= \frac{\sqrt{5}}{10}\left(\frac{(-1)^nn!\cdot (-(n+1))}{(x+\sqrt{5})^{(n+1)+1}} - \frac{(-1)^nn!\cdot (-(n+1))}{(x-\sqrt{5})^{(n+1)+1}}\right)\\
        &= \frac{\sqrt{5}}{10}\left(\frac{(-1)^{n+1}(n+1)!}{(x+\sqrt{5})^{n+2}} -\frac{(-1)^{n+1}(n+1)!}{(x-\sqrt{5})^{n+2}}\right).
    \end{align*}
    Therefore, by induction, the formula given is the closed form for the $n$-th derivative of $f$.
\end{proof}

Then we may take $n=100$ to find that \[f^{(100)}(x) = \frac{\sqrt{5}}{10}\left(\frac{100!}{(x+\sqrt{5})^{101}} -\frac{100!}{(x-\sqrt{5})^{101}}\right).\]
\end{document}