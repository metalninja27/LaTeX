\documentclass[11pt]{article}

% packages
\usepackage{physics}
% margin spacing
\usepackage[top=1in, bottom=1in, left=0.5in, right=0.5in]{geometry}
\usepackage{hanging}
\usepackage{amsfonts, amsmath, amssymb, amsthm}
\usepackage{systeme}
\usepackage[none]{hyphenat}
\usepackage{fancyhdr}
\usepackage[nottoc, notlot, notlof]{tocbibind}
\usepackage{graphicx}
\graphicspath{{./images/}}
\usepackage{float}
\usepackage{siunitx}
\usepackage{esint}
\usepackage{cancel}

% permutations (second line is for spacing)
\usepackage{permute}
\renewcommand*\pmtseparator{\,}

% colors
\usepackage{xcolor}
\definecolor{p}{HTML}{FFDDDD}
\definecolor{g}{HTML}{D9FFDF}
\definecolor{y}{HTML}{FFFFCF}
\definecolor{b}{HTML}{D9FFFF}
\definecolor{o}{HTML}{FADECB}
%\definecolor{}{HTML}{}

% \highlight[<color>]{<stuff>}
\newcommand{\highlight}[2][p]{\mathchoice%
  {\colorbox{#1}{$\displaystyle#2$}}%
  {\colorbox{#1}{$\textstyle#2$}}%
  {\colorbox{#1}{$\scriptstyle#2$}}%
  {\colorbox{#1}{$\scriptscriptstyle#2$}}}%

% header/footer formatting
\pagestyle{fancy}
\fancyhead{}
\fancyfoot{}
\fancyhead[L]{\textbf{Integral Problems}}
\fancyhead[C]{}
\fancyhead[R]{Sai Sivakumar}
\fancyfoot[R]{\thepage}
\renewcommand{\headrulewidth}{0pt}

% paragraph indentation/spacing
\setlength{\parindent}{0cm}
\setlength{\parskip}{10pt}
\renewcommand{\baselinestretch}{1.25}

% extra commands defined here
\newcommand{\ihat}{\boldsymbol{\hat{\textbf{\i}}}}
\newcommand{\jhat}{\boldsymbol{\hat{\textbf{\j}}}}
\newcommand{\dr}{\vec{r}~^{\prime}(t)}
\newcommand{\dx}{x^{\prime}(t)}
\newcommand{\dy}{y^{\prime}(t)}

\newcommand{\br}[1]{\left(#1\right)}
\newcommand{\sbr}[1]{\left[#1\right]}
\newcommand{\cbr}[1]{\left\{#1\right\}}

\newcommand{\dprime}{\prime\prime}
\newcommand{\lap}[2]{\mathcal{L}[#1](#2)}

\newcommand{\divides}{\mid}

% bracket notation for inner product
\usepackage{mathtools}

\DeclarePairedDelimiterX{\abr}[1]{\langle}{\rangle}{#1}

\DeclareMathOperator{\Span}{span}
\DeclareMathOperator{\nullity}{nullity}
\DeclareMathOperator\Aut{Aut}
\DeclareMathOperator\Inn{Inn}

% set page count index to begin from 1
\setcounter{page}{1}

\begin{document}
\textbf{1.} Evaluate $\displaystyle{\int_0^{\infty}\frac{\arctan(\pi x) - \arctan(x)}{x}\dd{x}}$.

$\displaystyle{\int_0^{\infty}\frac{\arctan(\pi x) - \arctan(x)}{x}\dd{x}} = \frac{1}{2}\pi\log(\pi)$.

Continuously extend the integrand by defining it to be $\pi-1$ when $x = 0$ so that we do not need to worry about taking an improper integral as $x$ tends to $0$. We still have to take the principal value of an improper integral as $x$ tends to $\infty$. Simplifying, we have
\begin{align*}
    \int_0^{\infty}\frac{\arctan(\pi x) - \arctan(x)}{x}\dd{x} &= \lim_{R\to\infty}\sbr{\int_0^{R}\frac{\arctan(\pi x^{\prime})}{x^{\prime}}\dd{x^{\prime}} - \int_0^{R}\frac{\arctan(x)}{x}\dd{x}}
    \intertext{Using the substitution $x = \pi x^{\prime}$, $\dd{x} = \pi \dd{x^{\prime}}$ in the first integral, continue:}
    &= \lim_{R\to\infty}\sbr{\int_0^{\pi R}\frac{\arctan( x)}{x}\dd{x} - \int_0^{R}\frac{\arctan(x)}{x}\dd{x}}\\
    &= \lim_{R\to\infty}\int_R^{\pi R}\frac{\arctan( x)}{x}\dd{x}.
\end{align*} Then we argue that the limit is $\pi\log(\pi)/2$. Given any $\varepsilon>0$, we can find $R^{\prime}>0$ such that for $x>R^{\prime}$, we have that $\abs{\pi/2-\arctan(x)} < \varepsilon$.

Observe that $\int_R^{\pi R}\pi/(2x)\dd{x} = \pi(\log(\pi R) - \log(R))/2 = \pi\log(\pi)/2$. Then by choosing $R$ greater than $R^{\prime}$, \begin{align*}
    \abs{\frac{1}{2}\pi\log(\pi) - \int_R^{\pi R}\frac{\arctan( x)}{x}\dd{x}}=\abs{\int_R^{\pi R} \frac{\pi/2 - \arctan(x)}{x}\dd{x}} &\leq \int_R^{\pi R} \frac{\abs{\pi/2 - \arctan(x)}}{x}\dd{x}\\
    &< \int_R^{\pi R} \frac{\varepsilon}{x}\dd{x}\\
    &= \varepsilon\log(\pi).
\end{align*}
Since $\varepsilon$ was arbitrary, as $\varepsilon\to 0$ and as $R\to\infty$, it follows that \[\lim_{R\to\infty}\int_R^{\pi R}\frac{\arctan( x)}{x}\dd{x} = \frac{1}{2}\pi\log(\pi)\] as desired.

\textbf{8.} Evaluate $\displaystyle{\int_1^{\infty} \frac{\dd{x}}{e^{x+1} + e^{3-x}}}$.

$\displaystyle{\int_1^{\infty} \frac{\dd{x}}{e^{x+1} + e^{3-x}}} = \frac{\pi}{4e^2}$.

Start by making the change of variables $x^{\prime} = x + 1, \dd{x^{\prime}} = \dd{x}$ so that \[\int_1^{\infty} \frac{\dd{x^{\prime}}}{e^{x^{\prime}+1} + e^{3-x^{\prime}}} = \int_0^{\infty}\frac{\dd{x}}{e^{x+2} + e^{-(x-2)}}.\] Then with some algebra and another change of variables $u = e^x, \dd{u} = e^x\dd{x}$, we have\begin{align*}
    \int_0^{\infty}\frac{\dd{x}}{e^{x+2} + e^{-(x-2)}} &= \frac{1}{e^2}\int_0^{\infty}\frac{\dd{x}}{e^{x} + e^{-x}}\\
    &= \frac{1}{e^2}\int_0^{\infty}\frac{e^x\dd{x}}{1 + e^{2x}}\\
    &= \frac{1}{e^2}\int_1^{\infty}\frac{\dd{u}}{1 + u^2}\\
    &= \frac{1}{e^2}\br{\eval{\arctan(u)}_1^{\infty}}\\
    &=\frac{1}{e^2}\br{\frac{\pi}{2}-\frac{\pi}{4}} = \frac{\pi}{4e^2}
\end{align*} as desired.
\end{document}