\documentclass[11pt]{article}

% packages
\usepackage{physics}
% margin spacing
\usepackage[top=1in, bottom=1in, left=0.5in, right=0.5in]{geometry}
\usepackage{hanging}
\usepackage{amsfonts, amsmath, amssymb, amsthm}
\usepackage{systeme}
\usepackage[none]{hyphenat}
\usepackage{fancyhdr}
\usepackage[nottoc, notlot, notlof]{tocbibind}
\usepackage{graphicx}
\graphicspath{{./images/}}
\usepackage{float}
\usepackage{siunitx}
\usepackage{esint}
\usepackage{cancel}

% permutations (second line is for spacing)
\usepackage{permute}
\renewcommand*\pmtseparator{\,}

% colors
\usepackage{xcolor}
\definecolor{p}{HTML}{FFDDDD}
\definecolor{g}{HTML}{D9FFDF}
\definecolor{y}{HTML}{FFFFCF}
\definecolor{b}{HTML}{D9FFFF}
\definecolor{o}{HTML}{FADECB}
%\definecolor{}{HTML}{}

% \highlight[<color>]{<stuff>}
\newcommand{\highlight}[2][p]{\mathchoice%
  {\colorbox{#1}{$\displaystyle#2$}}%
  {\colorbox{#1}{$\textstyle#2$}}%
  {\colorbox{#1}{$\scriptstyle#2$}}%
  {\colorbox{#1}{$\scriptscriptstyle#2$}}}%

% header/footer formatting
\pagestyle{fancy}
\fancyhead{}
\fancyfoot{}
\fancyhead[L]{\textbf{Strong Induction}}
\fancyhead[C]{}
\fancyhead[R]{Sai Sivakumar}
\fancyfoot[R]{\thepage}
\renewcommand{\headrulewidth}{0pt}

% paragraph indentation/spacing
\setlength{\parindent}{0cm}
\setlength{\parskip}{10pt}
\renewcommand{\baselinestretch}{1.25}

% extra commands defined here
\newcommand{\ihat}{\boldsymbol{\hat{\textbf{\i}}}}
\newcommand{\jhat}{\boldsymbol{\hat{\textbf{\j}}}}
\newcommand{\dr}{\vec{r}~^{\prime}(t)}
\newcommand{\dx}{x^{\prime}(t)}
\newcommand{\dy}{y^{\prime}(t)}

\newcommand{\br}[1]{\left(#1\right)}
\newcommand{\sbr}[1]{\left[#1\right]}
\newcommand{\cbr}[1]{\left\{#1\right\}}

\newcommand{\dprime}{\prime\prime}
\newcommand{\lap}[2]{\mathcal{L}[#1](#2)}

\newcommand{\divides}{\mid}

% bracket notation for inner product
\usepackage{mathtools}

\DeclarePairedDelimiterX{\abr}[1]{\langle}{\rangle}{#1}

\DeclareMathOperator{\Span}{span}
\DeclareMathOperator{\nullity}{nullity}
\DeclareMathOperator\Aut{Aut}
\DeclareMathOperator\Inn{Inn}

% set page count index to begin from 1
\setcounter{page}{1}

\begin{document}
\textbf{11.} Let $m,n$ be integers, and set $d = \gcd(m,n)$. Prove that there are integers $x,y$ such that $mx+ny = d$.
\begin{proof}
    It suffices to prove the theorem for $m,n$ being nonnegative integers since we can pass any negative signs into the coefficients $x,y$ (or pass them back into $\abs{m},\abs{n}$); that is, if $m$ is negative and we have coefficients $x,y$ that satisfy $mx+ny = d$, then we also have that $\abs{m}(-x) + ny = d$, so that $d = \gcd(m,n) = \gcd(\abs{m},n)$. We can do the same if $n$ was negative, or if both $m$ and $n$ were negative.

    Let $m,n$ be nonnegative integers and $d = \gcd(m,n)$ as given, where without loss of generality, let $n\geq m$. By strong induction on $h\geq 0$, suppose that for nonnegative integers $a,b$ where $a+b < h$ there exist integers $s,t$ such that $d = \gcd(a,b) = as+bt$. When $h = n+m = 0$, then both $m,n$ are zero and so $x,y$ are both zero as well (where the greatest common of $0$ and $0$ is $0$ because the divisor should be less than or equal to $0$ in the nonnegative integers). If $m$ (the lesser of the two) is $0$ and $n$ is nonzero (so that $h = n$), then the greatest common divisor is $n$, and we find that $n = n(1) + m(0)$. Then let $m\geq 1$, with $h = n+m$.

    Consider $\gcd(m,n-m)$. Observe that $m+(n-m) = n = h-m < h$, since $m\geq 1$. Then by the inductive hypothesis, there exist integers $a,b$ such that $\gcd(m,n-m) = ma+(n-m)b = m(a-b) + nb$. This quantity is actually the greatest common divisor of $m$ and $n$. Any common divisor of $m$ and $n$ will divide the quantity $m(a-b) + nb$ since this is a linear combination of $m,n$. Hence there are integers $x = (a-b), y = b$ such that $d = \gcd(m,n) = mx+ny$.

    Therefore, by mathematical induction, for any integers $m,n$, there exist integers $x,y$ such that $d = \gcd(m,n) = mx+ny$.
\end{proof}

\textbf{14.} Let $x$ be a real number such that $x+x^{-1}$ is an integer. Prove that $x^n+x^{-n}$ is an integer for all positive integers $n$.
\begin{proof}
    Let $x\in \mathbb{R}$ be given so that $x+x^{-1}\in \mathbb{Z}$ as given. Then by strong induction on $n$, suppose that for $1\leq k < n$, $x^k+x^{-k}\in \mathbb{Z}$.

    Since $x+x^{-1}\in \mathbb{Z}$, we have that $(x+x^{-1})^n \in \mathbb{Z}$. Then by binomial expansion, 
    \[(x+x^{-1})^n = x^n+x^{-n} + \sum_{k=1}^{n-1} \binom{n}{k}x^k(x^{-1})^{n-k},\]
    and we can further simplify this using the symmetry of binomial coefficients, where $\binom{n}{k} = \binom{n}{n-k}$. However, we must handle an ``extra'' constant term that forms when $n$ is even (where $k=n-k$ for some $k$). So in the first case when $n$ is even, write $n = 2a$ for some positive integer $a$. Then substitute and simplify using the symmetry of binomial coefficients to find 
    \[x^n+x^{-n} + \sum_{k=1}^{2a-1} \binom{2a}{k}x^k(x^{-1})^{2a-k} = x^n+x^{-n} + \binom{2a}{a} + \sum_{k=1}^{a-1} \binom{2a}{k}\left[x^{2(a-k)} + (x^{-1})^{2(a-k)}\right],\]
    and observe that $\binom{2a}{a}$ is an integer. By the inductive hypothesis, all of the terms $\binom{2a}{k}\left[x^{2(a-k)} + (x^{-1})^{2(a-k)}\right]$ for $1\leq k \leq a-1$ are integers as well since $2(a-k) < n$. Then it follows that $x^n+x^{-n} = (x+x^{-1})^n - \binom{2a}{a} - \sum_{k=1}^{a-1} \binom{2a}{k}\left[x^{2(a-k)} + (x^{-1})^{2(a-k)}\right]$ is an integer since a sum of integers is an integer.

    Similarly, when $n$ is odd, write $n=2b+1$ for some positive integer $b$, and we have that 
    \[x^n+x^{-n} + \sum_{k=1}^{2b} \binom{2b+1}{k}x^k(x^{-1})^{2b+1-k} = x^n+x^{-n} + \sum_{k=1}^{b} \binom{2b+1}{k}\left[x^{2(b-k)+1} + (x^{-1})^{2(b-k)+1}\right],\]
    where because $1\leq k\leq b$, $2(b-k)+1 < n$, all of the terms $\binom{2b+1}{k}\left[x^{2(b-k)+1} + (x^{-1})^{2(b-k)+1}\right]$ are integers. Again, we have that $x^n+x^{-n} = (x+x^{-1})^n - \sum_{k=1}^{b} \binom{2b+1}{k}\left[x^{2(b-k)+1} + (x^{-1})^{2(b-k)+1}\right]$ is an integer.

    Hence in both cases $x^n+x^{-n}$ is an integer, and by mathematical induction, $x^n+x^{-n}$ is an integer for all positive integers $n$.
\end{proof}
\end{document}