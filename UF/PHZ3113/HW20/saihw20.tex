\documentclass[11pt]{article}
\headheight = 14pt

% packages
\usepackage{physics}
% margin spacing
\usepackage[top=1in, bottom=1in, left=0.5in, right=0.5in]{geometry}
\usepackage{hanging}
\usepackage{amsfonts, amsmath, amssymb, amsthm}
\usepackage{systeme}
\usepackage[none]{hyphenat}
\usepackage{fancyhdr}
\usepackage[nottoc, notlot, notlof]{tocbibind}
\usepackage{graphicx}
\graphicspath{{./images/}}
\usepackage{float}
\usepackage{siunitx}
\usepackage{esint}
\usepackage{cancel}
\usepackage{enumitem}

% colors
\usepackage{xcolor}
\definecolor{p}{HTML}{FFDDDD}
\definecolor{g}{HTML}{D9FFDF}
\definecolor{y}{HTML}{FFFFCF}
\definecolor{b}{HTML}{D9FFFF}
\definecolor{o}{HTML}{FADECB}
%\definecolor{}{HTML}{}

% \highlight[<color>]{<stuff>}
\newcommand{\highlight}[2][p]{\mathchoice%
  {\colorbox{#1}{$\displaystyle#2$}}%
  {\colorbox{#1}{$\textstyle#2$}}%
  {\colorbox{#1}{$\scriptstyle#2$}}%
  {\colorbox{#1}{$\scriptscriptstyle#2$}}}%

% header/footer formatting
\pagestyle{fancy}
\fancyhead{}
\fancyfoot{}
\fancyhead[L]{PHZ3113}
\fancyhead[C]{HW20}
\fancyhead[R]{Sai Sivakumar}
\fancyfoot[R]{\thepage}
\renewcommand{\headrulewidth}{1pt}

% paragraph indentation/spacing
\setlength{\parindent}{0cm}
\setlength{\parskip}{5pt}
\renewcommand{\baselinestretch}{1.25}

% extra commands defined here
\newcommand{\ihat}{\boldsymbol{\hat{\textbf{\i}}}}
\newcommand{\jhat}{\boldsymbol{\hat{\textbf{\j}}}}
\newcommand{\khat}{\boldsymbol{\hat{\textbf{k}}}}
\newcommand{\dr}{\vec{r}~^{\prime}(t)}
\newcommand{\dx}{x^{\prime}(t)}
\newcommand{\dy}{y^{\prime}(t)}

\newcommand{\br}[1]{\left(#1\right)}
\newcommand{\sbr}[1]{\left[#1\right]}
\newcommand{\cbr}[1]{\left\{#1\right\}}

\newcommand{\dprime}{\prime\prime}
\newcommand{\lap}[2]{\mathcal{L}[#1](#2)}

% bracket notation for inner product
\usepackage{mathtools}

\DeclarePairedDelimiterX{\abr}[1]{\langle}{\rangle}{#1}

\DeclareMathOperator{\Span}{span}
\DeclareMathOperator\Arg{Arg}
\DeclareMathOperator\Log{Log}

% set page count index to begin from 1
\setcounter{page}{1}

\begin{document}
\begin{enumerate}[label=20.\arabic*]
    \item Consider an analytic function $W(z) = u(x,y) + iv(x,y)$.
    \begin{enumerate}[label=(\alph*)]
        \item Assuming all required derivatives exist, show that $\nabla^2u = \nabla^2v = 0$.
        \begin{proof}
            Let $W(z) = u(x,y) + iv(x,y)$ be sufficiently differentiable as given. Then observe that \[\pdv{x}\pdv{u}{x} = \pdv[2]{u}{x} = \pdv[2]{v}{y}{x} = \pdv{x} \pdv{v}{y}\] and \[\pdv{y}\pdv{v}{x} = \pdv[2]{v}{x}{y} = -\pdv[2]{u}{y} = -\pdv{y}\pdv{u}{y}.\]
            Then \[\boxed{\nabla^2u = \pdv[2]{u}{x} + \pdv[2]{u}{y} = \pdv[2]{v}{y}{x} - \pdv[2]{v}{x}{y} = \pdv[2]{v}{y}{x} - \pdv[2]{v}{y}{x} = 0},\] where the equality of mixed partial derivatives was used.
            Similarly, we have that \[\pdv{x}\pdv{v}{x} = \pdv[2]{v}{x} = -\pdv[2]{u}{y}{x} = -\pdv{x} \pdv{u}{y}\] and \[\pdv{y}\pdv{u}{x} = \pdv[2]{u}{x}{y} = \pdv[2]{v}{y} = \pdv{y}\pdv{v}{y}\] imply for the same reasons that \[\boxed{\nabla^2v = \pdv[2]{v}{x} + \pdv[2]{v}{y} = -\pdv[2]{u}{y}{x} + \pdv[2]{u}{x}{y} = -\pdv[2]{u}{y}{x} + \pdv[2]{u}{y}{x} = 0}.\]
        \end{proof}
        \item Show that \[\pdv{u}{x}\pdv{u}{y} + \pdv{v}{x}\pdv{v}{y} = 0.\]
        \begin{proof}
            By the C-R equations, \[\pdv{u}{x}\pdv{u}{y} + \pdv{v}{x}\pdv{v}{y} = \br{\pdv{v}{y}}\br{-\pdv{v}{x}} + \pdv{v}{x}\pdv{v}{y} = 0.\]
        \end{proof}
    \end{enumerate}
    \item Find if $f(z) = \Re(z) = x$ is analytic.
    
    The function $f$ is not analytic.
    \begin{proof}
        Write $f(z) = u(x,y) + iv(x,y)$ with $u(x,y) = x$ and $v(x,y) = 0$. The partial derivatives \[\pdv{u}{x} = 1, \pdv{u}{y} = 0, \pdv{v}{x} = 0, \pdv{v}{y} = 0\] exist and are continuous for all $z\in \mathbb{C}$. But $\pdv{u}{x} = 1\neq 0 = \pdv{v}{y}$, so the Cauchy-Riemann equations do not hold. Hence $f(z) = \Re(z) = x$ is not differentiable, and hence is not analytic.
    \end{proof}
    \item Evaluate $\oint_C \frac{\dd{z}}{z^2-1}$ where $C$ is the circle $\abs{z} = 2$ (traversed only once around).
    
    Let $C$ be the circle $\abs{z} = 2$ traversed only once around, and let $C_1$ be the circle $\abs{z-1} = \frac{1}{2}$ and $C_{-1}$ be the circle $\abs{z+1} = \frac{1}{2}$ both traversed only once around. Then due to contour surgery and Cauchy's integral theorem we have
    \begin{align*}
        \oint_C \frac{\dd{z}}{z^2-1} &= \oint_{C_1} \frac{1/2}{z-1}\dd{z} - \oint_{C_{-1}} \frac{1/2}{z+1}\dd{z}\\
        &= 2\pi i (1/2) - 2\pi i (1/2)\\
        &= 0.
    \end{align*}
\end{enumerate}
\end{document}