\documentclass[11pt]{article}
\headheight = 14pt

% packages
\usepackage{physics}
% margin spacing
\usepackage[top=1in, bottom=1in, left=0.5in, right=0.5in]{geometry}
\usepackage{hanging}
\usepackage{amsfonts, amsmath, amssymb, amsthm}
\usepackage{systeme}
\usepackage[none]{hyphenat}
\usepackage{fancyhdr}
\usepackage[nottoc, notlot, notlof]{tocbibind}
\usepackage{graphicx}
\graphicspath{{./images/}}
\usepackage{float}
\usepackage{siunitx}
\usepackage{esint}
\usepackage{cancel}
\usepackage{enumitem}

% colors
\usepackage{xcolor}
\definecolor{p}{HTML}{FFDDDD}
\definecolor{g}{HTML}{D9FFDF}
\definecolor{y}{HTML}{FFFFCF}
\definecolor{b}{HTML}{D9FFFF}
\definecolor{o}{HTML}{FADECB}
%\definecolor{}{HTML}{}

% \highlight[<color>]{<stuff>}
\newcommand{\highlight}[2][p]{\mathchoice%
  {\colorbox{#1}{$\displaystyle#2$}}%
  {\colorbox{#1}{$\textstyle#2$}}%
  {\colorbox{#1}{$\scriptstyle#2$}}%
  {\colorbox{#1}{$\scriptscriptstyle#2$}}}%

% header/footer formatting
\pagestyle{fancy}
\fancyhead{}
\fancyfoot{}
\fancyhead[L]{PHZ3113}
\fancyhead[C]{HW11}
\fancyhead[R]{Sai Sivakumar}
\fancyfoot[R]{\thepage}
\renewcommand{\headrulewidth}{1pt}

% paragraph indentation/spacing
\setlength{\parindent}{0cm}
\setlength{\parskip}{5pt}
\renewcommand{\baselinestretch}{1.25}

% extra commands defined here
\newcommand{\ihat}{\boldsymbol{\hat{\textbf{\i}}}}
\newcommand{\jhat}{\boldsymbol{\hat{\textbf{\j}}}}
\newcommand{\khat}{\boldsymbol{\hat{\textbf{k}}}}
\newcommand{\dr}{\vec{r}~^{\prime}(t)}
\newcommand{\dx}{x^{\prime}(t)}
\newcommand{\dy}{y^{\prime}(t)}

\newcommand{\br}[1]{\left(#1\right)}
\newcommand{\sbr}[1]{\left[#1\right]}
\newcommand{\cbr}[1]{\left\{#1\right\}}

\newcommand{\dprime}{\prime\prime}
\newcommand{\lap}[2]{\mathcal{L}[#1](#2)}

% bracket notation for inner product
\usepackage{mathtools}

\DeclarePairedDelimiterX{\abr}[1]{\langle}{\rangle}{#1}

\DeclareMathOperator{\Span}{span}
\DeclareMathOperator\Arg{Arg}
\DeclareMathOperator\Log{Log}


% set page count index to begin from 1
\setcounter{page}{1}

\begin{document}
11.1 The relativistic Lagrangian of a particle of mass $m$ in an electromagnetic field is given by \[\mathcal{L} = -mc^2\sqrt{1-v^2/c^2} + \frac{e}{c}\vec{A}\cdot\vec{v}-e\phi;\quad \vec{v} = \dot{\vec{r}}.\]
\begin{enumerate}[label=(\alph*)]
    \item Calculate $\partial \mathcal{L}/\partial \vec{r} = \vec{\nabla}\mathcal{L}$, keeping $\vec{v}$ constant. Then
    \begin{align*}
        \pdv{\mathcal{L}}{\vec{r}} = \vec{\nabla}\mathcal{L} = \frac{e}{c}\vec{\nabla}\br{\vec{A}\cdot\vec{v}} - e\vec{\nabla}\phi &= \frac{e}{c}\br{ (\vec{v}\cdot\vec{\nabla})\vec{A} + (\vec{A}\cdot\vec{\nabla})\vec{v} + \vec{A}\times (\vec{\nabla}\times \vec{v}) + \vec{v}\times (\vec{\nabla}\times \vec{A}) } - e\vec{\nabla}\phi \\
        &= \frac{e}{c}\br{ (\vec{v}\cdot\vec{\nabla})\vec{A} + \vec{v}\times (\vec{\nabla}\times \vec{A})} - e\vec{\nabla}\phi\\
        &= \frac{e}{c} (\vec{v}\cdot\vec{\nabla})\vec{A} + \frac{e}{c}\vec{v}\times (\vec{\nabla}\times \vec{A})  - e\vec{\nabla}\phi,
    \end{align*} where the first term in the Lagrangian was constant with respect to position (so the spatial derivative vanished), and all spatial derivatives of $\vec{v}$ vanished.
    
    \item Define the generalized momentum $\vec{P} = \pdv{\mathcal{L}}{\vec{v}}$ and show that $\vec{P}$ can be written as $\vec{P} = \vec{p} + \frac{e}{c}\vec{A}$ where $\vec{p} = \gamma m\vec{v}$ is the relativistic free particle momentum as obtained in HW 4.2.
    
    We have
    \begin{align*}
        \vec{P} &= \pdv{\vec{v}} \br{-mc^2\sqrt{1-(\vec{v}\cdot \vec{v})/c^2} + \frac{e}{c}\vec{A}\cdot\vec{v}-e\phi} \\
        &= \frac{-mc^2\cdot -2\pdv{\vec{v}}\vec{v}\cdot \vec{v}/c^2 }{2\sqrt{1-(\vec{v}\cdot \vec{v})/c^2}} + \frac{e}{c}\vec{A}\\
        &= \frac{m\vec{v}}{\sqrt{1-v^2/c^2}} + \frac{e}{c}\vec{A} = \gamma m\vec{v} + \frac{e}{c}\vec{A} = \vec{p} + \frac{e}{c}\vec{A}.
    \end{align*}
    
    \item Use the previous result as well as equations (7.12) to show that the E-L equation gives \[\dv{\vec{p}}{t} = e\vec{E} + \frac{e}{c}(\vec{v}\times \vec{B}).\]
    
    So by the Euler-Lagrange equations, we have $\dv{t}\pdv{\mathcal{L}}{\dot{\vec{r}}} = \pdv{\mathcal{L}}{\vec{r}}$. But $\dv{t}\pdv{\mathcal{L}}{\dot{\vec{r}}} = \dv{t}\pdv{\mathcal{L}}{\vec{v}} = \dv{t} \br{\vec{p} + \frac{e}{c}\vec{A}} = \dv{\vec{p}}{t} + \frac{e}{c}\dv{\vec{A}}{t}$. Then 
    \begin{align*}
        \dv{\vec{p}}{t} = \pdv{\mathcal{L}}{\vec{r}}- \frac{e}{c}\pdv{\vec{A}}{t} &= \frac{e}{c} (\vec{v}\cdot\vec{\nabla})\vec{A} + \frac{e}{c}\vec{v}\times (\vec{\nabla}\times \vec{A})  - e\vec{\nabla}\phi - \frac{e}{c}\dv{\vec{A}}{t}\\
        &= \frac{e}{c} (\vec{v}\cdot\vec{\nabla})\vec{A} + \frac{e}{c}\vec{v}\times \vec{B}  + e\br{\vec{E} + \frac{1}{c}\pdv{\vec{A}}{t}} - \frac{e}{c}\dv{\vec{A}}{t}\\
        &= \frac{e}{c}\br{\br{\pdv{\vec{r}}{t}\cdot\vec{\nabla}}\vec{A} + \pdv{\vec{A}}{t}} + \frac{e}{c}\vec{v}\times\vec{B}+ e\vec{E} - \frac{e}{c}\dv{\vec{A}}{t}\\
        &= \frac{e}{c}\dv{\vec{A}}{t} + \frac{e}{c}\vec{v}\times\vec{B}+ e\vec{E} - \frac{e}{c}\dv{\vec{A}}{t}\\
        &= \frac{e}{c}\vec{v}\times\vec{B} + e\vec{E},
    \end{align*} where we used the vector(?) chain rule.
\end{enumerate}
\end{document}