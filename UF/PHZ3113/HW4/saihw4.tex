\documentclass[11pt]{article}
\headheight = 14pt

% packages
\usepackage{physics}
% margin spacing
\usepackage[top=1in, bottom=1in, left=0.5in, right=0.5in]{geometry}
\usepackage{hanging}
\usepackage{amsfonts, amsmath, amssymb, amsthm}
\usepackage{systeme}
\usepackage[none]{hyphenat}
\usepackage{fancyhdr}
\usepackage[nottoc, notlot, notlof]{tocbibind}
\usepackage{graphicx}
\graphicspath{{./images/}}
\usepackage{float}
\usepackage{siunitx}
\usepackage{esint}
\usepackage{cancel}

% colors
\usepackage{xcolor}
\definecolor{p}{HTML}{FFDDDD}
\definecolor{g}{HTML}{D9FFDF}
\definecolor{y}{HTML}{FFFFCF}
\definecolor{b}{HTML}{D9FFFF}
\definecolor{o}{HTML}{FADECB}
%\definecolor{}{HTML}{}

% \highlight[<color>]{<stuff>}
\newcommand{\highlight}[2][p]{\mathchoice%
  {\colorbox{#1}{$\displaystyle#2$}}%
  {\colorbox{#1}{$\textstyle#2$}}%
  {\colorbox{#1}{$\scriptstyle#2$}}%
  {\colorbox{#1}{$\scriptscriptstyle#2$}}}%

% header/footer formatting
\pagestyle{fancy}
\fancyhead{}
\fancyfoot{}
\fancyhead[L]{PHZ3113}
\fancyhead[C]{HW4}
\fancyhead[R]{Sai Sivakumar}
\fancyfoot[R]{\thepage}
\renewcommand{\headrulewidth}{1pt}

% paragraph indentation/spacing
\setlength{\parindent}{0cm}
\setlength{\parskip}{5pt}
\renewcommand{\baselinestretch}{1.25}

% extra commands defined here
\newcommand{\ihat}{\boldsymbol{\hat{\textbf{\i}}}}
\newcommand{\jhat}{\boldsymbol{\hat{\textbf{\j}}}}
\newcommand{\dr}{\vec{r}~^{\prime}(t)}
\newcommand{\dx}{x^{\prime}(t)}
\newcommand{\dy}{y^{\prime}(t)}

\newcommand{\br}[1]{\left(#1\right)}
\newcommand{\sbr}[1]{\left[#1\right]}
\newcommand{\cbr}[1]{\left\{#1\right\}}

\newcommand{\dprime}{\prime\prime}
\newcommand{\lap}[2]{\mathcal{L}[#1](#2)}

% bracket notation for inner product
\usepackage{mathtools}

\DeclarePairedDelimiterX{\abr}[1]{\langle}{\rangle}{#1}

\DeclareMathOperator{\Span}{span}
\DeclareMathOperator{\nullity}{nullity}
\DeclareMathOperator\Arg{Arg}
\DeclareMathOperator\Log{Log}


% set page count index to begin from 1
\setcounter{page}{1}

\begin{document}

\begin{enumerate}
    \item (4.1) The central force problem.
    
    (a) Generalized momenta conjugate to generalized coordinates $r$ and $\phi$, where $\mathcal{L} = \frac{1}{2}m\left(\dot{r}^2+r^2\dot{\phi}^2\right) - V(r)$ are
    \begin{align*}
        p_r &= \pdv{\mathcal{L}}{\dot{r}} = m\dot{r}\\
        p_{\phi} &= \pdv{\mathcal{L}}{\dot{\phi}} = mr^2\dot{\phi}.
    \end{align*}
    (b) Apply the Legendre transformation to find \[\mathcal{H} = \dot{r}\br{m\dot{r}} + \dot{\phi}\br{mr^2\dot{\phi}} - \frac{1}{2}m\left(\dot{r}^2+r^2\dot{\phi}^2\right) + V(r) = \frac{1}{2}m\dot{r}^2 + \frac{1}{2}mr^2\dot{\phi}^2 + V(r) = \frac{p_r^2}{2m}+ \frac{p_{\phi}^2}{2mr^2} + V(r),\] which by inspection is equivalent to the total energy. The quantity $\frac{1}{2}m\dot{r}^2 + \frac{1}{2}mr^2\dot{\phi}^2$ is the sum of the ``radial'' kinetic energy and the angular kinetic energy, and $V(r)$ is the potential energy.

    (c) Then from Hamilton's equations, \begin{align*}
        \dot{r} &= \pdv{\mathcal{H}}{p_r} = \frac{p_r}{m} \quad (\text{no information})\\
        -m\ddot{r} = -\dot{p_r} &= \pdv{\mathcal{H}}{r} = mr\dot{\phi}^2 + \pdv{V(r)}{r} \quad (m\ddot{r} + mr\dot{\phi}^2 = f(r))\\
        \dot{\phi} &= \pdv{\mathcal{H}}{p_{\phi}} = \frac{p_{\phi}}{mr^2} \quad (\text{no information})\\
        -\dot{p_{\phi}} &= \pdv{\mathcal{H}}{\phi} = 0 \quad (p_{\phi} = \text{constant}).
    \end{align*}

    I have a sign error for some reason in the second equation, and I am unsure how to fix it. It should be equivalent to the equations obtained via the Lagrangian. The time derivative of $p_{\phi}$ is consistent with the result from the Lagrangian.

    \item (4.2) Free particle.
    
    (a) Let $\mathcal{L} = -mc^2\sqrt{1-\dot{x}^2/c^2}$, where $c$ is the speed of light and define $\gamma = 1/\sqrt{1-\dot{x}^2/c^2}$. Then the momentum $p$ conjugate to $x$ in terms of $\gamma$ is \[p = \pdv{\mathcal{L}}{\dot{x}} = m\dot{x}\br{1-\frac{\dot{x}^2}{c^2}}^{-\frac{1}{2}} = m\dot{x}\gamma.\]
    (b) Then the Hamiltonian is given by \[\mathcal{H} = \dot{x}\br{m\dot{x}\gamma} + mc^2\sqrt{1-\dot{x}^2/c^2} = \sbr{m\dot{x}+mc^2\br{1-\frac{\dot{x}}{c^2}}}\gamma = mc^2\gamma.\]
    (c) With some algebra, see that from the momentum equation that we have $p^2/(m^2c^2+p^2) = \dot{x}^2/c^2$ and so in $mc^2\gamma$, we simplify \[\mathcal{H} = \frac{mc^2}{\sqrt{1-\dot{x}^2/c^2}} = \frac{mc^2}{\sqrt{1-p^2/(m^2c^2+p^2)}} = \frac{mc^2}{\sqrt{m^2c^2/(m^2c^2+p^2)}} = mc^2\sqrt{\frac{p^2}{m^2c^2}+1},\] and this last term is the total energy in terms of $p$ alone (how the Hamiltonian should be written as there is no explicit dependence on $\dot{x}$).
\end{enumerate}
\end{document}