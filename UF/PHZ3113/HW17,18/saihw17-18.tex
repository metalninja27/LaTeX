\documentclass[11pt]{article}
\headheight = 14pt

% packages
\usepackage{physics}
% margin spacing
\usepackage[top=1in, bottom=1in, left=0.5in, right=0.5in]{geometry}
\usepackage{hanging}
\usepackage{amsfonts, amsmath, amssymb, amsthm}
\usepackage{systeme}
\usepackage[none]{hyphenat}
\usepackage{fancyhdr}
\usepackage[nottoc, notlot, notlof]{tocbibind}
\usepackage{graphicx}
\graphicspath{{./images/}}
\usepackage{float}
\usepackage{siunitx}
\usepackage{esint}
\usepackage{cancel}
\usepackage{enumitem}

% colors
\usepackage{xcolor}
\definecolor{p}{HTML}{FFDDDD}
\definecolor{g}{HTML}{D9FFDF}
\definecolor{y}{HTML}{FFFFCF}
\definecolor{b}{HTML}{D9FFFF}
\definecolor{o}{HTML}{FADECB}
%\definecolor{}{HTML}{}

% \highlight[<color>]{<stuff>}
\newcommand{\highlight}[2][p]{\mathchoice%
  {\colorbox{#1}{$\displaystyle#2$}}%
  {\colorbox{#1}{$\textstyle#2$}}%
  {\colorbox{#1}{$\scriptstyle#2$}}%
  {\colorbox{#1}{$\scriptscriptstyle#2$}}}%

% header/footer formatting
\pagestyle{fancy}
\fancyhead{}
\fancyfoot{}
\fancyhead[L]{PHZ3113}
\fancyhead[C]{HW17 \& 18}
\fancyhead[R]{Sai Sivakumar}
\fancyfoot[R]{\thepage}
\renewcommand{\headrulewidth}{1pt}

% paragraph indentation/spacing
\setlength{\parindent}{0cm}
\setlength{\parskip}{5pt}
\renewcommand{\baselinestretch}{1.25}

% extra commands defined here
\newcommand{\ihat}{\boldsymbol{\hat{\textbf{\i}}}}
\newcommand{\jhat}{\boldsymbol{\hat{\textbf{\j}}}}
\newcommand{\khat}{\boldsymbol{\hat{\textbf{k}}}}
\newcommand{\dr}{\vec{r}~^{\prime}(t)}
\newcommand{\dx}{x^{\prime}(t)}
\newcommand{\dy}{y^{\prime}(t)}

\newcommand{\br}[1]{\left(#1\right)}
\newcommand{\sbr}[1]{\left[#1\right]}
\newcommand{\cbr}[1]{\left\{#1\right\}}

\newcommand{\dprime}{\prime\prime}
\newcommand{\lap}[2]{\mathcal{L}[#1](#2)}

% bracket notation for inner product
\usepackage{mathtools}

\DeclarePairedDelimiterX{\abr}[1]{\langle}{\rangle}{#1}

\DeclareMathOperator{\Span}{span}
\DeclareMathOperator\Arg{Arg}
\DeclareMathOperator\Log{Log}

% set page count index to begin from 1
\setcounter{page}{1}

\begin{document}
HW17:

The normal modes $\omega_1 = \sqrt{k/m}$ and $\omega_2 =  \sqrt{(k+2k^{\prime})/m}$ with the corresponding normalized eigenvectors are \[\ket{\omega_1} = \frac{1}{\sqrt{2}}\begin{pmatrix}
    1 \\ 1
\end{pmatrix}\quad \text{and}\quad \ket{\omega_2} = \frac{1}{\sqrt{2}}\begin{pmatrix}
    1 \\ -1
\end{pmatrix}.\]
\begin{enumerate}[label=(\alph*)]
    \item With $t = 0$, we can write \[\begin{pmatrix}
        x_1(0) \\ x_2(0)
    \end{pmatrix} = \begin{pmatrix}
        1 \\ 0
    \end{pmatrix} = \frac{1}{\sqrt{2}}\ket{\omega_1} + \frac{1}{\sqrt{2}}\ket{\omega_2}\]
    so that $\ket{x(t)} = \frac{1}{\sqrt{2}}[\ket{\omega_1}\cos(\omega_1 t) + \ket{\omega_2}\cos(\omega_2 t)]$. Simplification yields \[\ket{x(t)} = \frac{1}{2}\begin{pmatrix}
        \cos(\omega_1 t) + \cos(\omega_2 t) \\ \cos(\omega_1 t) - \cos(\omega_2 t)
    \end{pmatrix},\] and of course taking $t\to 0$ we find that the first component tends to $1$ and the second component tends to $0$, which matches with $\ket{x(0)}$.
    \item The propogator $U(t)$ is found by taking $\ket{\omega_1}\bra{\omega_1}\cos(\omega_1 t) + \ket{\omega_2}\bra{\omega_2}\cos(\omega_2 t)$. Thus
    \begin{multline*}U(t) = \frac{1}{\sqrt{2}}\begin{pmatrix}
        1 \\ 1
    \end{pmatrix}\frac{1}{\sqrt{2}}\begin{pmatrix}
        1 & 1
    \end{pmatrix}\cos(\omega_1 t) + \frac{1}{\sqrt{2}}\begin{pmatrix}
        1 \\ -1
    \end{pmatrix}\frac{1}{\sqrt{2}}\begin{pmatrix}
        1 & -1
    \end{pmatrix}\cos(\omega_2 t) \\ = \frac{1}{2}\begin{pmatrix}
        \cos(\omega_1 t) + \cos(\omega_2 t) & \cos(\omega_1 t) - \cos(\omega_2 t) \\
        \cos(\omega_1 t) - \cos(\omega_2 t) & \cos(\omega_1 t) + \cos(\omega_2 t)
    \end{pmatrix}.\end{multline*}
    \item Indeed $\ket{x(t)} = U(t)\ket{x(0)}$ (the left column is $\ket{x(t)}$):
    \[U(t)\ket{x(0)} = \frac{1}{2}\begin{pmatrix}
        \cos(\omega_1 t) + \cos(\omega_2 t) & \cos(\omega_1 t) - \cos(\omega_2 t) \\
        \cos(\omega_1 t) - \cos(\omega_2 t) & \cos(\omega_1 t) + \cos(\omega_2 t)
    \end{pmatrix}\begin{pmatrix}
        1 \\ 0
    \end{pmatrix} = \frac{1}{2}\begin{pmatrix}
        \cos(\omega_1 t) + \cos(\omega_2 t) \\ \cos(\omega_1 t) - \cos(\omega_2 t)
    \end{pmatrix} = \ket{x(t)}.\]
\end{enumerate}
HW18:

\begin{enumerate}[label=18.\arabic*]
    \item Prove that $\cos(\theta + \phi) = \cos(\theta)\cos(\phi) - \sin(\theta)\sin(\phi)$ and $\sin(\theta + \phi) = \cos(\theta)\sin(\phi) + \sin(\theta)\cos(\phi)$.
    \begin{proof}
        Consider $\exp(i\theta)\exp(i\phi) = \exp(i(\theta + \phi))$, which has real part $\cos(\theta + \phi)$ and has imaginary part $\sin(\theta + \phi)$. Then 
        \begin{align*}
            \exp(i\theta)\exp(i\phi) &= [\cos(\theta) + i\sin(\theta)][\cos(\phi) + i\sin(\phi)] \\
            &= [\cos(\theta)\cos(\phi) - \sin(\theta)\sin(\phi)] + i[\cos(\theta)\sin(\phi) + \sin(\theta)\cos(\phi)],
        \end{align*}
        but since complex numbers are equal if and only if their components are equal, we have that $\cos(\theta + \phi) = \cos(\theta)\cos(\phi) - \sin(\theta)\sin(\phi)$ and $\sin(\theta + \phi) = \cos(\theta)\sin(\phi) + \sin(\theta)\cos(\phi)$.
    \end{proof}
    \item (a) We have with algebra and trigonometry that \[z = \frac{3+4i}{3-4i} = \frac{(3+4i)^2}{5} = \frac{1}{5}\left(\sqrt{5}\exp(i\arctan(\frac{4}{3}))\right)^2 = \exp(2i\arctan(\frac{4}{3})),\]
    from which $z = \cos(2\arctan(\frac{4}{3})) + i\sin(2\arctan(\frac{4}{3}))$, and $z^{\ast} = \exp(-2i\arctan(\frac{4}{3})) = \cos(2\arctan(\frac{4}{3})) - i\sin(2\arctan(\frac{4}{3}))$. Also $\abs{z}$ we can extract as the coefficient of $\exp(2i\arctan(\frac{4}{3}))$ since the exponential here we interpret as a rotated unit vector. Hence $\abs{z} = 1$.

    (b) We have $z_1 = 2\exp(i\pi/4) = 2(\frac{\sqrt{2}}{2} + i\frac{\sqrt{2}}{2}) = \sqrt{2}(1 + i)$ and $z_2 = 6\exp(i\pi/3) = 6(\frac{1}{2} + i\frac{\sqrt{3}}{2}) = 3(1 + i\sqrt{3})$.
    Then $z_1 + z_2 = (\sqrt{2} + 3) + i(\sqrt{2} + 3\sqrt{3})$.
    \item When $n\mapsto (n-i\alpha)$, we have \[\exp[i\omega(t-(n-i\alpha))x/c] = \exp[i\omega(t-nx/c) - \omega\alpha x/c] = \exp(- \omega\alpha x/c)\exp[i\omega(t-nx/c)].\]
    In effect the amplitude of the wave is reduced (when $\alpha > 0$; otherwise scaled up or unchanged).
\end{enumerate}
\end{document}