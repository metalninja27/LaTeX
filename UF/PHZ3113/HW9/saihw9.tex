\documentclass[11pt]{article}
\headheight = 14pt

% packages
\usepackage{physics}
% margin spacing
\usepackage[top=1in, bottom=1in, left=0.5in, right=0.5in]{geometry}
\usepackage{hanging}
\usepackage{amsfonts, amsmath, amssymb, amsthm}
\usepackage{systeme}
\usepackage[none]{hyphenat}
\usepackage{fancyhdr}
\usepackage[nottoc, notlot, notlof]{tocbibind}
\usepackage{graphicx}
\graphicspath{{./images/}}
\usepackage{float}
\usepackage{siunitx}
\usepackage{esint}
\usepackage{cancel}
\usepackage{enumitem}

% colors
\usepackage{xcolor}
\definecolor{p}{HTML}{FFDDDD}
\definecolor{g}{HTML}{D9FFDF}
\definecolor{y}{HTML}{FFFFCF}
\definecolor{b}{HTML}{D9FFFF}
\definecolor{o}{HTML}{FADECB}
%\definecolor{}{HTML}{}

% \highlight[<color>]{<stuff>}
\newcommand{\highlight}[2][p]{\mathchoice%
  {\colorbox{#1}{$\displaystyle#2$}}%
  {\colorbox{#1}{$\textstyle#2$}}%
  {\colorbox{#1}{$\scriptstyle#2$}}%
  {\colorbox{#1}{$\scriptscriptstyle#2$}}}%

% header/footer formatting
\pagestyle{fancy}
\fancyhead{}
\fancyfoot{}
\fancyhead[L]{PHZ3113}
\fancyhead[C]{HW9}
\fancyhead[R]{Sai Sivakumar}
\fancyfoot[R]{\thepage}
\renewcommand{\headrulewidth}{1pt}

% paragraph indentation/spacing
\setlength{\parindent}{0cm}
\setlength{\parskip}{5pt}
\renewcommand{\baselinestretch}{1.25}

% extra commands defined here
\newcommand{\ihat}{\boldsymbol{\hat{\textbf{\i}}}}
\newcommand{\jhat}{\boldsymbol{\hat{\textbf{\j}}}}
\newcommand{\khat}{\boldsymbol{\hat{\textbf{k}}}}
\newcommand{\dr}{\vec{r}~^{\prime}(t)}
\newcommand{\dx}{x^{\prime}(t)}
\newcommand{\dy}{y^{\prime}(t)}

\newcommand{\br}[1]{\left(#1\right)}
\newcommand{\sbr}[1]{\left[#1\right]}
\newcommand{\cbr}[1]{\left\{#1\right\}}

\newcommand{\dprime}{\prime\prime}
\newcommand{\lap}[2]{\mathcal{L}[#1](#2)}

% bracket notation for inner product
\usepackage{mathtools}

\DeclarePairedDelimiterX{\abr}[1]{\langle}{\rangle}{#1}

\DeclareMathOperator{\Span}{span}
\DeclareMathOperator\Arg{Arg}
\DeclareMathOperator\Log{Log}


% set page count index to begin from 1
\setcounter{page}{1}

\begin{document}
\begin{enumerate}%[label=(\alph*)]
    \item Consider the integral discussed in the lecture \[I = \int_C \vec{F}(\vec{r})\cdot \dd{\vec{r}}, \quad \vec{F} = \ihat 2xy^2 + \jhat x^2.\]
    \begin{enumerate}[label=(\alph*)]
        \item Evaluate the integral along path $C=A$. \begin{align*}
            I = \int_A \vec{F}(\vec{r})\cdot \dd{\vec{r}} &= \int_{x_1}^{x_2} 2ty_1^2\dd{t} + \int_{y_1}^{y_2} x_2^2\dd{t}\\
            &= \boxed{y_1^2(x_2^2-x_1^2) + x_2^2(y_2-y_1)}.
        \end{align*}
        \item Evaluate the integral along path $C=B$. \begin{align*}
            I = \int_B \vec{F}(\vec{r})\cdot \dd{\vec{r}} &= \int_{y_1}^{y_2}x_1^2\dd{t} + \int_{x_1}^{x_2}2ty_2^2\dd{t}\\
            &= \boxed{x_1^2(y_2-y_1) + y_2^2(x_2^2-x_1^2)}.
        \end{align*}
        These integrals are \textbf{not} equivalent to each other.
    \end{enumerate}
    \item In the presence of uniform charge density $\rho$ and current density $\vec{j}$, Maxwell's equations are changed to \[\vec{\nabla}\cdot \vec{E} = \frac{\rho}{\epsilon_0}, \quad \vec{\nabla}\times \vec{B} - \frac{1}{c^2}\pdv{\vec{E}}{t} = \mu_0 \vec{j},\] with $\mu_0\epsilon_0 = 1/c^2$. The total energy stored in the medium in a volume $V$ is $\mathcal{E} = \int_V \frac{1}{2}(\epsilon_0E^2 + B^2/\mu_0)\dd{V}$. Show that the rate of change $\dv{\mathcal{E}}{t}$ is given by \[-\int_V \vec{j}\cdot \vec{E}\dd{V} - \frac{1}{\mu_0}\oint_S (\vec{E}\times \vec{B})\cdot \dd{\vec{S}},\] where $S$ is the boundary of $V$.
    \begin{proof}
        Because we assume both the electric field and magnetic field are sufficiently differentiable, it is safe to interchange the differentiation operator with taking the volume integral. Then \begin{align*}
            \pdv{t}\mathcal{E} = \pdv{t}\int_V \frac{1}{2}(\epsilon_0E^2 + B^2/\mu_0)\dd{V} &= \int_V \frac{1}{2}\pdv{t}(\epsilon_0E^2 + B^2/\mu_0)\dd{V}\\
            &= \frac{1}{2\mu_0}\int_V \frac{1}{c^2}\pdv{t}(\vec{E}\cdot\vec{E})\dd{V} + \frac{1}{2\mu_0}\int_V \pdv{t}(\vec{B}\cdot\vec{B})\dd{V} \\
            &= \frac{1}{2\mu_0}\int_V \frac{1}{c^2}\left(2\vec{E}\cdot\pdv{\vec{E}}{t}\right)\dd{V} + \frac{1}{2\mu_0}\int_V \left(2\vec{B}\cdot\pdv{\vec{B}}{t}\right)\dd{V} \\
            &= \frac{1}{\mu_0}\int_V \left(\vec{E}\cdot\left(\vec{\nabla}\times \vec{B}-\mu_0\vec{j}\right)\right)\dd{V} - \frac{1}{\mu_0}\int_V \left(\vec{B}\cdot\left(\vec{\nabla}\times \vec{E}\right)\right)\dd{V} \\
            &= -\int_V \vec{j}\cdot \vec{E}\dd{V} + \frac{1}{\mu_0}\int_V \left(\vec{E}\cdot \left(\vec{\nabla}\times \vec{B}\right) - \vec{B}\cdot \left(\vec{\nabla}\times \vec{E}\right)\right)\dd{V}\\
            &= -\int_V \vec{j}\cdot \vec{E}\dd{V} - \frac{1}{\mu_0}\int_V \vec{\nabla}\cdot\left(\vec{E}\times\vec{B}\right)\dd{V} \\
            &= -\int_V \vec{j}\cdot \vec{E}\dd{V} - \frac{1}{\mu_0}\oint_S (\vec{E}\times \vec{B})\cdot \dd{\vec{S}},
        \end{align*} which is what we wanted to prove.
    \end{proof}
\end{enumerate}
\end{document}