\documentclass[11pt]{article}
\headheight = 14pt

% packages
\usepackage{physics}
% margin spacing
\usepackage[top=1in, bottom=1in, left=0.5in, right=0.5in]{geometry}
\usepackage{hanging}
\usepackage{amsfonts, amsmath, amssymb, amsthm}
\usepackage{systeme}
\usepackage[none]{hyphenat}
\usepackage{fancyhdr}
\usepackage[nottoc, notlot, notlof]{tocbibind}
\usepackage{graphicx}
\graphicspath{{./images/}}
\usepackage{float}
\usepackage{siunitx}
\usepackage{esint}
\usepackage{cancel}
\usepackage{enumitem}

% colors
\usepackage{xcolor}
\definecolor{p}{HTML}{FFDDDD}
\definecolor{g}{HTML}{D9FFDF}
\definecolor{y}{HTML}{FFFFCF}
\definecolor{b}{HTML}{D9FFFF}
\definecolor{o}{HTML}{FADECB}
%\definecolor{}{HTML}{}

% \highlight[<color>]{<stuff>}
\newcommand{\highlight}[2][p]{\mathchoice%
  {\colorbox{#1}{$\displaystyle#2$}}%
  {\colorbox{#1}{$\textstyle#2$}}%
  {\colorbox{#1}{$\scriptstyle#2$}}%
  {\colorbox{#1}{$\scriptscriptstyle#2$}}}%

% header/footer formatting
\pagestyle{fancy}
\fancyhead{}
\fancyfoot{}
\fancyhead[L]{PHZ3113}
\fancyhead[C]{HW19}
\fancyhead[R]{Sai Sivakumar}
\fancyfoot[R]{\thepage}
\renewcommand{\headrulewidth}{1pt}

% paragraph indentation/spacing
\setlength{\parindent}{0cm}
\setlength{\parskip}{5pt}
\renewcommand{\baselinestretch}{1.25}

% extra commands defined here
\newcommand{\ihat}{\boldsymbol{\hat{\textbf{\i}}}}
\newcommand{\jhat}{\boldsymbol{\hat{\textbf{\j}}}}
\newcommand{\khat}{\boldsymbol{\hat{\textbf{k}}}}
\newcommand{\dr}{\vec{r}~^{\prime}(t)}
\newcommand{\dx}{x^{\prime}(t)}
\newcommand{\dy}{y^{\prime}(t)}

\newcommand{\br}[1]{\left(#1\right)}
\newcommand{\sbr}[1]{\left[#1\right]}
\newcommand{\cbr}[1]{\left\{#1\right\}}

\newcommand{\dprime}{\prime\prime}
\newcommand{\lap}[2]{\mathcal{L}[#1](#2)}

% bracket notation for inner product
\usepackage{mathtools}

\DeclarePairedDelimiterX{\abr}[1]{\langle}{\rangle}{#1}

\DeclareMathOperator{\Span}{span}
\DeclareMathOperator\Arg{Arg}
\DeclareMathOperator\Log{Log}

% set page count index to begin from 1
\setcounter{page}{1}

\begin{document}
19.1 Consider an $LCR$ circuit with resistance $R$, capacitance $C$, and inductance $L$ and driven by a time-dependent voltage $V(t) = V_0\cos(\omega t)$. The resulting time-dependent charge $Q(t)$ in the circuit obeys the differential equation \[L\ddot{Q} + R\dot{Q} + \frac{1}{C}Q = V_0\cos(\omega t).\]
\begin{enumerate}[label=(\alph*)]
    \item With ansatz $Q(t) = Q_0e^{i\omega t}$ and `complex voltage' $V(t) = V_0e^{i\omega t}$, we have \begin{align*}
        L\ddot{Q} + R\dot{Q} + \frac{1}{C}Q = V_0\cos(\omega t) &\iff -LQ_0\omega^2e^{i\omega t} + RiQ_0\omega e^{i\omega t} + \frac{1}{C}e^{i\omega t} = V_0e^{i\omega t}\\
        &\iff Q_0(-L\omega^2 + Ri + \frac{1}{C}) = V_0,
        \intertext{so that} 
        Q_0 &= \frac{V_0}{(-L\omega^2 + Ri\omega  + \frac{1}{C})}.
    \end{align*}
    \item Computing $I(t) = \dot{Q}(t)$, we have \begin{align*}
        I(t) = \dot{Q}(t) &= i\omega Q_0 e^{i\omega t}\\
        &= \frac{i\omega V_0e^{i\omega t}}{(-L\omega^2 + Ri\omega  + \frac{1}{C})}\\
        &= \frac{V(t)}{\frac{-1}{\omega}(-Li\omega^2 - R\omega + \frac{i}{C})}\\
        &= \frac{V(t)}{Z}
    \end{align*} with $Z = \frac{-1}{\omega}(-Li\omega^2 - R\omega + \frac{i}{C})$.

    Then by inspection $\Re(Z) = R$ and $\Im(Z) = L\omega - \frac{1}{C\omega}$ Furthermore, \[\abs{Z} = \sqrt{R^2 + \left(L\omega -\frac{1}{C\omega}\right)^2}.\]
    \item It is easier to express the denominator of the quantity defining $Q_0$ in exponential form before taking the reciprocal. Note that $V_0$ is a real value, and assume it is nonnegative.
    
    So 
    \[\frac{1}{\rho} = \abs{-L\omega^2 + Ri\omega + \frac{1}{C}} = \sqrt{\left(-L\omega^2 +\frac{1}{C}\right)^2 + (R\omega)^2},\] and 
    \[-\phi = \arctan(\frac{R\omega}{-L\omega^2 + \frac{1}{C}}),\] so that 
    \[Q_0 = \rho e^{i\phi} = \frac{V_0}{\sqrt{\left(-L\omega^2 +\frac{1}{C}\right)^2 + (R\omega)^2}}e^{-i\arctan(\frac{R\omega}{-L\omega^2 + \frac{1}{C}})}.\]
    \item When $\phi = -\pi/2$, because $\arctan(\pi/2) $ is undefined (due to division by zero as $\cos(\pi/2) = 0$), we demand \[\frac{R\omega}{-L\omega^2 + \frac{1}{C}}\] to also be undefined by division by zero; that is, $-L\omega^2 + \frac{1}{C}$ must be zero. Thus $\omega = 1/\sqrt{LC}$.
    
    Sketching $\rho^2(\omega)$, we have something which looks like:\vspace*{5cm}

    The width of the resonance is determined by the circuit parameter $R$ (comparing with the solution found in the notes and adapting $\rho^2(\omega)$ to fit that form up to a factor).
\end{enumerate}
\end{document}