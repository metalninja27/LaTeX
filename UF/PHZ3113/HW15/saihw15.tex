\documentclass[11pt]{article}
\headheight = 14pt

% packages
\usepackage{physics}
% margin spacing
\usepackage[top=1in, bottom=1in, left=0.5in, right=0.5in]{geometry}
\usepackage{hanging}
\usepackage{amsfonts, amsmath, amssymb, amsthm}
\usepackage{systeme}
\usepackage[none]{hyphenat}
\usepackage{fancyhdr}
\usepackage[nottoc, notlot, notlof]{tocbibind}
\usepackage{graphicx}
\graphicspath{{./images/}}
\usepackage{float}
\usepackage{siunitx}
\usepackage{esint}
\usepackage{cancel}
\usepackage{enumitem}

% colors
\usepackage{xcolor}
\definecolor{p}{HTML}{FFDDDD}
\definecolor{g}{HTML}{D9FFDF}
\definecolor{y}{HTML}{FFFFCF}
\definecolor{b}{HTML}{D9FFFF}
\definecolor{o}{HTML}{FADECB}
%\definecolor{}{HTML}{}

% \highlight[<color>]{<stuff>}
\newcommand{\highlight}[2][p]{\mathchoice%
  {\colorbox{#1}{$\displaystyle#2$}}%
  {\colorbox{#1}{$\textstyle#2$}}%
  {\colorbox{#1}{$\scriptstyle#2$}}%
  {\colorbox{#1}{$\scriptscriptstyle#2$}}}%

% header/footer formatting
\pagestyle{fancy}
\fancyhead{}
\fancyfoot{}
\fancyhead[L]{PHZ3113}
\fancyhead[C]{HW14}
\fancyhead[R]{Sai Sivakumar}
\fancyfoot[R]{\thepage}
\renewcommand{\headrulewidth}{1pt}

% paragraph indentation/spacing
\setlength{\parindent}{0cm}
\setlength{\parskip}{5pt}
\renewcommand{\baselinestretch}{1.25}

% extra commands defined here
\newcommand{\ihat}{\boldsymbol{\hat{\textbf{\i}}}}
\newcommand{\jhat}{\boldsymbol{\hat{\textbf{\j}}}}
\newcommand{\khat}{\boldsymbol{\hat{\textbf{k}}}}
\newcommand{\dr}{\vec{r}~^{\prime}(t)}
\newcommand{\dx}{x^{\prime}(t)}
\newcommand{\dy}{y^{\prime}(t)}

\newcommand{\br}[1]{\left(#1\right)}
\newcommand{\sbr}[1]{\left[#1\right]}
\newcommand{\cbr}[1]{\left\{#1\right\}}

\newcommand{\dprime}{\prime\prime}
\newcommand{\lap}[2]{\mathcal{L}[#1](#2)}

% bracket notation for inner product
\usepackage{mathtools}

\DeclarePairedDelimiterX{\abr}[1]{\langle}{\rangle}{#1}

\DeclareMathOperator{\Span}{span}
\DeclareMathOperator\Arg{Arg}
\DeclareMathOperator\Log{Log}

% set page count index to begin from 1
\setcounter{page}{1}

\begin{document}
Consider the matrix 
\[A = \begin{pmatrix}
    \alpha & - \alpha & 0 \\
    -\beta & 2\beta & -\beta \\
    0 & -\alpha & \alpha
\end{pmatrix}\]
where $\alpha$ and $\beta$ are real numbers.
\begin{enumerate}[label=(\alph*)]
    \item Find the eigenvalues.
    
    The solutions to the secular equation 
    \begin{multline*}
        \det(A-\lambda I) = \det\begin{pmatrix}
            \alpha-\lambda & - \alpha & 0 \\
        -\beta & 2\beta-\lambda & -\beta \\
        0 & -\alpha & \alpha-\lambda
        \end{pmatrix} = (\alpha-\lambda)\sbr{(2\beta-\lambda)(\alpha-\lambda)-\alpha\beta} - \alpha\beta(\alpha-\lambda) \\
        = (\alpha-\lambda)\sbr{(2\beta-\lambda)(\alpha-\lambda)-2\alpha\beta}
    \end{multline*}
    are $\lambda = \alpha, 0, \alpha+2\beta$, which are the eigenvalues.
    \item Find the corresponding normalized eigenvectors. (By inspection; a guess is enough because eigenvectors can be scaled by nonzero coefficients before normalization. Furthermore, we have three eigenvalues which makes it impossible for any of the matrices below to have rank less than two.)
    
    With $\lambda = \alpha$,
    \[\begin{pmatrix}
        0 & - \alpha & 0 \\
    -\beta & 2\beta-\alpha & -\beta \\
    0 & -\alpha & 0
    \end{pmatrix}\ket{v^{(1)}} = \vec{0} \implies \ket{v^{(1)}} = \begin{pmatrix}
        1 \\ 0 \\ -1
    \end{pmatrix}.\]
    With $\lambda = 0$,
    \[\begin{pmatrix}
        \alpha & - \alpha & 0 \\
        -\beta & 2\beta & -\beta \\
        0 & -\alpha & \alpha
    \end{pmatrix}\ket{v^{(2)}} = \vec{0} \implies \ket{v^{(2)}} = \begin{pmatrix}
        1 \\ 1 \\ 1
    \end{pmatrix}.\]
    With $\lambda = \alpha+2\beta$,
    \[\begin{pmatrix}
        -2\beta & - \alpha & 0 \\
    -\beta & -\alpha & -\beta \\
    0 & -\alpha & -2\beta
    \end{pmatrix}\ket{v^{(3)}} = \vec{0} \implies \ket{v^{(3)}} = \begin{pmatrix}
        1 \\ \frac{-2\beta}{\alpha} \\ 1
    \end{pmatrix}.\]

    Normalizing we have 
        \[\ket{v^{\prime(1)}} = \frac{1}{\sqrt{2}}\begin{pmatrix}
            1 \\ 0 \\ -1
        \end{pmatrix}, \quad
        \ket{v^{\prime(2)}} = \frac{1}{\sqrt{3}}\begin{pmatrix}
            1 \\ 1 \\ 1
        \end{pmatrix}, \quad
        \ket{v^{\prime(3)}} = \frac{1}{\sqrt{2+4\frac{\beta^2}{\alpha^2}}}\begin{pmatrix}
            1 \\ \frac{-2\beta}{\alpha} \\ 1
        \end{pmatrix}.\]
    \item The matrix becomes symmetric if $\alpha = \beta$. Show that in this case the eigenvectors are orthogonal to each other.
    
    With $\alpha = \beta$,
    \[\ket{v^{\prime(3)}} = \frac{1}{\sqrt{6}}\begin{pmatrix}
        1 \\ -2 \\ 1
    \end{pmatrix}.\]
    We compute three dot products which all evaluate to zero:
    \begin{align*}
        \bra{v^{\prime(1)}}\ket{v^{\prime(2)}} &= \frac{1}{\sqrt{6}}(1 + 0 - 1) = 0\\
        \bra{v^{\prime(1)}}\ket{v^{\prime(3)}} &= \frac{1}{2\sqrt{3}}(1 + 0 - 1) = 0\\
        \bra{v^{\prime(2)}}\ket{v^{\prime(3)}} &= \frac{1}{3\sqrt{2}}(1 - 2 + 1) = 0
    \end{align*}
    Therefore the eigenvectors are orthogonal to each other.
\end{enumerate}
\end{document}