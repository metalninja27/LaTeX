\documentclass[11pt]{article}
\headheight = 14pt

% packages
\usepackage{physics}
% margin spacing
\usepackage[top=1in, bottom=1in, left=0.5in, right=0.5in]{geometry}
\usepackage{hanging}
\usepackage{amsfonts, amsmath, amssymb, amsthm}
\usepackage{systeme}
\usepackage[none]{hyphenat}
\usepackage{fancyhdr}
\usepackage[nottoc, notlot, notlof]{tocbibind}
\usepackage{graphicx}
\graphicspath{{./images/}}
\usepackage{float}
\usepackage{siunitx}
\usepackage{esint}
\usepackage{cancel}
\usepackage{enumitem}

% colors
\usepackage{xcolor}
\definecolor{p}{HTML}{FFDDDD}
\definecolor{g}{HTML}{D9FFDF}
\definecolor{y}{HTML}{FFFFCF}
\definecolor{b}{HTML}{D9FFFF}
\definecolor{o}{HTML}{FADECB}
%\definecolor{}{HTML}{}

% \highlight[<color>]{<stuff>}
\newcommand{\highlight}[2][p]{\mathchoice%
  {\colorbox{#1}{$\displaystyle#2$}}%
  {\colorbox{#1}{$\textstyle#2$}}%
  {\colorbox{#1}{$\scriptstyle#2$}}%
  {\colorbox{#1}{$\scriptscriptstyle#2$}}}%

% header/footer formatting
\pagestyle{fancy}
\fancyhead{}
\fancyfoot{}
\fancyhead[L]{PHZ3113}
\fancyhead[C]{HW14}
\fancyhead[R]{Sai Sivakumar}
\fancyfoot[R]{\thepage}
\renewcommand{\headrulewidth}{1pt}

% paragraph indentation/spacing
\setlength{\parindent}{0cm}
\setlength{\parskip}{5pt}
\renewcommand{\baselinestretch}{1.25}

% extra commands defined here
\newcommand{\ihat}{\boldsymbol{\hat{\textbf{\i}}}}
\newcommand{\jhat}{\boldsymbol{\hat{\textbf{\j}}}}
\newcommand{\khat}{\boldsymbol{\hat{\textbf{k}}}}
\newcommand{\dr}{\vec{r}~^{\prime}(t)}
\newcommand{\dx}{x^{\prime}(t)}
\newcommand{\dy}{y^{\prime}(t)}

\newcommand{\br}[1]{\left(#1\right)}
\newcommand{\sbr}[1]{\left[#1\right]}
\newcommand{\cbr}[1]{\left\{#1\right\}}

\newcommand{\dprime}{\prime\prime}
\newcommand{\lap}[2]{\mathcal{L}[#1](#2)}

% bracket notation for inner product
\usepackage{mathtools}

\DeclarePairedDelimiterX{\abr}[1]{\langle}{\rangle}{#1}

\DeclareMathOperator{\Span}{span}
\DeclareMathOperator\Arg{Arg}
\DeclareMathOperator\Log{Log}


% set page count index to begin from 1
\setcounter{page}{1}

\begin{document}

\begin{enumerate}[label=14.\arabic*]
    \item Show that \begin{enumerate}[label=(\alph*)]
        \item Product of two orthogonal matrices is orthogonal.
        \begin{proof}
            Let $A,B$ be two orthogonal matrices; that is, $A^TA = I$ and $B^TB = B$. Then $(AB)^T(AB) = B^TA^TAB = B^TIB = B^TB = I$ and the same is true for $BA$ by symmetry.
        \end{proof}
        \item Trace of a matrix remains invariant under a similarity transformation.
        \begin{proof}
            Let $A,S$ be matrices and let $S$ be invertible. Then $\Trace(S^{-1}AS) = \Trace(S^{-1}(AS)) = \Trace((AS)S^{-1}) = \Trace(A(SS^{-1})) = \Trace(A)$.
        \end{proof}
        \item A Hermitian matrix remains Hermitian under unitary transformation.
        \begin{proof}
            Let $H$ be a Hermitian matrix and $U$ be a unitary matrix. Then $U^{-1}HU = U^{\dagger}HU$, so that $(U^{\dagger}HU)^{\dagger} = U^{\dagger}H^{\dagger}(U^{\dagger})^{\dagger} = U^{\dagger}HU$. Hence $H$ under a unitary transformation is still Hermitian.
        \end{proof}
    \end{enumerate}
    \item \begin{enumerate}[label=(\alph*)]
        \item Show that if $\ket{v^{\prime}} = U\ket{v}$ where $\ket{v}$ is complex and $\bra{v}\ket{v} = \bra{v^{\prime}}\ket{v^{\prime}}$, then $U$ must be unitary.
        \begin{proof}
            Let $U$, $\ket{v}$ be as given, with $\ket{v^{\prime}} = U\ket{v}$. Suppose that $\bra{v}\ket{v} = \bra{v^{\prime}}\ket{v^{\prime}}$. Then by definition,
            \begin{align*}
                \bra{v}\ket{v} = v_i^{\ast}v_i &= \bra{v^{\prime}}\ket{v^{\prime}}\\
                &= (U_{ij}v_j)^{\ast}(U_{ik}v_k)\\
                &= U_{ij}^{\ast}U_{ik} v_jv_k\\
                &= U^{\dagger}_{ji}U_{ik} v_jv_k\\
                &= (U^{\dagger}U)_{jk}v_jv_k.
            \end{align*}
            So by enforcing the equality we must have that $(U^{\dagger}U)_{jk} = \delta_{jk}$, that is $U^{\dagger}U = I$. Hence $U$ is unitary (its adjoint is its inverse).
        \end{proof}
        \item Two matrices $U$ and $H$ are related by $U = e^{i\alpha H}$ where $\alpha$ is real and $H$ is independent of $\alpha$. Show that if $H$ is Hermitian, then $U$ is unitary.
        \begin{proof}
            Let $U$, $H$ be given with $U = e^{i\alpha H}$ and $H$ independent of $\alpha$. For any positive integer power, $(H^n)^{\dagger} = (H^{\dagger})^n$, as $H$ commutes with itself. Then with the definition of the exponential function, the fact that the adjoint is a linear transformation, and the above fact, we find that $U^{\dagger} = (\exp(i\alpha H))^{\dagger} = \exp(-i\alpha H^{\dagger})$:
            \[U^{\dagger} = (\exp(i\alpha H))^{\dagger} = \br{\sum_{n=0}^{\infty} \frac{(i\alpha H)^n}{n!}}^{\dagger} = \sum_{n=0}^{\infty} \frac{(-i\alpha H^{\dagger})^n}{n!} = \exp(-i\alpha H^{\dagger})\]
            But because $H$ is Hermitian, $U^{\dagger} = \exp(-i\alpha H)$. Again using the fact that $H$ commutes with itself, $U^{\dagger}U = \exp(i\alpha H)\exp(-i\alpha H) = \exp(0H) = I$. Hence $U$ is unitary.
        \end{proof}
    \end{enumerate}
    \item Consider the matrix $L = \begin{pmatrix}
        0 & -1 \\ 1 & 0
    \end{pmatrix}$.
    \begin{enumerate}[label = (\alph*)]
        \item Evaluate $L^2$.
        \[L^2 = \begin{pmatrix}
            0 & -1 \\ 1 & 0
        \end{pmatrix}\begin{pmatrix}
            0 & -1 \\ 1 & 0
        \end{pmatrix} = \begin{pmatrix}
            -1 & 0 \\ 0 & -1
        \end{pmatrix} = -I.\]
        \item Show that $e^{\theta L}$ can be written as a two-dimensional rotation matrix.
        
        Using the fact that $L^2 = -I$, it follows that \begin{multline*}
            \exp(\theta L) = \sum_{n=0}^{\infty} \frac{\theta^n L^n}{n!} = L\sum_{n = 0}^{\infty}\frac{(-1)^n\theta^{2n+1}}{(2n+1)!} + I\sum_{n=0}^{\infty}\frac{(-1)^n\theta^{2n}}{(2n)!} \\ = \begin{pmatrix}
                0 & -\sin(\theta)\\\sin(\theta) & 0
             \end{pmatrix} + \begin{pmatrix}
                \cos(\theta) & 0\\0 & \cos(\theta)
             \end{pmatrix} = \begin{pmatrix}
                 \cos(\theta) & -\sin(\theta)\\\sin(\theta) &  \cos(\theta)
             \end{pmatrix},
        \end{multline*}
        which is a rotation of vectors through $\theta$ degrees counterclockwise. (This is consistent with $e^{i\theta} \cdot z$ rotating $z$ through $\theta$ degrees counterclockwise in the complex plane; the algebra is the same due to an isomorphism where $i\mapsto L$.)
    \end{enumerate}
\end{enumerate}
\end{document}