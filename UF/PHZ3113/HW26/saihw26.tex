\documentclass[11pt]{article}
\headheight = 14pt

% packages
\usepackage{physics}
% margin spacing
\usepackage[top=1in, bottom=1in, left=0.5in, right=0.5in]{geometry}
\usepackage{hanging}
\usepackage{amsfonts, amsmath, amssymb, amsthm}
\usepackage{systeme}
\usepackage[none]{hyphenat}
\usepackage{fancyhdr}
\usepackage[nottoc, notlot, notlof]{tocbibind}
\usepackage{graphicx}
\graphicspath{{./images/}}
\usepackage{float}
\usepackage{siunitx}
\usepackage{esint}
\usepackage{cancel}
\usepackage{enumitem}

% colors
\usepackage{xcolor}
\definecolor{p}{HTML}{FFDDDD}
\definecolor{g}{HTML}{D9FFDF}
\definecolor{y}{HTML}{FFFFCF}
\definecolor{b}{HTML}{D9FFFF}
\definecolor{o}{HTML}{FADECB}
%\definecolor{}{HTML}{}

% \highlight[<color>]{<stuff>}
\newcommand{\highlight}[2][p]{\mathchoice%
  {\colorbox{#1}{$\displaystyle#2$}}%
  {\colorbox{#1}{$\textstyle#2$}}%
  {\colorbox{#1}{$\scriptstyle#2$}}%
  {\colorbox{#1}{$\scriptscriptstyle#2$}}}%

% header/footer formatting
\pagestyle{fancy}
\fancyhead{}
\fancyfoot{}
\fancyhead[L]{PHZ3113}
\fancyhead[C]{HW26}
\fancyhead[R]{Sai Sivakumar}
\fancyfoot[R]{\thepage}
\renewcommand{\headrulewidth}{1pt}

% paragraph indentation/spacing
\setlength{\parindent}{0cm}
\setlength{\parskip}{5pt}
\renewcommand{\baselinestretch}{1.25}

% extra commands defined here
\newcommand{\ihat}{\boldsymbol{\hat{\textbf{\i}}}}
\newcommand{\jhat}{\boldsymbol{\hat{\textbf{\j}}}}
\newcommand{\khat}{\boldsymbol{\hat{\textbf{k}}}}
\newcommand{\dr}{\vec{r}~^{\prime}(t)}
\newcommand{\dx}{x^{\prime}(t)}
\newcommand{\dy}{y^{\prime}(t)}

\newcommand{\br}[1]{\left(#1\right)}
\newcommand{\sbr}[1]{\left[#1\right]}
\newcommand{\cbr}[1]{\left\{#1\right\}}

\newcommand{\dprime}{\prime\prime}
\newcommand{\lap}[2]{\mathcal{L}[#1](#2)}

% bracket notation for inner product
\usepackage{mathtools}

\DeclarePairedDelimiterX{\abr}[1]{\langle}{\rangle}{#1}

\DeclareMathOperator{\Span}{span}
\DeclareMathOperator\Arg{Arg}
\DeclareMathOperator\Log{Log}

% set page count index to begin from 1
\setcounter{page}{1}

\begin{document}
\begin{enumerate}[label=26.\arabic*]
    \item Consider an operator $A$ in the Shr\"odinger picture which has no explicit time dependence and which does not evolve in time. The vector $\ket{\psi}$ evolves in time according to (26.4). Consider the expectation value $\abr{A} = \bra{\psi}A\ket{\psi}$.
    \begin{enumerate}[label=(\alph*)]
        \item Show that \[\dv{t}\abr{A} = \frac{i}{\hbar}\abr{[H,A]}.\]
        \begin{proof}
            Let $A$ be an operator which has no explicit time dependence and which does not evolve in time, and let $\ket{\psi}$ evolve in time according to (26.4).
            Using the product rule, we have that \begin{align*}
                \dv{t}\abr{A} = \dv{t}(\bra{\psi}A\ket{\psi}) &= \bra{\psi(0)}\dv{t}(U^{\dagger}(t)AU(t))\ket{\psi(0)}\\
                &= \bra{\psi(0)}\br{\pdv{U^{\dagger}}{t}AU + U^{\dagger}A\pdv{U}{t} + \pdv{A}{t}}\ket{\psi(0)}\\
                &= \bra{\psi(0)}\br{\frac{i}{\hbar}HU^{\dagger}AU - \frac{i}{\hbar}U^{\dagger}AHU}\ket{\psi(0)}\\
                &= \frac{i}{\hbar}\bra{\psi}\br{HA-AH}\ket{\psi}\\
                &= \frac{i}{\hbar}\bra{\psi}\sbr{H,A}\ket{\psi} = \frac{i}{\hbar}\abr{[H,A]},
            \end{align*} as desired.
        \end{proof}
        \item Use \[H = \frac{p_x^2}{2m}+ V(x) \quad \text{with $p_x = -i\hbar\pdv{x}$}\] and the result of (a) to show that \[(i)~~ \dv{\abr{p_x}}{t} = -\left\langle\pdv{V}{x}\right\rangle\quad \text{and} \quad (ii)~~ \dv{\abr{x}}{t} = \frac{\abr{p_x}}{m}.\]
        \begin{proof}
            Directly, we have \begin{align*}
                \dv{\abr{p_x}}{t} &= \frac{i}{\hbar}\left\langle\sbr{\frac{p_x^2}{2m}+ V(x),p_x}\right\rangle\\
                &= \frac{i}{\hbar}\left\langle\br{\br{\frac{p_x^3}{2m}+ V(x)p_x }- \br{\frac{p_x^3}{2m}+ p_xV(x) + V(x)p_x} }\right\rangle\\
                &= \left\langle\br{-\frac{i}{\hbar}p_xV(x)}\right\rangle = -\left\langle\pdv{V}{x}\right\rangle\\
            \end{align*} and \begin{align*}
                \dv{\abr{x}}{t} &= \frac{i}{\hbar}\left\langle\sbr{\frac{p_x^2}{2m}+ V(x),x}\right\rangle\\
                &= \frac{i}{\hbar}\left\langle\br{\br{\frac{1}{2m}(p_x(xp_x -i\hbar)) + V(x)x}- \br{x\frac{p_x^2}{2m}+ V(x)x} }\right\rangle\\
                &= \frac{i}{\hbar}\left\langle\br{\br{\frac{1}{2m}( -2i\hbar p_x + xp_x^2 ) + V(x)x}- \br{x\frac{p_x^2}{2m}+ V(x)x} }\right\rangle\\
                &= \frac{\abr{p_x}}{m},
            \end{align*} as desired.
        \end{proof}
    \end{enumerate}
\end{enumerate}
\end{document}