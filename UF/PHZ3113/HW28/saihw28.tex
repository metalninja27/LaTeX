\documentclass[11pt]{article}
\headheight = 14pt

% packages
\usepackage{physics}
% margin spacing
\usepackage[top=1in, bottom=1in, left=0.5in, right=0.5in]{geometry}
\usepackage{hanging}
\usepackage{amsfonts, amsmath, amssymb, amsthm}
\usepackage{systeme}
\usepackage[none]{hyphenat}
\usepackage{fancyhdr}
\usepackage[nottoc, notlot, notlof]{tocbibind}
\usepackage{graphicx}
\graphicspath{{./images/}}
\usepackage{float}
\usepackage{siunitx}
\usepackage{esint}
\usepackage{cancel}
\usepackage{enumitem}

% colors
\usepackage{xcolor}
\definecolor{p}{HTML}{FFDDDD}
\definecolor{g}{HTML}{D9FFDF}
\definecolor{y}{HTML}{FFFFCF}
\definecolor{b}{HTML}{D9FFFF}
\definecolor{o}{HTML}{FADECB}
%\definecolor{}{HTML}{}

% \highlight[<color>]{<stuff>}
\newcommand{\highlight}[2][p]{\mathchoice%
  {\colorbox{#1}{$\displaystyle#2$}}%
  {\colorbox{#1}{$\textstyle#2$}}%
  {\colorbox{#1}{$\scriptstyle#2$}}%
  {\colorbox{#1}{$\scriptscriptstyle#2$}}}%

% header/footer formatting
\pagestyle{fancy}
\fancyhead{}
\fancyfoot{}
\fancyhead[L]{PHZ3113}
\fancyhead[C]{HW28}
\fancyhead[R]{Sai Sivakumar}
\fancyfoot[R]{\thepage}
\renewcommand{\headrulewidth}{1pt}

% paragraph indentation/spacing
\setlength{\parindent}{0cm}
\setlength{\parskip}{5pt}
\renewcommand{\baselinestretch}{1.25}

% extra commands defined here
\newcommand{\ihat}{\boldsymbol{\hat{\textbf{\i}}}}
\newcommand{\jhat}{\boldsymbol{\hat{\textbf{\j}}}}
\newcommand{\khat}{\boldsymbol{\hat{\textbf{k}}}}
\newcommand{\dr}{\vec{r}~^{\prime}(t)}
\newcommand{\dx}{x^{\prime}(t)}
\newcommand{\dy}{y^{\prime}(t)}

\newcommand{\br}[1]{\left(#1\right)}
\newcommand{\sbr}[1]{\left[#1\right]}
\newcommand{\cbr}[1]{\left\{#1\right\}}

\newcommand{\dprime}{\prime\prime}
\newcommand{\lap}[2]{\mathcal{L}[#1](#2)}

% bracket notation for inner product
\usepackage{mathtools}

\DeclarePairedDelimiterX{\abr}[1]{\langle}{\rangle}{#1}

\DeclareMathOperator{\Span}{span}
\DeclareMathOperator\Arg{Arg}
\DeclareMathOperator\Log{Log}

% set page count index to begin from 1
\setcounter{page}{1}

\begin{document}
\begin{enumerate}[label=28.\arabic*]
    \item \begin{enumerate}[label=(\alph*)]
        \item Find the Fourier series of the function $f(x) = x$ in the range $-\pi < x \leq \pi$.
        
        The Fourier coefficients for $f(x) = x$ are computed as \[a_0 = \frac{1}{2\pi} \int_{-\pi}^{\pi} x\dd{x} = 0\] and \[a_n = \frac{1}{2\pi}\int_{-\pi}^{\pi}xe^{-inx} \dd{x} = \frac{-i\sin(\pi n)}{\pi n^2} + \frac{i\cos(\pi n)}{n} = \frac{i(-1)^n}{n}\] so that the Fourier series for $f(x) = x$ on $(-\pi,\pi]$ is given by \begin{align*}
            f(x) = x \sim \sum_{n\neq 0} \frac{i(-1)^n}{n} e^{inx} &= \sum_{n=1}^\infty \frac{i(-1)^n}{n}\br{e^{inx} - e^{-inx}}\\
            &= \sum_{n=1}^\infty \frac{-2\left(-1\right)^{n}}{n}\sin(nx).
        \end{align*}
        \item Use (a) to show that \[1-\frac{1}{3} + \frac{1}{5} - \frac{1}{7} + \cdots = \frac{\pi}{4}.\]
        
        With $x = \pi/2$, we have \[\frac{\pi}{2} = \sum_{n=1}^\infty \frac{-2\left(-1\right)^{n}}{n}\sin(\frac{n\pi}{2})\] so that \[\frac{\pi}{4} = \sum_{n=0}^\infty \frac{(-1)^n}{2n+1} = 1-\frac{1}{3} + \frac{1}{5} - \frac{1}{7} + \cdots\] as desired.
    \end{enumerate}
    \item For the function $f(x) = 1-x$ for $0\leq x \leq 1$, find \begin{enumerate}[label=(\alph*)]
        \item the Fourier sine series:
        
        The function $f$ is extended to be an odd function on $[-1,1]$ so that the Fourier coefficients are given by \[a_n = 2\int_0^1 (1-x)\sin(\pi nx)\dd{x} = \frac{2\br{n \pi - \sin(n \pi)}}{n^2 \pi^2}\] so that the Fourier sine series for the odd extension of $f(x) = 1-x$ is given by \[\sum_{n=1}^{\infty} \frac{2\br{n \pi - \sin(n \pi)}}{n^2 \pi^2}\sin(\pi n x)\]
        \item the Fourier cosine series:
        
        The function $f$ is extended to be an even function on $[-1,1]$ so that the Fourier coefficients are given by \[a_0 = \int_0^1 (1-x)\dd{x} = \frac{1}{2}\] and \[a_n = 2\int_0^1 (1-x)\cos(\pi n x)\dd{x} = \frac{2 (1 - \cos(n \pi))}{n^2 \pi^2}\] so that the Fourier cosine series for the even extension of $f(x) = 1-x$ is given by \[\frac{1}{2} + \sum_{n=1}^{\infty}\frac{2 (1 - \cos(n \pi))}{n^2 \pi^2}\cos(\pi n x)\]
        \item and which of the two series is better for a numerical evaluation:
        
        The cosine series is better since in the odd extension of $f(x) = 1-x$, there is a jump discontinuity at $x=0$, so there would be unusual behaviour of the Fourier series. The Fourier series for a given function tends to converge better for continuous functions, and for this example in particular after plotting the series for some large finite number of terms the behavior around $0$ for the sine series was shady at best due to the series being continuous, despite the odd extension of $f(x) = 1-x$ not being continuous.
    \end{enumerate}
    \item Verify that \[f(\phi_1-\phi_2) \equiv \frac{1}{2\pi}\sum_{m=-\infty}^{\infty} e^{im(\phi_1-\phi_2)}\] a Dirac delta function.
    \begin{proof}
        Directly, by convoluting $f$ with a smooth enough function $g$ (which without loss of generality is defined on the circle $(-\pi,\pi]$) at $x_0$ we have \begin{align*}
            f\ast g(x_0) &= \int_{-\pi}^{\pi} g(x)f(x_0-x)\dd{x}\\
            &= \int_{-\pi}^{\pi} g(x)\br{\frac{1}{2\pi}\sum_{m=-\infty}^{\infty} e^{im(x_0-x)}}\dd{x}\\
            &= \sum_{m=-\infty}^{\infty} e^{imx_0} \br{\frac{1}{2\pi} \int_{-\pi}^{\pi} g(x)e^{-imx}\dd{x}}\\
            &=\sum_{m=-\infty}^{\infty} a_m e^{imx_0},
        \end{align*} where $a_m$ is the $m$-th Fourier coefficient of $g$. Because $g$ was sufficiently smooth, the Fourier series of $g$ converges to $g$ and so the last sum can be observed to be the Fourier series of $g$ evaluated at $x_0$, which is exactly $g(x_0)$ as desired. Hence $f$ is a Dirac delta function.
    \end{proof}
\end{enumerate}
\end{document}