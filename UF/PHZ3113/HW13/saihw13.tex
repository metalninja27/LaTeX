\documentclass[11pt]{article}
\headheight = 14pt

% packages
\usepackage{physics}
% margin spacing
\usepackage[top=1in, bottom=1in, left=0.5in, right=0.5in]{geometry}
\usepackage{hanging}
\usepackage{amsfonts, amsmath, amssymb, amsthm}
\usepackage{systeme}
\usepackage[none]{hyphenat}
\usepackage{fancyhdr}
\usepackage[nottoc, notlot, notlof]{tocbibind}
\usepackage{graphicx}
\graphicspath{{./images/}}
\usepackage{float}
\usepackage{siunitx}
\usepackage{esint}
\usepackage{cancel}
\usepackage{enumitem}

% colors
\usepackage{xcolor}
\definecolor{p}{HTML}{FFDDDD}
\definecolor{g}{HTML}{D9FFDF}
\definecolor{y}{HTML}{FFFFCF}
\definecolor{b}{HTML}{D9FFFF}
\definecolor{o}{HTML}{FADECB}
%\definecolor{}{HTML}{}

% \highlight[<color>]{<stuff>}
\newcommand{\highlight}[2][p]{\mathchoice%
  {\colorbox{#1}{$\displaystyle#2$}}%
  {\colorbox{#1}{$\textstyle#2$}}%
  {\colorbox{#1}{$\scriptstyle#2$}}%
  {\colorbox{#1}{$\scriptscriptstyle#2$}}}%

% header/footer formatting
\pagestyle{fancy}
\fancyhead{}
\fancyfoot{}
\fancyhead[L]{PHZ3113}
\fancyhead[C]{HW13}
\fancyhead[R]{Sai Sivakumar}
\fancyfoot[R]{\thepage}
\renewcommand{\headrulewidth}{1pt}

% paragraph indentation/spacing
\setlength{\parindent}{0cm}
\setlength{\parskip}{5pt}
\renewcommand{\baselinestretch}{1.25}

% extra commands defined here
\newcommand{\ihat}{\boldsymbol{\hat{\textbf{\i}}}}
\newcommand{\jhat}{\boldsymbol{\hat{\textbf{\j}}}}
\newcommand{\khat}{\boldsymbol{\hat{\textbf{k}}}}
\newcommand{\dr}{\vec{r}~^{\prime}(t)}
\newcommand{\dx}{x^{\prime}(t)}
\newcommand{\dy}{y^{\prime}(t)}

\newcommand{\br}[1]{\left(#1\right)}
\newcommand{\sbr}[1]{\left[#1\right]}
\newcommand{\cbr}[1]{\left\{#1\right\}}

\newcommand{\dprime}{\prime\prime}
\newcommand{\lap}[2]{\mathcal{L}[#1](#2)}

% bracket notation for inner product
\usepackage{mathtools}

\DeclarePairedDelimiterX{\abr}[1]{\langle}{\rangle}{#1}

\DeclareMathOperator{\Span}{span}
\DeclareMathOperator\Arg{Arg}
\DeclareMathOperator\Log{Log}

% augmented matrices
\makeatletter
\renewcommand*\env@matrix[1][*\c@MaxMatrixCols c]{%
  \hskip -\arraycolsep
  \let\@ifnextchar\new@ifnextchar
  \array{#1}}
\makeatother

% set page count index to begin from 1
\setcounter{page}{1}

\begin{document}
\begin{enumerate}
    \item Consider two successive two-dimensional rotations given by $r^{\dprime} = R_{\phi^{\prime}}r^{\prime} = R_{\phi^{\prime}}R_{\phi}r$. Show that $R_{\phi^{\prime}}R_{\phi} = R_{\phi^{\prime}+\phi}$.
    Directly, we have \begin{multline*}
        R_{\phi^{\prime}}R_{\phi} =\begin{pmatrix}
            \cos(\phi^{\prime}) & \sin(\phi^{\prime})\\
            -\sin(\phi^{\prime}) & \cos(\phi^{\prime})
        \end{pmatrix}\begin{pmatrix}
            \cos(\phi) & \sin(\phi)\\
            -\sin(\phi) & \cos(\phi)
        \end{pmatrix} \\
        = \begin{pmatrix}
            \cos(\phi^{\prime})\cos(\phi) - \sin(\phi^{\prime})\sin(\phi) & \cos(\phi^{\prime})\sin(\phi) + \sin(\phi^{\prime})\cos(\phi)\\
            -(\cos(\phi^{\prime})\sin(\phi) + \sin(\phi^{\prime})\cos(\phi)) & \cos(\phi^{\prime})\cos(\phi) - \sin(\phi^{\prime})\sin(\phi)
        \end{pmatrix}
    \end{multline*} which by trigonometric angle sum formulas, is equal to \[\begin{pmatrix}
        \cos(\phi^{\prime}+\phi) & \sin(\phi^{\prime}+\phi)\\
        -\sin(\phi^{\prime}+\phi) & \cos(\phi^{\prime}+\phi)
    \end{pmatrix}= R_{\phi^{\prime}+\phi}.\]
    \item Given \[A = \begin{pmatrix}
        3 & 2 & -1 \\
        0 & 3 & 2 \\
        1 & -3 & 4
    \end{pmatrix}, \quad B = \begin{pmatrix}
        2 & -2 & 3 \\
        1 & 1 & 0 \\
        3 & 2 & 1
    \end{pmatrix}\]
    evaluate the commutator $[A,B]\equiv AB-BA$.
    The commutator is \begin{align*}
        [A,B]\equiv AB-BA &= \begin{pmatrix}
            3 & 2 & -1 \\
            0 & 3 & 2 \\
            1 & -3 & 4
        \end{pmatrix}\begin{pmatrix}
            2 & -2 & 3 \\
            1 & 1 & 0 \\
            3 & 2 & 1
        \end{pmatrix} - \begin{pmatrix}
            2 & -2 & 3 \\
            1 & 1 & 0 \\
            3 & 2 & 1
        \end{pmatrix}\begin{pmatrix}
            3 & 2 & -1 \\
            0 & 3 & 2 \\
            1 & -3 & 4
        \end{pmatrix}\\
        &= \begin{pmatrix}
            5 & -6 & 8 \\
            9 & 7 & 2 \\
            11 & 3 & 7
        \end{pmatrix} - \begin{pmatrix}
            9 & -11 & 6 \\
            3 & 5 & 1 \\
            10 & 9 & 5
        \end{pmatrix}\\
        &= \begin{pmatrix}
            -4 & 5 & 2 \\
            6 & 2 & 1 \\
            1 & -6 & 2
        \end{pmatrix}.
    \end{align*}
    \item Write the set of simultaneous equations 
    \[\systeme{2x_1 + 4x_2 + 3x_3 = 4, x_1 - 2x_2 - 2x_3 = 0, -3x_1 + 3x_2 + 2x_3 = -7}\]
    in matrix form and obtain the solutions by inverting the matrix.

    The system in matrix form is \[\begin{pmatrix}
        2 & 4 & 3\\
        1 & -2 & -2\\
        -3 & 3 & 2
    \end{pmatrix}\begin{pmatrix}
        x_1\\x_2\\x_3
    \end{pmatrix} = \begin{pmatrix}
        4\\0\\-7
    \end{pmatrix},\] and by using an augmented matrix we can form the inverse matrix:
    \begin{equation*}
        \begin{pmatrix}[ccc|ccc]
            2 & 4 & 3  &  1 & 0 & 0\\
            1 & -2 & -2  &  0 & 1 & 0\\
            -3 & 3 & 2  &  0 & 0 & 1
        \end{pmatrix}\to \begin{pmatrix}[ccc|ccc]
            1 & 2 & \frac{3}{2}  &  \frac{1}{2} & 0 & 0\\
            0 & 1 & \frac{7}{8}  &  \frac{1}{8} & \frac{-1}{4} & 0\\
            0 & 0 & 1  &  \frac{-3}{11} & \frac{-18}{11} & \frac{-8}{11}
        \end{pmatrix} \to \begin{pmatrix}[ccc|ccc]
            1 & 0 & 0  &  \frac{2}{11} & \frac{1}{11} & \frac{-2}{11}\\
            0 & 1 & 0  &  \frac{4}{11} & \frac{13}{11} & \frac{7}{11}\\
            0 & 0 & 1  &  \frac{-3}{11} & \frac{-18}{11} & \frac{-8}{11}
        \end{pmatrix},
    \end{equation*} so that the solution is \begin{equation*}
        \begin{pmatrix}
            x_1\\x_2\\x_3
        \end{pmatrix} = \frac{1}{11}\begin{pmatrix}
            2 & 1 & -2\\
            4 & 13 & 7\\
            -3 & -18 & -8
        \end{pmatrix}\begin{pmatrix}
            4\\0\\-7
        \end{pmatrix} = \begin{pmatrix}
            2\\-3\\4
        \end{pmatrix}.
    \end{equation*}
\end{enumerate}
\end{document}