\documentclass[11pt]{article}

% packages
\usepackage{physics}
% margin spacing
\usepackage[top=1in, bottom=1in, left=0.5in, right=0.5in]{geometry}
\usepackage{hanging}
\usepackage{amsfonts, amsmath, amssymb, amsthm}
\usepackage{systeme}
\usepackage[none]{hyphenat}
\usepackage{fancyhdr}
\usepackage[nottoc, notlot, notlof]{tocbibind}
\usepackage{graphicx}
\graphicspath{{./images/}}
\usepackage{float}
\usepackage{siunitx}
\usepackage{esint}
\usepackage{cancel}

% permutations (second line is for spacing)
\usepackage{permute}
\renewcommand*\pmtseparator{\,}

% colors
\usepackage{xcolor}
\definecolor{p}{HTML}{FFDDDD}
\definecolor{g}{HTML}{D9FFDF}
\definecolor{y}{HTML}{FFFFCF}
\definecolor{b}{HTML}{D9FFFF}
\definecolor{o}{HTML}{FADECB}
%\definecolor{}{HTML}{}

% \highlight[<color>]{<stuff>}
\newcommand{\highlight}[2][p]{\mathchoice%
  {\colorbox{#1}{$\displaystyle#2$}}%
  {\colorbox{#1}{$\textstyle#2$}}%
  {\colorbox{#1}{$\scriptstyle#2$}}%
  {\colorbox{#1}{$\scriptscriptstyle#2$}}}%

% header/footer formatting
\pagestyle{fancy}
\fancyhead{}
\fancyfoot{}
\fancyhead[L]{MAS4301 Abstract Algebra 1}
\fancyhead[C]{Dr. Garvan}
\fancyhead[R]{Spring 2021}
\fancyfoot[R]{\thepage}
%\renewcommand{\headrulewidth}{0pt}

% paragraph indentation/spacing
\setlength{\parindent}{0cm}
\setlength{\parskip}{10pt}
\renewcommand{\baselinestretch}{1.25}

% extra commands defined here
\newcommand{\ihat}{\boldsymbol{\hat{\textbf{\i}}}}
\newcommand{\jhat}{\boldsymbol{\hat{\textbf{\j}}}}
\newcommand{\dr}{\vec{r}~^{\prime}(t)}
\newcommand{\dx}{x^{\prime}(t)}
\newcommand{\dy}{y^{\prime}(t)}

\newcommand{\br}[1]{\left(#1\right)}
\newcommand{\sbr}[1]{\left[#1\right]}
\newcommand{\cbr}[1]{\left\{#1\right\}}

\newcommand{\dprime}{\prime\prime}
\newcommand{\lap}[2]{\mathcal{L}[#1](#2)}

\newcommand{\divides}{\mid}

% bracket notation for inner product
\usepackage{mathtools}

\DeclarePairedDelimiterX{\abr}[1]{\langle}{\rangle}{#1}

\DeclareMathOperator{\Span}{span}
\DeclareMathOperator{\nullity}{nullity}
\DeclareMathOperator\Aut{Aut}
\DeclareMathOperator\Inn{Inn}

% set page count index to begin from 1
\setcounter{page}{1}

% theorem/corollary/lemma/definition/etc.
\newtheorem{theorem}{Theorem}[section]
\newtheorem{corollary}{Corollary}[theorem]
\newtheorem{lemma}[theorem]{Lemma}

\theoremstyle{remark}
\newtheorem*{note}{Note}

\theoremstyle{definition}
\newtheorem{definition}{Definition}[section]

\theoremstyle{remark}
\newtheorem*{remark}{Remark}

\theoremstyle{definition}
\newtheorem*{example}{Example}

\theoremstyle{remark}
\newtheorem*{exercise}{Exercise}

% sections starting from zero
\setcounter{section}{-1}

\begin{document}
\section{Preliminaries}

\subsection*{Properties of Integers}
\begin{theorem}[Well Ordering Principle] Every non-empty subset of the positive integers has a least element.
\end{theorem}

\begin{theorem}[Division Algorithm]
  Let $a,b\in\mathbb{Z}$ with $b>0$. Then there exist unique integers $q,r$ such that $$a = bq+r\text{ and } 0\leq r < b.$$
\end{theorem}

\begin{example}\ \\
  (i) $a = 13, b=5$. We have that $13 = 5\cdot 2 + 3$. \\
  (ii) $a = -13, b=5$. We have that $-13 = 5\cdot (-3) + 2$.
\end{example}

\textit{Sketch of Proof of Theorem.} Let $a, b \in\mathbb{Z}$ and $b>0$. let $S = \cbr{a-bk\mid k\in\mathbb{Z}, a-bk\geq 0}$.\\
\textit{Case 1:} $0\in S$. Then $a-bk = 0$ for some $k\in\mathbb{Z}$ and $a = bk$ and $a = bq+r$ where $q=k$ and $r = 0$.\\
\textit{Case 2:} $0\notin S$. Then S is a subset of the positive integers. \begin{exercise} Show $S$ is nonempty. \end{exercise}

By the Well Ordering Principle, $S$ has a least element. Let $r$ be this least element. Then $r =a-bq$ for some $q\in\mathbb{Z}$. Show $r<b$, and show uniqueness of $q$ and $r$.

\begin{definition}
  Let $a,b\in \mathbb{Z}$. We say $a$ divides $b$ and write $a\divides b$ if $b = ac$ for some $c\in\mathbb{Z}$. We say $a$ is a divisor of $b$.
\end{definition}

\begin{example}
  $8\divides 24$ since $24 = 8\cdot 3$ and $3\in\mathbb{Z}$.
\end{example}

\begin{definition}
  Let $a,b\in\mathbb{Z}$ where $a,b$ are not both zero. Then the greatest common divisor of $a$ and $b$, denoted by $\gcd(a,b)$, is the largest integer $d$ such that $d\divides a$ and $d\divides b$.

  \begin{note}
    In number theory $\gcd(a,b)$ is denoted by $(a,b)$.
  \end{note}
\end{definition}

\begin{example}
  $\gcd(8,60) = 4$.
\end{example}

\begin{theorem}
  Let $a,b\in\mathbb{Z}$, where $a,b$ are not both zero. Then $\gcd(a,b) = \min\{ax+by\mid x,y\in\mathbb{Z}, ax+by> 0\}$
\end{theorem}

\textit{Sketch of Proof.} Let $a,b\in\mathbb{Z}$ not both be zero. Then let $S = \cbr{as+bt\mid s,t\in\mathbb{Z}, as+bt>0}$. See that $S$ is nonempty because $a,-a,b,-b\in S$. \\
So $S$ is a nonempty set of positive integers. By the Well Ordering Principle, $S$ has a least element $d$. We show $d = \gcd(a,b)$. \\
So $d = as+bt$ for some $s,t\in\mathbb{Z}$. Then by the Division Algorithm, $a = dq+r$ fo some integers $q,r$ where $0\leq r < d$. \\




\section{previous chapters TODO}
\section{previous chapters TODO}

\newpage
\section{Cyclic Groups}

Let $a\in G$ where $G$ is a group. The cyclic group generated by $a$ is $$\abr{a} \coloneqq \cbr{a^n : n\in\mathbb{Z}}. $$

\begin{theorem}
    Let $G$ be a group and suppose $a\in G$. \\ (1) If $a$ has infinite order then $a^i = a^j$ ($i,j\in\mathbb{Z}$) if and only if $i=j$. \\ (2) If $a$ has finite order $n$, then $$\abr{a} = \cbr{e,a^1,a^2,\dots,a^{n-1}},$$ and $a^i = a^j$ ($i,j\in\mathbb{Z}$) if and only if $i \equiv j \pmod n$
\end{theorem}

\begin{proof}
    Let $G$ be a group and suppose $a\in G$.
    
    (1) Suppose $a$ has infinite order. If $n$ is a positive integer then $a^n \neq e$. Suppose $a^i = a^j$ where $i,j\in\mathbb{Z}$, and without loss of generality, $i\leq j$. Then $a^{j-i} = a^j\br{a^i}^-1 = a^i\br{a^i}^-1 = e$. It follows that $j-i = 0$ since $j-i\in\mathbb{Z}$ and $j-i\geq 0$, so $i=j$.

    Conversely, if $i=j$, then $a^i = a^j$.

    (2) Suppose $a$ has finite order $n$.

    Case 1: $n = 1$. Then $a^1 = a = e$, so $$\abr{a} = \cbr{e^n : n\in\mathbb{Z}} = \cbr{e}.$$ Note that 
\end{proof}

\section{Isomorphisms?}
\dots

\begin{example}
  Find $\Aut(\mathbb{Z}_8)$.

  Suppose $\alpha\in\Aut(\mathbb{Z}_8)$. Then $\alpha : \mathbb{Z}_8\to \mathbb{Z}_8$ is an isomorphism. \begin{align*}
    &x = x\cdot 1 &|x| = \frac{8}{\gcd(x,8)}\\
    &0 &1 \\
    &1 &8 \\
    &2 &4 \\
    &3 &8 \\
    &4 &2 \\
    &5 &8 \\
    &6 &4 \\
    &7 &8 
  \end{align*}
  $\abs{1} = 8$ and $1$ is a generator of $\mathbb{Z}_8$.
  
  By a previous theorem, $\abs{\alpha(1)} = 8$ and hence $\alpha(1) = 1,3,5,$ or $7$.

  Let $x\in\mathbb{Z}_8$. Then $x = x\cdot 1 = \overbrace{1+\cdots+ 1}^{x\text{-times}}$. So $\alpha(x) = \alpha(1) + \cdots + \alpha(1) = x\alpha(1)$.

  The automorphism $\alpha$ is completely determined by the value of $\alpha(1)$.

  For $j = 1,3,5,$ or $7$, we define $\alpha_j : \mathbb{Z}_8 \to \mathbb{Z}_8$ by $\alpha_j(x) = xj \pmod{8} = \overbrace{j+\cdots+ j}^{x\text{-times}}\pmod{8}$.

  We show that each $\alpha_j$ is an automorphism of $\mathbb{Z}_8$. Clearly each $\alpha_j$ is well-defined. Let $j = 1,3,5,$ or $7$. Suppose $x_1,x_2\in\mathbb{Z}_8$ and $\alpha_j(x_1) = \alpha_j(x_2)$. Then $jx_1 \equiv jx_2 \pmod{8}$.

  Observe that $j\in\cbr{1,3,5,7} = U(8)$. The operation in $U(8)$ is multiplication mod $8$. Each $j$ has a multiplicative inverse $\bar{j} \pmod{8}$, i.e. $\bar{j}j \equiv 1 \pmod{8}$.

  In this example, $\bar{j} = j$. Then \begin{align*}
    \bar{j}\br{jx_1} &\equiv \bar{j}\br{jx_2}\pmod{8} \\
    \br{\bar{j}j}x_1 &\equiv \br{\bar{j}j}x_2\pmod{8} \\
    1x_1 &\equiv 1x_2\pmod{8} \\
    x_1 &\equiv x_2\pmod{8}
  \end{align*} so that $x_1=x_2$ in $\mathbb{Z}_8$. So $\alpha_j$ is one-to-one. 

  Let $y\in\mathbb{Z}_8$. Then $\alpha_j(\bar{j}y) = j\br{\bar{j}y} \pmod{8} = \br{j\bar{j}}y \pmod{8} = y \pmod{8}$. Then $\alpha_j(\bar{j}y) = y$, so $\alpha_j$ is onto.

  Then \begin{align*}
    \alpha_j(x_1+x_2) &= j(x_1+x_2) \pmod{8} \\
    &= \br{jx_1+jx_2}\pmod{8}\\
    &= \br{jx_1\pmod{8}} + \br{jx_1\pmod{8}} \\
    &= \alpha_j(x_1) + \alpha_j(x_2)
  \end{align*}
  So $\alpha_j$ preserves the group operation and $\alpha_j$ is an automorphism.
  \begin{note}
    $\alpha_1, \alpha_3, \alpha_5, \alpha_7$ are the automorphisms of $\mathbb{Z}_8$, i.e. $$\Aut{\mathbb{Z}_8} = \cbr{\alpha_1, \alpha_3, \alpha_5, \alpha_7}.$$
  \end{note}
\end{example}

\begin{note}
  $\Aut{\mathbb{Z}_8} \approx U(8)$.
\end{note}
\begin{proof}
  Define $T : \cbr{\alpha_1, \alpha_3, \alpha_5, \alpha_7} \to U(8)$ by $T(\alpha_j) = j$. So \begin{align*}
    \alpha_1 &\to 1 \\
    \alpha_3 &\to 3 \\
    \alpha_5 &\to 5 \\
    \alpha_7 &\to 7 
  \end{align*}
  T is clearly well-defined, one-to-one, and onto. Let $i,j\in U(8)$. Suppose $ij=k$ in $U(8)$, i.e. $ij \equiv k \pmod{8}$.

  Then $$T(\alpha_i\circ \alpha_j) = T(\alpha_k) = k$$ $$= i\cdot j \pmod{8} = T(\alpha_i)T(\alpha_j)\pmod{8}$$ since $\alpha_i\circ\alpha_j = \alpha_k$, which is true because $$(\alpha_i\circ\alpha_j)(x) \equiv \alpha_i(\alpha_j(x)) \equiv i(jx) \pmod{8}$$ $$\equiv (ij)x \equiv kx \pmod{8} = \alpha_k(x)$$ for any $x\in U(8)$. Hence $T$ is an isomorphism and so $\Aut(\mathbb{Z}_8)\approx U(8)$. 
\end{proof}

Similarly, we have the following theorem:
\begin{theorem}
  $\Aut(\mathbb{Z}_n)\approx U(n)$.
\end{theorem}

\begin{example}
  Suppose $\Phi : \mathbb{Z}_8 \to \mathbb{Z}_8$ is an automorphism and $\Phi(5) = 7$. Find a formula for $\Phi(x)$.

  \textit{Hint.} $5\cdot5 = 1$ in $\mathbb{Z}_8$.

  To find a formula for $\Phi(x)$ we only need to find $\Phi(1)$ since $\Phi(x) = \Phi(x\cdot 1) = x\Phi(1)$. So from the hint given we find that $\Phi(1) = \Phi(5\cdot 5) = 5\Phi(5) = 5\cdot 7 = 35 \equiv 3 \pmod{8}$. So $\Phi(1) = 3$, which means $\Phi(x) = x\Phi(1) = x\cdot3$.

  Hence $\Phi(x) = 3x \pmod{8}$.
\end{example}

\section{Cosets and Lagrange's Theorem (ch7)}

\begin{definition}
  Let $H$ be a subset of a group $G$. Let $a\in G$. Define $$aH = \cbr{ah : h\in H}$$ and $$Ha = \cbr{ha : h\in H}.$$ When $H$ is a subgroup of $G$ the set $aH$ is called the left coset of $H$ in $G$ containing $a$, and $Ha$ is called the right coset of $H$ in $G$ containing $a$.
\end{definition}

\begin{example}
  Let $H = \cbr{e,\pmt*{(12)}} = \abr{\pmt*{(12)}}$ and $G = S_3 = \cbr{e,\pmt*{(12)}, \pmt*{(13)}, \pmt*{(23)}, \pmt*{(123)}, \pmt*{(132)} }$.

  Then \begin{align*}
    eH &= H = \cbr{e,\pmt*{(12)}}\\
    \pmt*{(12)}H &= \cbr{\pmt*{(12)},e} = H\\ 
    \pmt*{(13)}H &= \cbr{\pmt*{(13)}, \pmt*{(13)}\pmt*{(12)}} = \cbr{\pmt*{(12)}, \pmt*{(123)}} \\
    \pmt*{(23)}H &= \cbr{}
  \end{align*}
\end{example}

Testing the git repository updating mechanism.
hopefully this realizes that i did change something.

\end{document}