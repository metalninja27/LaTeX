\documentclass[12pt]{amsart}

\textwidth = 6.2 in
\textheight = 8.5 in
\oddsidemargin = 0.0 in
\evensidemargin = 0.0 in
\topmargin = 0.0 in
\headheight = 0.0 in
\headsep = 0.3 in
\parskip = 0.05 in
\parindent = 0.3 in

\usepackage{enumerate}
\usepackage{amsmath}
\usepackage{color}
\def\cc{\color{blue}}
\usepackage[normalem]{ulem}
\usepackage{amsfonts, amsmath, amssymb, amsthm}
\usepackage{systeme}
\usepackage[none]{hyphenat}
\usepackage{graphicx}
\graphicspath{{./images/}}
\usepackage{esint}
\usepackage{cancel}
\usepackage{physics}

\title{Homework 11}
\author{Sai Sivakumar}

\newtheorem{theorem}            {Theorem}[section]
\newtheorem{proposition}        [theorem]{Proposition}

\newcommand{\RR}{\mathbb{R}}

\begin{document}
\maketitle

Suppose $f:[0,1]\to\mathbb{R}$ is bounded, $f(x)\ge 0$ for all $x\in [0,1]$ and there 
 is point $a\in (0,1)$ such that $f(a)>0.$ Show, if $f$ is continuous at $a,$
 then  the lower integral
 of $f$ is positive.  Conclude, if $f$ is Riemann integrable, then
\[
 \int_0^1 f\, dx >0.
\]
I spoke with Jude Flynn about a few details of the proof.
%[Suggestion: Find  an $0<\eta$ such that, with  $P$ is the partition of $[0,1]$ given by \[P=\{0=x_0 < a-\eta=x_1 < a+\eta=x_2 < 1=x_3\},\] the lower sum $L(f,P)$ is positive; e.g., $L(f,P)\ge \eta f(a).$]
\begin{proof}
Let $f$ be as given. Because $f$ is continuous at $a$, there exists $0 < \eta< \min\{a, 1-a\}$ such that if $\abs{x-a} < \eta$, then $\abs{f(x) - f(a)} < f(a)/2$; that is, $0 < f(a)/2 < f(x) < 3f(a)/2$.

With the partition $P = \{0=x_0 < a-\eta=x_1 < a+\eta=x_2 < 1=x_3\}$, \begin{align*}
    0 < \eta f(a) &\leq (a-\eta) \inf\{f(x)\colon x\in[x_0,x_1]\} + 2\eta \inf\{f(x)\colon x\in[x_1,x_2]\}\\
    &\hspace*{7cm}+ (1-a-\eta)\inf\{f(x)\colon x\in[x_2,x_3]\}\\
    &= L(f,P) \\
    &\leq \lefteqn{\int_0^1 f}\lefteqn{\hspace{0.0ex}\rule[-2.25ex]{1ex}{.05ex}}
    \phantom{\int_0^1 f}\dd{x}.
\end{align*}
If $f$ is Riemann integrable, it follows that
\[0 < \lefteqn{\int_0^1 f}\lefteqn{\hspace{0.0ex}\rule[-2.25ex]{1ex}{.05ex}}
\phantom{\int_0^1 f}\dd{x} \leq \int_0^1 f\dd{x}\]
as desired.
\end{proof}
\end{document}