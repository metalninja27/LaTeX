\documentclass[12pt]{amsart}

\textwidth = 6.2 in
\textheight = 8.5 in
\oddsidemargin = 0.0 in
\evensidemargin = 0.0 in
\topmargin = 0.0 in
\headheight = 0.0 in
\headsep = 0.3 in
\parskip = 0.05 in
\parindent = 0.3 in

\usepackage{enumerate}
\usepackage{amsmath}
\usepackage{color}
\def\cc{\color{blue}}
\usepackage[normalem]{ulem}
\usepackage{amsfonts, amsmath, amssymb, amsthm}
\usepackage{systeme}
\usepackage[none]{hyphenat}
\usepackage{graphicx}
\graphicspath{{./images/}}
\usepackage{esint}
\usepackage{cancel}
\usepackage{physics}

\title{Homework 9}
\author{Sai Sivakumar}

\newtheorem{theorem}            {Theorem}[section]
\newtheorem{proposition}        [theorem]{Proposition}

\newcommand{\RR}{\mathbb{R}}

\begin{document}
\maketitle

 Given $L\in \mathbb{R}$  and  $f:[0,\infty)\to \mathbb{R},$
  the function  \emph{$f$ has limit $L$ at infinity,}   written,
 \begin{equation*}
   \lim_{x\to \infty}f(x)=L,
 \end{equation*}
  if for every $\epsilon>0$ there is a $C>0$ such that
  if $x>C$, then $|f(x)-L| <\epsilon.$

  Prove if $f:[0,\infty)\to \mathbb{R}$ is continuous, $L\in \mathbb{R}$  and has
  limit $L$  at infinity, then $f$ is uniformly continuous.

  I worked with Jude Flynn, Nicholas Kapsos, Elaine Danielson, and Silas Rickards to come up with the idea for the proof.

\begin{proof}
    Let $f\colon [0,\infty)\to \mathbb{R}$ be continuous with limit $L\in \mathbb{R}$ at infinity as given.

    Given $\varepsilon>0$, because $f$ has limit $L$ at infinity, there exists $C>0$ such that if $z>C$, then $\abs{f(z)-L}< \varepsilon/2$.

    The function $f$ is continuous on $[0,\infty)$, and so $f$ is continuous on the nonempty compact interval $[0,C+1]$. Hence $f$ is uniformly continuous on $[0,C+1]$. Therefore when $x,y\in[0,C+1]$, we can choose $0 < \delta < 1$ such that if $\abs{x-y}< \delta$, then $\abs{f(x)-f(y)}< \varepsilon$.

    When $x,y$ are not both in $[0,C+1]$ (with $x,y\in [0,\infty)$) and $\abs{x-y}<\delta$, it follows that both $x,y > C$ since $\delta < 1$ (without loss of generality, if $x\not\in [0,C+1]$, then $C+1< x$, which implies $C+1-\delta < y$). Thus the only other scenario which remains is to consider when $x,y > C$. When $x,y>C$, \begin{align*}
        \abs{f(x)-f(y)} &= \abs{f(x)-L+L-f(y)}\\
        &\leq \abs{f(x)-L} + \abs{L-f(y)}\\
        &< \varepsilon/2 + \varepsilon/2 = \varepsilon.
    \end{align*} Note that in this case we did not use the hypothesis that $\abs{x-y}< \delta$. Hence $f$ is uniformly continuous on $[C,\infty)$. Thus $f$ is uniformly continuous on $[0,C+1]$ and $[C,\infty)$.
    
    It follows that given $\varepsilon>0$, there exists $\delta >0$ such that for any $x,y\in [0,\infty)$, if $\abs{x-y}< \delta$, then $\abs{f(x)-f(y)} < \varepsilon$ (in particular the choice of $\delta>0$ is the one made in the first case). Hence $f$ is uniformly continuous on $[0,\infty)$.
\end{proof}
\end{document}